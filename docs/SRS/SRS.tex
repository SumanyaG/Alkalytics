% THIS DOCUMENT IS FOLLOWS THE VOLERE TEMPLATE BY Suzanne Robertson and James Robertson
% ONLY THE SECTION HEADINGS ARE PROVIDED
%
% Initial draft from https://github.com/Dieblich/volere
%
% Risks are removed because they are covered by the Hazard Analysis
\documentclass[12pt]{article}

\usepackage{booktabs}
\usepackage{tabularx}
\usepackage{hyperref}
\hypersetup{
    bookmarks=true,         % show bookmarks bar?
      colorlinks=true,      % false: boxed links; true: colored links
    linkcolor=red,          % color of internal links (change box color with linkbordercolor)
    citecolor=green,        % color of links to bibliography
    filecolor=magenta,      % color of file links
    urlcolor=cyan           % color of external links
}

\newcommand{\lips}{\textit{Insert your content here.}}

%% Comments

\usepackage{color}

\newif\ifcomments\commentstrue %displays comments
%\newif\ifcomments\commentsfalse %so that comments do not display

\ifcomments
\newcommand{\authornote}[3]{\textcolor{#1}{[#3 ---#2]}}
\newcommand{\todo}[1]{\textcolor{red}{[TODO: #1]}}
\else
\newcommand{\authornote}[3]{}
\newcommand{\todo}[1]{}
\fi

\newcommand{\wss}[1]{\authornote{blue}{SS}{#1}} 
\newcommand{\plt}[1]{\authornote{magenta}{TPLT}{#1}} %For explanation of the template
\newcommand{\an}[1]{\authornote{cyan}{Author}{#1}}

%% Common Parts

\newcommand{\progname}{Software Engineering} % PUT YOUR PROGRAM NAME HERE
\newcommand{\authname}{Team 21, Alkalytics
\\ Sumanya Gulati - gulats10
\\ Kate Min - mink9
\\ Jennifer Ye - yej52
\\ Jason Tran - tranj78} % AUTHOR NAMES                  

\usepackage{hyperref}
    \hypersetup{colorlinks=true, linkcolor=blue, citecolor=blue, filecolor=blue,
                urlcolor=blue, unicode=false}
    \urlstyle{same}
                                


\begin{document}

\title{Software Requirements Specification for \progname: Alkalytics} 
\author{\authname}
\date{\today}
	
\maketitle

~\newpage

\pagenumbering{roman}

\tableofcontents

~\newpage

\section*{Revision History}

\begin{tabularx}{\textwidth}{p{3cm}p{2cm}X}
\toprule {\textbf{Date}} & {\textbf{Version}} & {\textbf{Notes}}\\
\midrule
Date 1 & 1.0 & Notes\\
Date 2 & 1.1 & Notes\\
\bottomrule
\end{tabularx}

~\\

~\newpage
\section{Purpose of the Project}
\subsection{User Business}
\lips
\subsection{Goals of the Project}
\lips
\section{Stakeholders}
\subsection{Client}
\lips
\subsection{Customer}
\lips
\subsection{Other Stakeholders}
\lips
\subsection{Hands-On Users of the Project}
\lips
\subsection{Personas}
\lips
\subsection{Priorities Assigned to Users}
\lips
\subsection{User Participation}
\lips
\subsection{Maintenance Users and Service Technicians}
\lips

\section{Mandated Constraints}
\subsection{Solution Constraints}
\textbf{Description}: The product must accept Comma-Separated Value (CSV) files as input.\\
\textbf{Rationale}: The lab apparatus generates and stores results as CSV files.\\
\textbf{Fit Criterion}: The product's input process (the processing and acceptance of input data) into the database shall be approved by testers and developers.

\subsection{Implementation Environment of the Current System}
\textbf{Description}: The product must be able to run on a Windows machine.\\
\textbf{Rationale}: Currently, the lab has a Windows machine that is used to operate the machine and analyse the produced results.\\
\textbf{Fit Criterion}: The product shall be approved as Windows compliant by testers and developers.

\subsection{Off-the-Shelf Software}
\textbf{Description}: \href{https://www.mongodb.com/}{MongoDB} - a document-oriented, NoSQL database product shall be used to store the datapoints.\\
\textbf{Rationale}: Using an existing, verstaile and scalable solution like MongoDB that does not use SQL and is thus, non-relational, will allow greater flexibility in storing datapoints.\\

\subsection{Anticipated Workplace Environment}
\textbf{Description}: The product shall be used in the Chemical Engineering Lab run by Dr. Charles de Lannoy and Bassel Abdelkader.

\subsection{Partner or Collaborative Applications}
\textbf{Description}: The product shall be used in collaboration with the \emph{name of lab software}.\\
\textbf{Rationale}: The \emph{name of lab software} is used to retrieve data from the lab apparatus. The retrieved data shall be used as input for the product.

\subsection{Schedule Constraints}
\textbf{Description}: The project must be finished within the course of the current academic year.\\
\textbf{Rationale}: The finished product, as outlined in the project requirements, must be submitted by the end of the academic year.\\
\newline
A few relevant deadlines include:
\begin{itemize}
  \item Proof of Concept Demonstration: November 11 to 22, 2024
  \item Revision 0 Demonstration: February 3 to 14, 2025
  \item Final Demonstration (Revision 1): March 24 to 30, 2025
\end{itemize}

\subsection{Budget Constraints}
\textbf{Description}: The total cost of the project must not exceed \$750.\\
\textbf{Rationale}: The product must be economically feasible and all teams must have an equal budget to ensure conformity and equality in terms of access of resources.

\subsection{Enterprise Constraints}
\emph{N/A}

\section{Naming Conventions and Terminology}
\subsection{Glossary of All Terms, Including Acronyms, Used by Stakeholders
involved in the Project}
\lips
\section{Relevant Facts And Assumptions}
\subsection{Relevant Facts}
Currently, two sources of data input are used -
\begin{itemize}
  \item The CSV files that contain datapoints generated by the apparatus.
  \item The initial parameter values for each experiment such as power voltage, age of membrane, density module and more. These values are manually inputted by the user 
  and remain constant throughout each experiment.
\end{itemize}

\subsection{Business Rules}
\emph{N/A}
\subsection{Assumptions}
\emph{N/A}

\section{The Scope of the Work}
\subsection{The Current Situation}
\lips
\subsection{The Context of the Work}
\lips
\subsection{Work Partitioning}
\lips
\subsection{Specifying a Business Use Case (BUC)}
\lips

\section{Business Data Model and Data Dictionary}
\subsection{Business Data Model}
\lips
\subsection{Data Dictionary}
\lips

\section{The Scope of the Product}
\subsection{Product Boundary}
\lips
\subsection{Product Use Case Table}
\lips
\subsection{Individual Product Use Cases (PUC's)}
\lips

\section{Functional Requirements}
\subsection{Data Input Requirements}
  \begin{enumerate}
    \item[\textbf{FR-1.}] The system shall allow the user to input new experiment data or parameters.
    \begin{itemize}
    \item \textit{Rationale:} The system needs to be kept up-to-date with ongoing experiments, which may include new parameters that did not exist previously.
    \item \textit{Fit Criterion:} The user should be able to input new data and parameters with 0 errors.
    \end{itemize}
    \item[\textbf{FR-2.}] The system shall store experiment data in the database with all associated parameters and values correctly labelled.
    \begin{itemize}
      \item \textit{Rationale:} Ensures that data retrieval and analysis will be correct and accurate.
      \item \textit{Fit Criterion:} The system database parameters and values shall match the original experiment data parameters and values.
    \end{itemize}
  \end{enumerate}

\subsection{Data Migration and Organization Requirements}
  \begin{enumerate}
    \item[\textbf{FR-3.}] The system shall read existing experiment data stored in .CSV files.
    \begin{itemize}
      \item \textit{Rationale:} Existing experiment data is stored in Excel spreadsheets and must be integrated into the new system for continuity and analysis.
      \item \textit{Fit Criterion:} The system shall read and import the data files with 0 errors.
    \end{itemize}
    \item[\textbf{FR-4.}] The system shall organize experiment data by timestamps and experiment ID for unique identification.
    \begin{itemize}
      \item \textit{Rationale:} Each experiment needs to be separately identified for quick retrieval of data and efficiency in search or query actions.
      \item \textit{Fit Criterion:} Each ID and timestamp shall be traceable to one experiment.
    \end{itemize}
  \end{enumerate}

\subsection{Data Search and Query Requirements}
\begin{enumerate}
  \item[\textbf{FR-5.}] The system shall allow the user to search for specific datasets based on different parameters.
  \begin{itemize}
    \item \textit{Rationale:} Allows for quick look-ups of certain experiments and their results.
    \item \textit{Fit Criterion:} The system shall retrieve the correct experiments based on the matching parameters.
  \end{itemize}
  \item[\textbf{FR-6.}] The system shall allow the user to query two or more parameters or datasets for comparison and analysis.
  \begin{itemize}
    \item \textit{Rationale:} Allows for direct comparisons between different experiment parameters and/or results, which is necessary for analysis.
    \item \textit{Fit Criterion:} The system shall retrieve the correct parameters and/or experiments based on the query inputs.
  \end{itemize}
  \item[\textbf{FR-7.}] The system shall display the results of a user’s selected search or query in a format that is readable to the user.
  \begin{itemize}
    \item \textit{Rationale:} The user needs to see the results in a format that they can interpret.
    \item \textit{Fit Criterion:} The results shall be displayed in a table with all labels correct and legible.
  \end{itemize}
\end{enumerate}

\subsection{Data Visualization Requirements}
\begin{enumerate}
  \item[\textbf{FR-8.}] The system shall generate visual graphs based on selected parameters and datasets.
  \begin{itemize}
    \item \textit{Rationale:} Visual representation of the data allows for easy interpretation and graphical analysis.
    \item \textit{Fit Criterion:} The result should display a graphical plot with a title, axes, labels, and a legend.
  \end{itemize}
  \item[\textbf{FR-9.}] The system shall allow the user to customize the data visualization by adjusting axes, data ranges, labels, etc.
  \begin{itemize}
    \item \textit{Rationale:} Allows the user to adjust the graphical representation to their needs for their analysis.
    \item \textit{Fit Criterion:} Modifications to axes, data ranges, labels should be reflected in the generated graph in real-time.
  \end{itemize}
\end{enumerate}

\subsection{Data Analysis Requirements}
\begin{enumerate}
    \item[\textbf{FR-10.}] The system shall analyze patterns and trends in the experiment data based on the user’s selected parameters.
    \begin{itemize}
      \item \textit{Rationale:} Trend analysis is critical for the user to discover important findings pertaining to the experiment.
      \item \textit{Fit Criterion:} The system shall generate a result of the analysis to display to the user.
    \end{itemize}
    \item[\textbf{FR-11.}] The system shall use machine learning algorithms to predict and interpolate the data.
    \begin{itemize}
      \item \textit{Rationale:} Allows for future predictions of data and efficiency in running future experiments.
      \item \textit{Fit Criterion:} The system shall generate a report of value predictions or interpolate a graph and provide the interpolated data points.
    \end{itemize}
\end{enumerate}

\subsection{Error Tracking Requirements}
This section outlines functional requirements for one of the project's stretch goals.
\begin{enumerate}
  \item[\textbf{FR-12.}] The system shall track and log errors in the experiment data.
  \begin{itemize}
    \item \textit{Rationale:} Helps users identify irrelevant or missing parameters.
    \item \textit{Fit Criterion:} Missing values from input data should be flagged.
  \end{itemize}
  \item[\textbf{FR-13.}] The system shall remove data logged as errors.
  \begin{itemize}
    \item \textit{Rationale:} Ensures data is organized and produce accurate results in analysis.
    \item \textit{Fit Criterion:} Flagged data should be removed from the database after user confirmation.
  \end{itemize}
\end{enumerate}

\subsection{User Access Management Requirements}
This section outlines functional requirements for one of the project's stretch goals.
\begin{enumerate}
  \item[\textbf{FR-14.}] The system shall allow the user to sign in with valid credentials.
  \begin{itemize}
    \item \textit{Rationale:} Ensures the data can only be accessed and modified by authorized users.
    \item \textit{Fit Criterion:} The user shall be able to log in with a username and password.
  \end{itemize}
\end{enumerate}

\subsection{Data Export Requirements}
This section outlines functional requirements for one of the project's stretch goals.
\begin{enumerate}
  \item[\textbf{FR-15.}] The system shall generate a report of queries in a session for the user to save or download.
  \begin{itemize}
    \item \textit{Rationale:} Allows user to keep a record of their findings for future use or reference.
    \item \textit{Fit Criterion:} The report should be exported in CSV or PDF format.
  \end{itemize}
\end{enumerate}

\section{Look and Feel Requirements}
\subsection{Appearance Requirements}
\lips
\subsection{Style Requirements}
\lips

\section{Usability and Humanity Requirements}
\subsection{Ease of Use Requirements}
\textbf{Description}: The product must be easy to navigate and use for individuals with basic computer literacy.\\
\textbf{Rationale}: The product must be user-friendly. In the context of this project, basic computer literacy is defined to encompass five computer skills - using a keyboard
to type, using a mouse to navigate, understanding basic software applications such as word processing and spreadsheets, browsing the internet, and managaing files and folders.\\ 
\textbf{Fit Criterion}: An individual with basic computer literacy must be able to launch the application and upload an input file without any assistance from the administrator.

\subsection{Personalization and Internationalization Requirements}
\textbf{Description}: The current version of the product will only be available in English (EN-US) and more languages can be added in the later versions.\\
\textbf{Rationale}: Currently, the product is only expected to be used by McMaster faculty and staff who are fluent in English.\\
\newline
\textbf{Description}: The product must recognize commonly used scientific and mathematical symbols.\\
\textbf{Rationale}: The product shall be used to store scientific parameters as datapoints so the product must be able to recognize commonly used symbols used to specify scientific properties.\\
\textbf{Fit Criterion}: The product must be able to recognize the uppercase and lowercase Greek Alphabet.

\subsection{Learning Requirements}
\textbf{Description}: Users must be able to use the product without any formal training and with minimal guidance.\\
\textbf{Rationale}: The product shall be intuitive to use. Users must be able to freely naviagte and experiment with the product after a simple product walkthrough.\\
\textbf{Fit Criterion}: A new user with basic computer literacy skills should be able to upload an input file, enter initial experiment parameters, select fields to be compared and view their graph
after a simple product walkthrough by the administrator.

\subsection{Understandability and Politeness Requirements}
\emph{N/A}

\subsection{Accessibility Requirements}
\emph{N/A}

\section{Performance Requirements}
\subsection{Speed and Latency Requirements}
\begin{enumerate}
\item The system shall store new data or parameters within 60 seconds of input.
\item The system shall retrieve data from the database within 50ms for typical search and queries.
\item The interaction between the interface and the user shall have a maximum response time of 2 seconds.
\item The system shall have a maximum latency of 2 seconds for typical search and queries.
\item The system shall generate a visualization of the data within 5 seconds.
\end{enumerate}
\begin{itemize}
  \item \textit{Rationale:} Quick response times ensure efficiency and smooth user experience without disrupting the flow of the user's thought processes.
  \item \textit{Fit Criterion:} The system shall satisfy the requirements above.
\end{itemize}

\subsection{Safety-Critical Requirements}
The product does not have safety-critical requirements to consider.

\subsection{Precision or Accuracy Requirements}
\begin{enumerate}
  \item All parameter values shall be accurate to four decimal places.
  \item All timestamps of experiment data shall be accurate to milliseconds. 
  \item Values on visual data plots shall be accurate to four decimal places.
\end{enumerate}
\begin{itemize}
  \item \textit{Rationale:} Accuracy of the data is critical for data analysis, prediction, and interpolation.
  \item \textit{Fit Criterion:} The system shall satisfy the requirements above.
\end{itemize}

\subsection{Robustness or Fault-Tolerance Requirements}
\begin{enumerate}
  \item The application shall not terminate but display an error message if it loses connection to the backend server.
  \item The application shall provide basic functionality if it loses connection to the internet.
\end{enumerate}
\begin{itemize}
  \item \textit{Rationale:} The system should not fail or crash when experiencing unexpected circumstances.
\end{itemize}

\subsection{Capacity Requirements}
\begin{enumerate}
  \item The application shall allow for up to three simultaneous users.
  \item The system shall store up to x amount of data.
\end{enumerate}
\begin{itemize}
  \item \textit{Rationale:} The system must be capable of storing and processing large amounts of data.
  \item \textit{Fit Criterion:} The system shall satisfy the requirements above.
\end{itemize}

\subsection{Scalability or Extensibility Requirements}
\begin{enumerate}
  \item The system shall be able to process and store the existing data. The amount of data going into the system is expected to grow until the experiment study comes to an end.
  \item The system shall be able to add additional parameters that did not previously exist in the database at the discretion of the user.
\end{enumerate}
\begin{itemize}
  \item \textit{Rationale:} The system must be able to expand to keep up with future experiments.
\end{itemize}

\subsection{Longevity Requirements}
\begin{enumerate}
  \item The system shall operate for the duration of the experiment study.
\end{enumerate}

\section{Operational and Environmental Requirements}
\subsection{Expected Physical Environment}
\begin{enumerate}
  \item The application shall operate in a typical office environment with reliable internet connectivity.
  \item The application shall be compatible with a desktop or laptop environment.
\end{enumerate}
\begin{itemize}
  \item \textit{Rationale:} Ensures functionality in environments where end-users are most likely to use the application, accomodating several screen sizes and operating systems. 
  \item \textit{Fit Criterion:} Testing will be conducted on the two most common operating systems, Windows and macOS.
\end{itemize}

\subsection{Wider Environment Requirements}
\lips

\subsection{Requirements for Interfacing with Adjacent Systems}
\begin{enumerate}
  \item The application shall operate on the most recent versions of Google Chrome and Apple Safari.
\end{enumerate}
\begin{itemize}
  \item \textit{Rationale:} The application must be able to operate on these two most common web browsers, as these will be the primary platforms where it is hosted and accessed by users.
  \item \textit{Fit Criterion:} Performance testing shall be done to ensure the application functions correctly.
\end{itemize}

\subsection{Productization Requirements}
\begin{enumerate}
  \item The system shall be distributed as a web application.
  \item The system shall have an easy onboarding process with user documentation.
  \begin{itemize}
    \item \textit{Rationale:} Ensures that users can use the application without needing frequent support.
    \item \textit{Fit Criterion:} Usability testing shall be done to ensure users are able to onboard easily.
  \end{itemize}
\end{enumerate}

\subsection{Release Requirements}
\begin{enumerate}
  \item The first version of the system shall be released after project completion.
\end{enumerate}

\section{Maintainability and Support Requirements}
\subsection{Maintenance Requirements}
\lips
\subsection{Supportability Requirements}
\lips
\subsection{Adaptability Requirements}
\lips

\section{Security Requirements}
\subsection{Access Requirements}
\lips
\subsection{Integrity Requirements}
\lips
\subsection{Privacy Requirements}
\lips
\subsection{Audit Requirements}
\lips
\subsection{Immunity Requirements}
\lips

\section{Cultural Requirements}
\subsection{Cultural Requirements}
\lips

\section{Compliance Requirements}
\subsection{Legal Requirements}
\lips
\subsection{Standards Compliance Requirements}
\lips

\section{Open Issues}
\lips

\section{Off-the-Shelf Solutions}
\subsection{Ready-Made Products}
\lips
\subsection{Reusable Components}
\lips
\subsection{Products That Can Be Copied}
\lips

\section{New Problems}
\subsection{Effects on the Current Environment}
\lips
\subsection{Effects on the Installed Systems}
\lips
\subsection{Potential User Problems}
\lips
\subsection{Limitations in the Anticipated Implementation Environment That May
Inhibit the New Product}
\lips
\subsection{Follow-Up Problems}
\lips

\section{Tasks}
\subsection{Project Planning}
\lips
\subsection{Planning of the Development Phases}
\lips

\section{Migration to the New Product}
\subsection{Requirements for Migration to the New Product}
\lips
\subsection{Data That Has to be Modified or Translated for the New System}
\lips

\section{Costs}
\lips
\section{User Documentation and Training}
\subsection{User Documentation Requirements}
\lips
\subsection{Training Requirements}
\lips

\section{Waiting Room}
\lips

\section{Ideas for Solution}
\lips

\newpage{}
\section*{Appendix --- Reflection}

The information in this section will be used to evaluate the team members on the
graduate attribute of Lifelong Learning.  Please answer the following questions:

\begin{enumerate}
  \item What knowledge and skills will the team collectively need to acquire to
  successfully complete this capstone project?  Examples of possible knowledge
  to acquire include domain specific knowledge from the domain of your
  application, or software engineering knowledge, mechatronics knowledge or
  computer science knowledge.  Skills may be related to technology, or writing,
  or presentation, or team management, etc.  You should look to identify at
  least one item for each team member.
  \item For each of the knowledge areas and skills identified in the previous
  question, what are at least two approaches to acquiring the knowledge or
  mastering the skill?  Of the identified approaches, which will each team
  member pursue, and why did they make this choice?
\end{enumerate}

\end{document}
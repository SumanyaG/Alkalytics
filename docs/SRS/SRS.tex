% THIS DOCUMENT IS FOLLOWS THE VOLERE TEMPLATE BY Suzanne Robertson and James Robertson
% ONLY THE SECTION HEADINGS ARE PROVIDED
%
% Initial draft from https://github.com/Dieblich/volere
%
% Risks are removed because they are covered by the Hazard Analysis
\documentclass[12pt]{article}

\usepackage{booktabs}
\usepackage{tabularx}
\usepackage{hyperref}
\hypersetup{
    bookmarks=true,         % show bookmarks bar?
      colorlinks=true,      % false: boxed links; true: colored links
    linkcolor=red,          % color of internal links (change box color with linkbordercolor)
    citecolor=green,        % color of links to bibliography
    filecolor=magenta,      % color of file links
    urlcolor=cyan           % color of external links
}

\newcommand{\lips}{\textit{Insert your content here.}}

%% Comments

\usepackage{color}

\newif\ifcomments\commentstrue %displays comments
%\newif\ifcomments\commentsfalse %so that comments do not display

\ifcomments
\newcommand{\authornote}[3]{\textcolor{#1}{[#3 ---#2]}}
\newcommand{\todo}[1]{\textcolor{red}{[TODO: #1]}}
\else
\newcommand{\authornote}[3]{}
\newcommand{\todo}[1]{}
\fi

\newcommand{\wss}[1]{\authornote{blue}{SS}{#1}} 
\newcommand{\plt}[1]{\authornote{magenta}{TPLT}{#1}} %For explanation of the template
\newcommand{\an}[1]{\authornote{cyan}{Author}{#1}}

%% Common Parts

\newcommand{\progname}{Software Engineering} % PUT YOUR PROGRAM NAME HERE
\newcommand{\authname}{Team 21, Alkalytics
\\ Sumanya Gulati - gulats10
\\ Kate Min - mink9
\\ Jennifer Ye - yej52
\\ Jason Tran - tranj78} % AUTHOR NAMES                  

\usepackage{hyperref}
    \hypersetup{colorlinks=true, linkcolor=blue, citecolor=blue, filecolor=blue,
                urlcolor=blue, unicode=false}
    \urlstyle{same}
                                


\begin{document}

\title{Software Requirements Specification for \progname: subtitle describing software} 
\author{\authname}
\date{\today}
	
\maketitle

~\newpage

\pagenumbering{roman}

\tableofcontents

~\newpage

\section*{Revision History}

\begin{tabularx}{\textwidth}{p{3cm}p{2cm}X}
\toprule {\textbf{Date}} & {\textbf{Version}} & {\textbf{Notes}}\\
\midrule
Date 1 & 1.0 & Notes\\
Date 2 & 1.1 & Notes\\
\bottomrule
\end{tabularx}

~\\

~\newpage
\section{Purpose of the Project}
\subsection{User Business}
\lips
\subsection{Goals of the Project}
\lips
\section{Stakeholders}
\subsection{Client}
\lips
\subsection{Customer}
\lips
\subsection{Other Stakeholders}
\lips
\subsection{Hands-On Users of the Project}
\lips
\subsection{Personas}
\lips
\subsection{Priorities Assigned to Users}
\lips
\subsection{User Participation}
\lips
\subsection{Maintenance Users and Service Technicians}
\lips

\section{Mandated Constraints}
\subsection{Solution Constraints}
\lips
\subsection{Implementation Environment of the Current System}
\lips
\subsection{Partner or Collaborative Applications}
\lips
\subsection{Off-the-Shelf Software}
\lips
\subsection{Anticipated Workplace Environment}
\lips
\subsection{Schedule Constraints}
\lips
\subsection{Budget Constraints}
\lips
\subsection{Enterprise Constraints}
\lips

\section{Naming Conventions and Terminology}
\subsection{Glossary of All Terms, Including Acronyms, Used by Stakeholders
involved in the Project}
\lips

\section{Relevant Facts And Assumptions}
\subsection{Relevant Facts}
\lips
\subsection{Business Rules}
\lips
\subsection{Assumptions}
\lips

\section{The Scope of the Work}
\subsection{The Current Situation}
\lips
\subsection{The Context of the Work}
\lips
\subsection{Work Partitioning}
\lips
\subsection{Specifying a Business Use Case (BUC)}
\lips

\section{Business Data Model and Data Dictionary}
\subsection{Business Data Model}
\lips
\subsection{Data Dictionary}
\lips

\section{The Scope of the Product}
\subsection{Product Boundary}
\lips
\subsection{Product Use Case Table}
\lips
\subsection{Individual Product Use Cases (PUC's)}
\lips

\section{Functional Requirements}
\subsection{Data Input Requirements}
  \begin{enumerate}
    \item[\textbf{FR-1.}] The system shall allow the user to input new experiment data or parameters.
    \item[\textbf{FR-2.}] The system shall store experiment data in the database with all associated parameters and values correctly labelled.
  \end{enumerate}

\subsection{Data Migration and Organization Requirements}
  \begin{enumerate}
    \item[\textbf{FR-3.}] The system shall read existing experiment data stored in .CSV files.
    \item[\textbf{FR-4.}] The system shall migrate existing experiment data into the database.
    \item[\textbf{FR-5.}] The system shall organize experiment data by timestamps and experiment ID for unique identification.
  \end{enumerate}

\subsection{Data Search and Query Requirements}
\begin{enumerate}
  \item[\textbf{FR-6.}] The system shall allow the user to search for specific datasets based on different parameters.
  \item[\textbf{FR-7.}] The system shall allow the user to query two or more parameters or datasets for comparison and analysis.
  \item[\textbf{FR-8.}] The system shall display the results of a user’s selected search or query in a format that is readable to the user.
\end{enumerate}

\subsection{Data Visualization Requirements}
\begin{enumerate}
  \item[\textbf{FR-9.}] The system shall generate visual graphs based on selected parameters and datasets.
  \item[\textbf{FR-10.}] The system shall allow the user to customize the data visualization by adjusting axes, data ranges, labels, etc.
\end{enumerate}

\subsection{Data Analysis Requirements}
\begin{enumerate}
    \item[\textbf{FR-11.}] The system shall analyze patterns and trends in the experiment data based on the user’s selected parameters.
    \item[\textbf{FR-12.}] The system shall use machine learning algorithms to predict and interpolate the data.
\end{enumerate}

\subsection{Error Tracking Requirements}
This section outlines functional requirements for one of the project's stretch goals.
\begin{enumerate}
  \item[\textbf{FR-13.}] The system shall track and log errors in the experiment data.
\end{enumerate}

\subsection{User Access Management Requirements}
This section outlines functional requirements for one of the project's stretch goals.
\begin{enumerate}
  \item[\textbf{FR-14.}] The system shall allow the user to sign in with valid credentials.
\end{enumerate}

\subsection{Data Export Requirements}
This section outlines functional requirements for one of the project's stretch goals.
\begin{enumerate}
  \item[\textbf{FR-15.}] The system shall generate a report of queries in a session for the user to save or download.
\end{enumerate}

\section{Look and Feel Requirements}
\subsection{Appearance Requirements}
\lips
\subsection{Style Requirements}
\lips

\section{Usability and Humanity Requirements}
\subsection{Ease of Use Requirements}
\lips
\subsection{Personalization and Internationalization Requirements}
\lips
\subsection{Learning Requirements}
\lips
\subsection{Understandability and Politeness Requirements}
\lips
\subsection{Accessibility Requirements}
\lips

\section{Performance Requirements}
\subsection{Speed and Latency Requirements}
\begin{enumerate}
\item The system shall store new data or parameters within 60 seconds of input.
\item The system shall retrieve data from the database within 50ms for typical search and queries.
\item The interaction between the interface and the user shall have a maximum response time of 2 seconds.
\item The system shall have a maximum latency of 2 seconds for typical search and queries.
\item The system shall generate a visualization of the data within 5 seconds.
\end{enumerate}

\subsection{Safety-Critical Requirements}
The product does not have safety-critical requirements to consider.

\subsection{Precision or Accuracy Requirements}
\begin{enumerate}
  \item All parameter values shall be accurate to four decimal places.
  \item All timestamps of experiment data shall be accurate to milliseconds. 
  \item Values on visual data plots shall be accurate to four decimal places.
\end{enumerate}

\subsection{Robustness or Fault-Tolerance Requirements}
\begin{enumerate}
  \item The application shall not terminate but display an error message if it loses connection to the backend server.
  \item The application shall provide basic functionality if it loses connection to the internet.
\end{enumerate}

\subsection{Capacity Requirements}
\begin{enumerate}
  \item The application shall allow for up to three simultaneous users.
  \item The system shall store up to x amount of data.
\end{enumerate}

\subsection{Scalability or Extensibility Requirements}
\begin{enumerate}
  \item The system shall be able to process and store the existing data. The amount of data going into the system is expected to grow until the experiment study comes to an end.
  \item The system shall be able to add additional parameters that did not previously exist in the database at the discretion of the user.
\end{enumerate}

\subsection{Longevity Requirements}
\begin{enumerate}
  \item The system shall operate for the duration of the experiment study.
\end{enumerate}

\section{Operational and Environmental Requirements}
\subsection{Expected Physical Environment}
\begin{enumerate}
  \item The system shall operate in a typical office environment with internet connectivity.
  \item The system shall be compatible with a desktop and laptop environment.
\end{enumerate}

\subsection{Wider Environment Requirements}
\lips
\subsection{Requirements for Interfacing with Adjacent Systems}
\begin{enumerate}
  \item The system shall operate on the most recent versions of Google Chrome and Apple Safari.
\end{enumerate}

\subsection{Productization Requirements}
\begin{enumerate}
  \item The system shall be distributed as a web application.
  \item The system shall have an easy onboarding process with user documentation.
\end{enumerate}

\subsection{Release Requirements}
\begin{enumerate}
  \item The first version of the system shall be released once project completion is reached.
\end{enumerate}

\section{Maintainability and Support Requirements}
\subsection{Maintenance Requirements}
\lips
\subsection{Supportability Requirements}
\lips
\subsection{Adaptability Requirements}
\lips

\section{Security Requirements}
\subsection{Access Requirements}
\lips
\subsection{Integrity Requirements}
\lips
\subsection{Privacy Requirements}
\lips
\subsection{Audit Requirements}
\lips
\subsection{Immunity Requirements}
\lips

\section{Cultural Requirements}
\subsection{Cultural Requirements}
\lips

\section{Compliance Requirements}
\subsection{Legal Requirements}
\lips
\subsection{Standards Compliance Requirements}
\lips

\section{Open Issues}
\lips

\section{Off-the-Shelf Solutions}
\subsection{Ready-Made Products}
\lips
\subsection{Reusable Components}
\lips
\subsection{Products That Can Be Copied}
\lips

\section{New Problems}
\subsection{Effects on the Current Environment}
\lips
\subsection{Effects on the Installed Systems}
\lips
\subsection{Potential User Problems}
\lips
\subsection{Limitations in the Anticipated Implementation Environment That May
Inhibit the New Product}
\lips
\subsection{Follow-Up Problems}
\lips

\section{Tasks}
\subsection{Project Planning}
\lips
\subsection{Planning of the Development Phases}
\lips

\section{Migration to the New Product}
\subsection{Requirements for Migration to the New Product}
\lips
\subsection{Data That Has to be Modified or Translated for the New System}
\lips

\section{Costs}
\lips
\section{User Documentation and Training}
\subsection{User Documentation Requirements}
\lips
\subsection{Training Requirements}
\lips

\section{Waiting Room}
\lips

\section{Ideas for Solution}
\lips

\newpage{}
\section*{Appendix --- Reflection}

The information in this section will be used to evaluate the team members on the
graduate attribute of Lifelong Learning.  Please answer the following questions:

\begin{enumerate}
  \item What knowledge and skills will the team collectively need to acquire to
  successfully complete this capstone project?  Examples of possible knowledge
  to acquire include domain specific knowledge from the domain of your
  application, or software engineering knowledge, mechatronics knowledge or
  computer science knowledge.  Skills may be related to technology, or writing,
  or presentation, or team management, etc.  You should look to identify at
  least one item for each team member.
  \item For each of the knowledge areas and skills identified in the previous
  question, what are at least two approaches to acquiring the knowledge or
  mastering the skill?  Of the identified approaches, which will each team
  member pursue, and why did they make this choice?
\end{enumerate}

\end{document}
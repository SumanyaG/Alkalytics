\documentclass{article}

\usepackage{tabularx}
\usepackage{booktabs}
\usepackage{float}

\title{Reflection and Traceability Report on \progname: Alkalytics}

\author{\authname}

\date{}

%% Comments

\usepackage{color}

\newif\ifcomments\commentstrue %displays comments
%\newif\ifcomments\commentsfalse %so that comments do not display

\ifcomments
\newcommand{\authornote}[3]{\textcolor{#1}{[#3 ---#2]}}
\newcommand{\todo}[1]{\textcolor{red}{[TODO: #1]}}
\else
\newcommand{\authornote}[3]{}
\newcommand{\todo}[1]{}
\fi

\newcommand{\wss}[1]{\authornote{blue}{SS}{#1}} 
\newcommand{\plt}[1]{\authornote{magenta}{TPLT}{#1}} %For explanation of the template
\newcommand{\an}[1]{\authornote{cyan}{Author}{#1}}

%% Common Parts

\newcommand{\progname}{Software Engineering} % PUT YOUR PROGRAM NAME HERE
\newcommand{\authname}{Team 21, Alkalytics
\\ Sumanya Gulati - gulats10
\\ Kate Min - mink9
\\ Jennifer Ye - yej52
\\ Jason Tran - tranj78} % AUTHOR NAMES                  

\usepackage{hyperref}
    \hypersetup{colorlinks=true, linkcolor=blue, citecolor=blue, filecolor=blue,
                urlcolor=blue, unicode=false}
    \urlstyle{same}
                                


\begin{document}

\maketitle

% \plt{Reflection is an important component of getting the full benefits from a
% learning experience.  Besides the intrinsic benefits of reflection, this
% document will be used to help the TAs grade how well your team responded to
% feedback.  Therefore, traceability between Revision 0 and Revision 1 is and
% important part of the reflection exercise.  In addition, several CEAB
% (Canadian Engineering Accreditation Board) Learning Outcomes (LOs) will be
% assessed based on your reflections.}

\section{Changes in Response to Feedback}
This section summarizes the feedback received on various documents throughout
the course of the project as well as feedback on the product itself.

This information is summarized in tables, with the feedback in one column (some
are paraphrased for conciseness), the source of the feedback in the next column,
the change made in response to the feedback in the third column, and a link to
the issue in the last column. 

For documentation related sections, each item of feedback has a link to the
issue that addresses the change wih a link to the specific commit. For feedback
and changes related to code, specific commits or issue links are not provided.

% \plt{Summarize the changes made over the course of the project in response to
% feedback from TAs, the instructor, teammates, other teams, the project
% supervisor (if present), and from user testers.}

% \plt{For those teams with an external supervisor, please highlight how the
% feedback from the supervisor shaped your project.  In particular, you should
% highlight the supervisor's response to your Rev 0 demonstration to them.}

% \plt{Version control can make the summary relatively easy, if you used issues
% and meaningful commits.  If you feedback is in an issue, and you responded in
% the issue tracker, you can point to the issue as part of explaining your
% changes.  If addressing the issue required changes to code or documentation,
% you can point to the specific commit that made the changes.  Although the
% links are helpful for the details, you should include a label for each item of
% feedback so that the reader has an idea of what each item is about without the
% need to click on everything to find out.}

% \plt{If you were not organized with your commits, traceability between
% feedback and commits will not be feasible to capture after the fact.  You will
% instead need to spend time writing down a summary of the changes made in
% response to each item of feedback.}

% \plt{You should address EVERY item of feedback.  A table or itemized list is
% recommended.  You should record every item of feedback, along with the source
% of that feedback and the change you made in response to that feedback.  The
% response can be a change to your documentation, code, or development process.
% The response can also be the reason why no changes were made in response to
% the feedback.  To make this information manageable, you will record the
% feedback and response separately for each deliverable in the sections that
% follow.}

% \plt{If the feedback is general or incomplete, the TA (or instructor) will not
% be able to grade your response to feedback.  In that case your grade on this
% document, and likely the Revision 1 versions of the other documents will be
% low.}

%% DONE %%
\subsection{Problem Statement and Goals, Development Plan Documentation}
Table \ref{table:PSGDV} summarizes all feedback received on the Problem
Statement and Goals and Development Plan documents and the changes made in
response to the feedback.
\begin{table}[H]
\centering
\begin{tabularx}{\textwidth}{|X|c|X|p{1cm}|}
    \hline
    \textbf{Feedback} & \textbf{Source} & \textbf{Summary of Change} &
    \textbf{Issue Link} \\
    \hline
    Development Plan lists software without describing or justifying their use.
    & TA & Added justification for the software used in the development plan. &
    \href{https://github.com/SumanyaG/Alkalytics/issues/146}{Issue 146} \\
    \hline
    Unclear what the product does from the problem description. Start with the
    problem and state what you will do to fix it. & TA & Added more detail about
    the problem and what the product would do to solve it. &
    \href{https://github.com/SumanyaG/Alkalytics/issues/147}{Issue 147} \\
    \hline
    Team charter should have quantifiable goals for, e.g. meeting attendance,
    etc, and describe ways you will remediate or discipline if necessary. & TA &
    Added quantifiable metrics and detailed remediation method and potential
    disciplinary action. &
    \href{https://github.com/SumanyaG/Alkalytics/issues/148}{Issue 148} \\
    \hline
\end{tabularx}
\caption{Feedback and Changes Made for Problem Statement and Goals, Development Plan Documentation}
\label{table:PSGDV}
\end{table}

%% SOMEWHAT DONE %%
\subsection{SRS Documentation}
Tables \ref{table:SRS1} and \ref{table:SRS2} summarizes all feedback received on
the SRS document and the changes made in response to that feedback.
\begin{table}[H]
\centering
\begin{tabularx}{\textwidth}{|X|c|X|p{1cm}|}
    \hline
    \textbf{Feedback} & \textbf{Source} & \textbf{Summary of Change} &
    \textbf{Issue Link} \\
    \hline
    Should not have figures without referring to it and explaining it in the
    document. & TA & Added references to all figures. &
    \href{https://github.com/SumanyaG/Alkalytics/issues/149}{Issue 149} \\
    \hline
    Use symbolic parameters all listed in a certain section instead of ``magic
    numbers''. & TA & Added section for symbolic parameters &
    \href{https://github.com/SumanyaG/Alkalytics/issues/149}{Issue 149} \\
    \hline
    Small grammar/spelling errors & TA & Fixed grammar/spelling errors &
    \href{https://github.com/SumanyaG/Alkalytics/issues/150}{Issue 150} \\
    \hline
    Diagrams should be PDFs. & TA & Changed diagrams to PDFs &
    \href{https://github.com/SumanyaG/Alkalytics/issues/150}{Issue 150} \\
    \hline
    Instead of saying ``traceability'' say something more descriptive, like
    ``related requirements''. & TA & Reworded to use ``related requirements'' &
    \href{https://github.com/SumanyaG/Alkalytics/issues/151}{Issue 151} \\
    \hline
    Must check all requirements and assess them for verifiability. & TA & Added
    verifiability assessment for all requirements. &
    \href{https://github.com/SumanyaG/Alkalytics/issues/152}{Issue 152} \\
    \hline
    Priority and date on the same line is confusing. Try to make it a bit
    clearer. & TA & Separated priority and phase-in date into two distinct
    fields. & \href{https://github.com/SumanyaG/Alkalytics/issues/153}{Issue
    153} \\
    \hline
    Data privacy requirements for research data? Or general laws/privacy
    standards for web applications? & TA & Added more detail about data privacy
    requirements and web application standards. &
    \href{https://github.com/SumanyaG/Alkalytics/issues/154}{Issue 154}
    \\
    \hline
    No formalized part of the document. & TA & Added formalization for
    validation & \href{https://github.com/SumanyaG/Alkalytics/issues/155}{Issue
    155} \\
    \hline
    Maintenance Requirements: Will there be training and continuity
    consideration in addition to documentation for my lab and other users? & Dr.
    de Lannoy & Rephrased maintenance requirements for clarity & No issue link
    available. \\
    \hline
\end{tabularx}
\caption{Feedback and Changes Made for SRS Documentation, Part 1}
\label{table:SRS1}
\end{table}

\begin{table}[H]
\centering
\begin{tabularx}{\textwidth}{|X|c|X|p{1cm}|}
    \hline
    \textbf{Feedback} & \textbf{Source} & \textbf{Summary of Change} &
    \textbf{Issue Link} \\
    \hline
    Missing stakeholder: Developers/Testers are considered other stakeholders &
    Team 1 & No changes made. Refer to issue comments for justification. &
    \href{https://github.com/SumanyaG/Alkalytics/issues/88}{Issue 88}\\
    \hline
    Missing customer: For in house development the customer and client are often
    the same person. In this case, the initial customer would be the client. &
    Team 1 & Identified client as a customer &
    \href{https://github.com/SumanyaG/Alkalytics/issues/89}{Issue 89}\\
    \hline
    Missing business process models: Lack of business process models as
    suggested in the volere template guidance. & Team 1 & No changes made. Refer
    to issue comments for justification. &
    \href{https://github.com/SumanyaG/Alkalytics/issues/90}{Issue 90}\\
    \hline
    Inconsistency between UHR requirements: Should be specifying same percentage
    of users & Team 1 & Changed UHR-1 percentage to 85\% to ensure consistency
    with UHR-4. & \href{https://github.com/SumanyaG/Alkalytics/issues/91}{Issue
    91}\\
    \hline
    OER-5 not verifiable & Team 1 & Added testing percentage metric for
    verifiability &
    \href{https://github.com/SumanyaG/Alkalytics/issues/92}{Issue 92} \\
    \hline
    Inputs and outputs missing in use case diagram & Team 1 & No changes made.
    Refer to issue comments for justification. &
    \href{https://github.com/SumanyaG/Alkalytics/issues/93}{Issue 93}\\
    \hline
\end{tabularx}
\caption{Feedback and Changes Made for SRS Documentation, Part 2}
\label{table:SRS2}
\end{table}

%% DONE %%
\subsection{Hazard Analysis Documentation}
Table \ref{table:HA} summarizes all feedback received on the Hazard Analysis
document and the changes made in response to that feedback.
\begin{table}[H]
\centering
\begin{tabularx}{\textwidth}{|X|c|X|p{1.5cm}|}
    \hline
    \textbf{Feedback} & \textbf{Source} & \textbf{Summary of Change} &
    \textbf{Issue Link} \\
    \hline
    Table doesn't have title or reference in the body of the document. & TA &
    Added title and reference to the table. &
    \href{https://github.com/SumanyaG/Alkalytics/issues/156}{Issue 156} \\
    \hline
    Scope of what's being considered in the hazard analysis isn't clear. & TA &
    Clarified scope &
    \href{https://github.com/SumanyaG/Alkalytics/issues/157}{Issue 157} \\
    \hline
    Be more specific with some. E.g. H2-2, what exactly happens if it does fail
    halfway through? & TA & All requirements re-reviewed and made more specific
    & \href{https://github.com/SumanyaG/Alkalytics/issues/158}{Issue 158} \\
    \hline
    Critical Assumption could be a failure mode & Team 1 & Removed the critical
    assumption & \href{https://github.com/SumanyaG/Alkalytics/issues/106}{Issue
    106}, \newline
    \href{https://github.com/SumanyaG/Alkalytics/issues/108}{Issue 108} \\
    \hline
    Inconsistency between FMEA and assumptions: Assumes internet connection will
    be available, but ``network issues, server downtime'' listed as cause of
    failure. & Team 1 & Modified critical assumption to only consider local
    server infrastructure. &
    \href{https://github.com/SumanyaG/Alkalytics/issues/107}{Issue 107},
    \newline
    \href{https://github.com/SumanyaG/Alkalytics/issues/110}{Issue 110} \\
    \hline
    Inconsistency between assumption and requirement: Assumes system will have
    enough resources, but SR-17 states there must be monitoring and optimization
    to prevent crashes from too many resources being used. & Team 1 & Removed
    assumption. & \href{https://github.com/SumanyaG/Alkalytics/issues/109}{Issue
    109} \\
    \hline
    Authentication effect of failure is unclear: Should clarify what ``loss of
    productivity'' exactly means & Team 1 & Clarified loss of productivity is
    for users & \href{https://github.com/SumanyaG/Alkalytics/issues/111}{Issue
    111} \\
    \hline
\end{tabularx}
\caption{Feedback and Changes Made for Hazard Analysis Documentation}
\label{table:HA}
\end{table}

%% NOT DONE %%
\subsection{Design Documentation}
Tables \ref{table:Design1} and \ref{table:Design2} summarizes all feedback
received on the Design document and the changes made in response to that
feedback.
\begin{table}[H]
\centering
\begin{tabularx}{\textwidth}{|X|c|X|p{1.5cm}|}
    \hline
    \textbf{Feedback} & \textbf{Source} & \textbf{Summary of Change} &
    \textbf{Issue Link} \\
    \hline
    AC5 and UC1 seem to be contradictory to one another. & TA & Removed &
    \href{https://github.com/SumanyaG/Alkalytics/issues/250}{Issue 250} \\
    \hline
    Change project name and provide description in abbreviations table & TA & X
    & \href{https://github.com/SumanyaG/Alkalytics/issues/251}{Issue 251} \\
    \hline
    Use PDF figures. & TA & Changed figures to PDF &
    \href{https://github.com/SumanyaG/Alkalytics/issues/251}{Issue 251} \\
    \hline
    List possible exceptions for each method in syntax table. Environment vars
    should be actual vars (with names) that represent things like the local
    filesystem and should be used by the semantics. & TA & Removed description
    of exceptions in each access programs table. &
    \href{https://github.com/SumanyaG/Alkalytics/issues/252}{Issue 252} \\
    \hline
    Data validation module: should be, e.g. ``output := MIN VOLTAGE $\le$ v
    $\le$ MAX VOLTAGE'', no need for words. & TA & X &
    \href{https://github.com/SumanyaG/Alkalytics/issues/252}{Issue 252} \\
    \hline
    Output and exceptions are different, should be their own bullet points.
    Different modules are using different formatting. ``Output'' labels are
    completely omitted in the data validation module. & TA & Separated output
    and exception into their own bullet point fields. Fixed every modules syntax
    and semantics structure for consistency. &
    \href{https://github.com/SumanyaG/Alkalytics/issues/253}{Issue 253} \\
    \hline
    More general validations you may need to do, like if something is the right
    datatype, format, etc. This module could be more generic. & TA & X &
    \href{https://github.com/SumanyaG/Alkalytics/issues/254}{Issue 254} \\
    \hline
    Limitations hould be more specific and detailed. Focus on se principles
    (encapsulation, information hiding, modularity, etc) & TA & X &
    \href{https://github.com/SumanyaG/Alkalytics/issues/255}{Issue 255} \\
    \hline
\end{tabularx}
\caption{Feedback and Changes Made for Design Documentation}
\label{table:Design1}
\end{table}

\begin{table}[H]
\centering
\begin{tabularx}{\textwidth}{|X|c|X|p{1.5cm}|}
    \hline
    \textbf{Feedback} & \textbf{Source} & \textbf{Summary of Change} &
    \textbf{Issue Link} \\
    \hline
    Exported Constants Implementation: Make pH and Flow Rate have max and min
    value constants instead of range & Team 1 & X &
    \href{https://github.com/SumanyaG/Alkalytics/issues/173}{Issue 173} \\
    \hline
    Unnecessary state variable: isTransformed seems unnecessary if
    transformedData has a value & Team 1 & X
    &\href{https://github.com/SumanyaG/Alkalytics/issues/174}{Issue 174} \\
    \hline
    Consider defining types for user management module access routines & Team 1
    & Defined custom \texttt{USER} type as environment variable for better type
    safety and maintainability &
    \href{https://github.com/SumanyaG/Alkalytics/issues/176}{Issue 176} \\
    \hline
    Inconsistent outputs in UI Module: No outputs in access programs table but
    redundantly mention them in the semantics & Team 1 & Removed output fields
    from semantics to be consistent with table. &
    \href{https://github.com/SumanyaG/Alkalytics/issues/177}{Issue 177} \\
    \hline
    Inconsistencies between anticipated change descriptions and traceability
    matrix & Team 1 & X &
    \href{https://github.com/SumanyaG/Alkalytics/issues/178}{Issue 178} \\
    \hline
\end{tabularx}
\caption{Feedback and Changes Made for Design Documentation}
\label{table:Design2}
\end{table}

\subsection{VnV Plan and Report Documentation}
Table \ref{table:VnV1} summarizes all feedback received on the VnV Plan document
and the changes made in response to that feedback, Table \ref{table:VnV2} does
the same for the VnV Report document.
\begin{table}[H]
\centering
\begin{tabularx}{\textwidth}{|X|c|X|p{1.5cm}|}
    \hline
    \textbf{Feedback} & \textbf{Source} & \textbf{Summary of Change} &
    \textbf{Issue Link} \\
    \hline
    Sentence structure/flow problems, reference tables. & TA & Fixed sentence
    structure and added references to tables. &
    \href{https://github.com/SumanyaG/Alkalytics/issues/246}{Issue 246} \\
    \hline
    System tests should be more specific about what kinds of errors you will
    detect. & TA & Added more detail about the types of errors that would be
    detected in the tests &
    \href{https://github.com/SumanyaG/Alkalytics/issues/247}{Issue 247} \\
    \hline
    Not clear how survey data will be collected and analyzed for validation. &
    TA & Clarified structure for survey response collection and added passing
    criteria/metrics &
    \href{https://github.com/SumanyaG/Alkalytics/issues/248}{Issue 248} \\
    \hline
    Include traceability inside the tests instead of outside. Tests should be
    made more granular. & TA & No changes made. Refer to issue comments for
    justification. &
    \href{https://github.com/SumanyaG/Alkalytics/issues/249}{Issue 249} \\
    \hline
    Change traceability table to a matrix. & TA & Converted to matrix. &
    \href{https://github.com/SumanyaG/Alkalytics/issues/249}{Issue 249} \\
    \hline
    Design verification plan should only include verifying design documents and
    overall system design, not code-related. & Team 1 & Rephrased verification
    method & \href{https://github.com/SumanyaG/Alkalytics/issues/135}{Issue 135}
    \\
    \hline
    Make NFR-LF2 test more specific by mentioning what devices will used to test
    device compatibility & Team 1 & Specified specific devices and dimensions
    that would be used to test. &
    \href{https://github.com/SumanyaG/Alkalytics/issues/136}{Issue 136} \\
    \hline
    FR-ST1 corresponds to 4 different FRS, should split this test in to two,
    more specific, tests. & Team 1 & Added FR-ST1.1 to make the two tests more
    specific and granular &
    \href{https://github.com/SumanyaG/Alkalytics/issues/137}{Issue 137} \\
    \hline
    Specified output for FR-ST1 is not really an output, but rather a process
    that is supposed to happen. & Team 1 & Updated output to be more clear &
    \href{https://github.com/SumanyaG/Alkalytics/issues/138}{Issue 138} \\
    \hline
    All of the tests for functional requirements are manual. & Team 1 & No
    changes made. Refer to issue comments for justification. &
    \href{https://github.com/SumanyaG/Alkalytics/issues/139}{Issue 139} \\
    \hline
    No test for ensuring scenario where incorrect credentials are provided. &
    Team 1 & No changes made. Refer to issue comments for justification. &
    \href{https://github.com/SumanyaG/Alkalytics/issues/141}{Issue 141} \\
    \hline
\end{tabularx}
\caption{Feedback and Changes Made for VnV Plan Documentation}
\label{table:VnV1}
\end{table}

\begin{table}[H]
\centering
\begin{tabularx}{\textwidth}{|X|c|X|p{1.5cm}|}
    \hline
    \textbf{Feedback} & \textbf{Source} & \textbf{Summary of Change} &
    \textbf{Issue Link} \\
    \hline
    Expand user pool for usability testing: 2 users is a relatively small user
    group & Team 1 & No changes made. Refer to issue comments for justification.
    & \href{https://github.com/SumanyaG/Alkalytics/issues/235}{Issue 235} \\
    \hline
    Clarify test result: Should modify the test plan to match the new
    requirement and mark the new test as pass/fail instead of neutral. & Team 1
    & Updated test to reflect changes due to the new requirement (as opposed to
    modifying the test plan itself). Test now marked as pass. &
    \href{https://github.com/SumanyaG/Alkalytics/issues/236}{Issue 236} \\
    \hline
    Web browser tests: For NFR-OE1, mention which web browsers the system was
    tested on for clarity. Suggest adding a test for Opera as well. & Team 1 &
    Clarified which browser (Google Chrome) system was tested on. Refer to issue
    comments for further explanation. &
    \href{https://github.com/SumanyaG/Alkalytics/issues/237}{Issue 237} \\
    \hline
    Make list of tasks for users to complete to be more concrete. & Team 1 &
    Updated to include this information. &
    \href{https://github.com/SumanyaG/Alkalytics/issues/238}{Issue 238} \\ 
    \hline
    Explicitly state the input period of inactivity and the expected period
    inactivity to cause the user to be logged out. & Team 1 & Updated to include
    this information. &
    \href{https://github.com/SumanyaG/Alkalytics/issues/239}{Issue 239} \\
    \hline
    Include number of sample uploads that were completed to get the average
    upload time & Team 1 & Updated to include this information. &
    \href{https://github.com/SumanyaG/Alkalytics/issues/240}{Issue 240} \\
    \hline
\end{tabularx}
\caption{Feedback and Changes Made for VnV Report Documentation}
\label{table:VnV2}
\end{table}


\subsection{Supervisor Feedback}
Table \ref{table:Code} summarizes the feedback received by the supervisors
regarding the actual product and the changes made in response to that feedback.
\begin{table}[H]
\centering
\begin{tabularx}{\textwidth}{|X|c|X|}
    \hline
    \textbf{Feedback} & \textbf{Source} & \textbf{Summary of Change} \\
    \hline
    Should be able to do direct operations on experiment log data and apply
    Excel formulas of choosing. & Bassel Adbelkader & CRUD operations for
    experiment log and Excel function bar implemented. \\
    \hline
    Should be able to compute efficiency factors. & Dr. de Lannoy & Feature
    implemented. \\
    \hline
    Should be able to filter through the data based on certain criteria, as just
    selecting data to plot by the experiment's date is not helpful. & Dr. de
    Lannoy & Feature implemented. \\
    \hline
    Various UI changes & Dr. de Lannoy & UI changes made. Refer to
    \href{https://github.com/SumanyaG/Alkalytics/blob/main/docs/UsabilityTestingReport/UsabilityTestingReport.pdf}{Usability
    Testing Report} for more details. \\
    \hline
    Various UI changes & Meghna Saha & UI changes made. Refer to
    \href{https://github.com/SumanyaG/Alkalytics/blob/main/docs/UsabilityTestingReport/UsabilityTestingReport.pdf}{Usability
    Testing Report} for more details. \\
    \hline
\end{tabularx}
\caption{Feedback and Changes Made for Application}
\label{table:Code}
\end{table}

\section{Challenge Level and Extras}
This section outlines the challenge level and extras for the Alkalytics project.

\subsection{Challenge Level}

The challenge level for this project has been identified as \textbf{general}.
This was determined from the domain knowledge, implementation challenges, and
complexity that the project demands. There was no research component or extra
domain knowledge required for the project to be considered advanced.
% \plt{State the challenge level (advanced, general, basic) for your project.
% Your challenge level should exactly match what is included in your problem
% statement.  This should be the challenge level agreed on between you and the
% course instructor.}

\subsection{Extras}

The two extras completed as part of this project are as follows:
\begin{itemize}
    \item[(a)] \textbf{Usability Testing Report}: A comprehensive report
    detailing the assessment of the application's usablity, including the
    objectives, methodologies, results of the conducted usability testing
    sessions, and the changes proposed and implemented based on user feedback.
    \item[(b)] \textbf{User Manual}: A detailed guide outlining the
    application's features, installation and usage instructions, and
    troubleshooting steps for common issues.
\end{itemize}

% \plt{Summarize the extras (if any) that were tackled by this project.  Extras
% can include usability testing, code walkthroughs, user documentation, formal
% proof, GenderMag personas, Design Thinking, etc.  Extras should have already
% been approved by the course instructor as included in your problem statement.}

\section{Design Iteration (LO11 (PrototypeIterate))}

\plt{Explain how you arrived at your final design and implementation.  How did
the design evolve from the first version to the final version?} 

\plt{Don't just say what you changed, say why you changed it.  The needs of the
client should be part of the explanation.  For example, if you made changes in
response to usability testing, explain what the testing found and what changes
it led to.}

\section{Design Decisions (LO12)}

\plt{Reflect and justify your design decisions.  How did limitations,
 assumptions, and constraints influence your decisions?  Discuss each of these
 separately.}

\section{Economic Considerations (LO23)}

\plt{Is there a market for your product? What would be involved in marketing
your product? What is your estimate of the cost to produce a version that you
could sell?  What would you charge for your product?  How many units would you
have to sell to make money? If your product isn't something that would be sold,
like an open source project, how would you go about attracting users?  How many
potential users currently exist?}

\section{Reflection on Project Management (LO24)}

\plt{This question focuses on processes and tools used for project management.}

\subsection{How Does Your Project Management Compare to Your Development Plan}

\plt{Did you follow your Development plan, with respect to the team meeting
plan, team communication plan, team member roles and workflow plan.  Did you use
the technology you planned on using?}

\subsubsection{Team Meeting Plan}
The team held weekly meetings on Fridays during the fall term, as outlined in
the Development Plan. In the winter term, these meetings were rescheduled to
Mondays during tutorial time. Most meetings were conducted virtually via
Microsoft Teams, with additional sessions scheduled as needed, particularly
leading up to deadlines. Meeting structures followed the Development Plan, with
a designated chair preparing an agenda and a note taker recording and posting
meeting minutes to the corresponding issue in the repository.

Meetings with the primary supervisor, Dr. Charles de Lannoy, were infrequent and
primarily virtual due to scheduling constraints and a mutual decision that
biweekly meetings were not necessary. During the fall term, most supervisory
meetings were held with the secondary supervisor, Bassel Abdelkader, who
provided updates to Dr. de Lannoy, reducing the need for separate meetings.

The team met with Bassel on a weekly basis, transitioning to biweekly meetings
as the project progressed. These meetings were conducted virtually. Beginning in
the winter term, Bassel was no longer available, and the team shifted to
meetings with Dr. de Lannoy instead, with Bassel attending when possible.

Additionally, the team met three times with Meghna Saha, an undergraduate
student in the research lab whose work aligned with the objectives of
Alkalytics. These meetings focused on knowledge transfer and conducting a
usability testing session.

\subsubsection{Team Communication Plan}
As outlined in the Development Plan, the team primarily communicated through an
Instagram group chat, supplemented by Microsoft Teams for resource sharing
(external files, sample works, etc).

GitHub was used for version control and issue tracking. While issues were
effectively utilized for assigning and tracking most tasks, they were not
consistently used for all deliverables. Issues were primarily employed for
meeting logs and managing feedback-related changes. Code-related tasks were not
tracked through issues; instead, individual branches and pull requests were used
to manage and review code changes.


\subsubsection{Team Member Roles}
The team largely adhered to the roles and rotation of roles outlined in the
Development Plan. The meeting chair and notetaker for the stages in the Project
Decomposition and Scheduling section was roughly as follows:

\begin{itemize}
    \item \textbf{Stages 1 - 2}
        \begin{itemize}
            \item Meeting Chair: Sumanya Gulati
            \item Notetaker: Jennifer Ye
        \end{itemize}
    \item \textbf{Stages 2 - 3}
        \begin{itemize}
            \item Meeting Chair: Jason Tran
            \item Notetaker: Kate Min
        \end{itemize}
    \item \textbf{Stages 3 - 5}
    \item \begin{itemize}
            \item Meeting Chair: Jennifer Ye
            \item Notetaker: Sumanya Gulati
        \end{itemize}
    \item \textbf{Stages 5 - 7}
        \begin{itemize}
            \item Meeting Chair: Kate Min
            \item Notetaker: Jason Tran
        \end{itemize}
\end{itemize}

One deviation from the Development Plan was the decision not to appoint a
primary reviewer for pull requests, as the team deemed it unnecessary to have a
single official ``approver'' in order to merge. Instead, pull requests were
reviewed and approved by team members as needed.

Roles related to implementation remained aligned with the Development Plan, with
tasks divided based on each team member's respective strengths, as initially
outlined.

\subsubsection{Workflow Plan}
The workflow plan was generally effective and mostly followed, espcially the
branches and commit naming conventions. However, as deadlines approached, the
naming convention became a lower priority, leading to lapses in ensuring
correctness. The merging strategy was implemented smoothly, with efficient
integration of code contributions. Continuous Integration (CI) was not as much
of a help, as each developer took personal responsibility for validating their
work beforehand. Initially, issue management was often overlooked, but as the
project progressed, it became more important, allowing for better tracking of
tasks and ultimately streamlining the development process. 

\subsection{What Went Well?}

The project benefited from effective communication among team members, which
created a sense of responsibility and collaboration. Team members volunteered to
assist one another, ensuring that tasks were evenly distributed, while
considering individual circumstances. This approach, coupled with a lack of
negative assumptions, allowed for natural assignment of roles based on each
member's strengths. If a team member wanted to challenge themselves, they could
take on a tasks freely. This promoted a learning environment where patience was
encouraged. Expectations among team members remained aligned throughout the
project, contributing to a fluid workflow.

The implementation of MongoDB and the migration algorithm was particularly
successful, as the transition from CSV to JSON format proved to be
straightforward. Moreover, working on each other's code allow for a better time
learning within the full-stack development process. Python's capabilities in
analyzing Excel and CSV files performed exceptionally well, as performance-wise
it met all metrics even before optimization. The integration of various
technologies allowed for seamless data handling and processing.

\subsection{What Went Wrong?}

Despite the strengths in communication, the team occasionally faced challenges
in aligning with the client's expectations. Ideas exchanged between the team and
the client were often assumed to be a requirement without confirming feasibility
or  timelines, which forces the team to make assumptions or unclear actions. 

The use of D3 and TypeScript became tedious; distracting the team from what is
really important. The complexity of D3 graph generation was increased by unclear
requirements from the client as well. The learning curve associated with D3 was
steep, and the lack of clear directives resulted in wasted time and effort,
often with backwards progression. Furthemore, the integration of TypeScript
introduced type-related issues that required additional debugging. 

\subsection{What Would you Do Differently Next Time?}

If we were given another chance to do this project, there are many things we'd
do differently to address the struggles mentioned above.

Firstly, we'd try to consider feasibility of requirements during our meetings
with clients, rather than putting it on the backburner and revisiting it later
to decide whether it was feasible or not. This way, we can avoid making any
assumptions or unclear actions, and gather more information during our meetings
instead of looking back to ask for more details. With a tight deadline, this
would have helped us save a lot of time in doing different iterations. 

We would also try to look for other technologies that can aid in graph
generation. Learning D3 while trying to do this project caused a lot of delays
in the progress of the project, as it had a steep learning curve. 

The learning curve of D3 combined with the unclear/unfeasible requirements
provided by our clients introduced unnecessary difficulty into this project. We
had to decline certain requests about graphing as we had to spend time learning
the library and implementing it into our app in the ways we needed it to be. The
request itself was not impossible, rather we had little time to complete the
development of the project.
\section{Reflection on Capstone}

Through this Capstone project, we got the chance to apply the knowledge we
learned from our time here in the Software Engineering program. But the project
also allowed room for us to apply what we learned from working in the industry
as well. 
\subsection{Which Courses Were Relevant}

For this project, there were many courses we took that we felt were relevant.
\textbf{Discrete Math I (SFWRENG 2DM3)} - Useful when considering the Boolean state
evaluation of the different components in our project.

\textbf{Intro to Software Development (SFWRENG 2AA4)} - A lot of software design
principles and patterns taught from this course was used in developing the
program. Adhering to these principles ensured the project was built in a
maintainable and scalable fashion.

\textbf{Databases (SFWRENG 3DB3)} - Understanding the underlying concepts of databases
was very useful. Although we used a NoSQL database for our project,
understanding the idea of how to build the right query to retrieve the desired
data ensured that the application got the correct data that it needs.

\textbf{Software Requirements (SFWRENG 3RA3)} - This course was focused heavily around
eliciting requirements, and formatting them into a document that will streamline
the development process. The SRS was one of the most important documents in this
entire project, highlighting the desired results of this project. This includes
what the project should do, and any properties that we want the project to hold. 

\textbf{Software Design III (SFWRENG 3A04)} - This course was a "mini capstone" over the
span of one semester. It included a pre-determined project, where we had to
write our own documentation, then designing the project, and finally building
it. All the skills acquired from this course could be directly transferred to
our capstone project, since everything done from that course was also done in
our capstone project. 

\textbf{Software Testing (SFWRENG 3S03)} - This course revolved around testing software,
ranging from the different kinds of tests and how they are used, as well as the
importance of testing. In our capstone project, a lot of testing was required to
be done. It ranged from unit test, system test, useability testing and more.
Tests had to be planned, then performed in order to ensure that the project
meets requirements, and avoids any hazard states.

\textbf{Human Computer Interfaces (SFWRENG 4HC3)} - Many of the things done in this
course aimed to teach us how to design and build interfaces, with the user
experience in mind. This was helpful since a lot of the non-functional
requirements focused on the user experience, which were also based on Norman's
design principles along with HCL, which was taught in this course.
\subsection{Knowledge/Skills Outside of Courses}

When building a real project, the courses we take in university by themselves
are not enough. The courses teach us what we are required to do, and a lot of
concepts for software design and development. However, it doesn't teach us about
the possible tech stack that we could use in order to satisfy our requirements.
For example, in Databases, we learned how SQL tables are structured, and how
queries are formed. However, in the real world, there are many different SQL
databases. We relied heavily on our internship experiences to decide which
specific technologies we would use for each component. In our own experiences,
we gained valuable experience in full stack development, and UI/UX design. The
experiences we have pushed us to lean towards a certain tech stack, such as
MongoDB for a NoSQL database, GraphQL to handle API requests, and React to build
the front end. It would be much harder to build this project if we chose a tech
stack we were not familiar with, thus our experiences helped us decide on a tech
stack what would make the development process as smooth as possible

\end{document}
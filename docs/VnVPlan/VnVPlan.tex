\documentclass[12pt, titlepage]{article}

\usepackage{booktabs}
\usepackage{caption}
\usepackage{tabularx}
\usepackage{hyperref}
\hypersetup{
    colorlinks,
    citecolor=blue,
    filecolor=black,
    linkcolor=red,
    urlcolor=blue
}
\usepackage[round]{natbib}
\usepackage{longtable}


%% Comments

\usepackage{color}

\newif\ifcomments\commentstrue %displays comments
%\newif\ifcomments\commentsfalse %so that comments do not display

\ifcomments
\newcommand{\authornote}[3]{\textcolor{#1}{[#3 ---#2]}}
\newcommand{\todo}[1]{\textcolor{red}{[TODO: #1]}}
\else
\newcommand{\authornote}[3]{}
\newcommand{\todo}[1]{}
\fi

\newcommand{\wss}[1]{\authornote{blue}{SS}{#1}} 
\newcommand{\plt}[1]{\authornote{magenta}{TPLT}{#1}} %For explanation of the template
\newcommand{\an}[1]{\authornote{cyan}{Author}{#1}}

%% Common Parts

\newcommand{\progname}{Software Engineering} % PUT YOUR PROGRAM NAME HERE
\newcommand{\authname}{Team 21, Alkalytics
\\ Sumanya Gulati - gulats10
\\ Kate Min - mink9
\\ Jennifer Ye - yej52
\\ Jason Tran - tranj78} % AUTHOR NAMES                  

\usepackage{hyperref}
    \hypersetup{colorlinks=true, linkcolor=blue, citecolor=blue, filecolor=blue,
                urlcolor=blue, unicode=false}
    \urlstyle{same}
                                


\begin{document}

\title{System Verification and Validation Plan for \progname{}} 
\author{\authname}
\date{\today}
	
\maketitle

\pagenumbering{roman}

\section*{Revision History}

\begin{tabularx}{\textwidth}{p{3cm}p{2cm}X}
\toprule {\bf Date} & {\bf Version} & {\bf Notes}\\
\midrule
Date 1 & 1.0 & Notes\\
Date 2 & 1.1 & Notes\\
\bottomrule
\end{tabularx}

~\\
\wss{The intention of the VnV plan is to increase confidence in the software.
However, this does not mean listing every verification and validation technique
that has ever been devised.  The VnV plan should also be a \textbf{feasible}
plan. Execution of the plan should be possible with the time and team available.
If the full plan cannot be completed during the time available, it can either be
modified to ``fake it'', or a better solution is to add a section describing
what work has been completed and what work is still planned for the future.}

\wss{The VnV plan is typically started after the requirements stage, but before
the design stage.  This means that the sections related to unit testing cannot
initially be completed.  The sections will be filled in after the design stage
is complete.  the final version of the VnV plan should have all sections filled
in.}

\newpage

\tableofcontents

\listoftables
\wss{Remove this section if it isn't needed}

\listoffigures
\wss{Remove this section if it isn't needed}

\newpage

\section{Symbols, Abbreviations, and Acronyms}

\renewcommand{\arraystretch}{1.2}
\begin{tabular}{l l} 
  \toprule		
  \textbf{symbol} & \textbf{description}\\
  \midrule 
  T & Test\\
  \bottomrule
\end{tabular}\\

\wss{symbols, abbreviations, or acronyms --- you can simply reference the SRS
  \citep{SRS} tables, if appropriate}

\wss{Remove this section if it isn't needed}

\newpage

\pagenumbering{arabic}

This document ... \wss{provide an introductory blurb and roadmap of the
  Verification and Validation plan}

\section{General Information}

\subsection{Summary}

\wss{Say what software is being tested.  Give its name and a brief overview of
  its general functions.}

\subsection{Objectives}

\wss{State what is intended to be accomplished.  The objective will be around
  the qualities that are most important for your project.  You might have
  something like: ``build confidence in the software correctness,''
  ``demonstrate adequate usability.'' etc.  You won't list all of the qualities,
  just those that are most important.}

\wss{You should also list the objectives that are out of scope.  You don't have 
the resources to do everything, so what will you be leaving out.  For instance, 
if you are not going to verify the quality of usability, state this.  It is also 
worthwhile to justify why the objectives are left out.}

\wss{The objectives are important because they highlight that you are aware of 
limitations in your resources for verification and validation.  You can't do everything, 
so what are you going to prioritize?  As an example, if your system depends on an 
external library, you can explicitly state that you will assume that external library 
has already been verified by its implementation team.}

\subsection{Challenge Level and Extras}

\wss{State the challenge level (advanced, general, basic) for your project.
Your challenge level should exactly match what is included in your problem
statement.  This should be the challenge level agreed on between you and the
course instructor.  You can use a pull request to update your challenge level
(in TeamComposition.csv or Repos.csv) if your plan changes as a result of the
VnV planning exercise.}

\wss{Summarize the extras (if any) that were tackled by this project.  Extras
can include usability testing, code walkthroughs, user documentation, formal
proof, GenderMag personas, Design Thinking, etc.  Extras should have already
been approved by the course instructor as included in your problem statement.
You can use a pull request to update your extras (in TeamComposition.csv or
Repos.csv) if your plan changes as a result of the VnV planning exercise.}

\subsection{Relevant Documentation}

\wss{Reference relevant documentation.  This will definitely include your SRS
  and your other project documents (design documents, like MG, MIS, etc).  You
  can include these even before they are written, since by the time the project
  is done, they will be written.  You can create BibTeX entries for your
  documents and within those entries include a hyperlink to the documents.}

\citet{SRS}

\wss{Don't just list the other documents.  You should explain why they are relevant and 
how they relate to your VnV efforts.}

\section{Plan}

\wss{Introduce this section.  You can provide a roadmap of the sections to
  come.}

\subsection{Verification and Validation Team}

\wss{Your teammates.  Maybe your supervisor.
  You should do more than list names.  You should say what each person's role is
  for the project's verification.  A table is a good way to summarize this information.}

\subsection{SRS Verification Plan}

\wss{List any approaches you intend to use for SRS verification.  This may
  include ad hoc feedback from reviewers, like your classmates (like your
  primary reviewer), or you may plan for something more rigorous/systematic.}

\wss{If you have a supervisor for the project, you shouldn't just say they will
read over the SRS.  You should explain your structured approach to the review.
Will you have a meeting?  What will you present?  What questions will you ask?
Will you give them instructions for a task-based inspection?  Will you use your
issue tracker?}

\wss{Maybe create an SRS checklist?}

\subsection{Design Verification Plan}

\wss{Plans for design verification}

\wss{The review will include reviews by your classmates}

\wss{Create a checklists?}

\subsection{Verification and Validation Plan Verification Plan}

\wss{The verification and validation plan is an artifact that should also be
verified.  Techniques for this include review and mutation testing.}

\wss{The review will include reviews by your classmates}

\wss{Create a checklists?}

\subsection{Implementation Verification Plan}

\wss{You should at least point to the tests listed in this document and the unit
  testing plan.}

\wss{In this section you would also give any details of any plans for static
  verification of the implementation.  Potential techniques include code
  walkthroughs, code inspection, static analyzers, etc.}

\wss{The final class presentation in CAS 741 could be used as a code
walkthrough.  There is also a possibility of using the final presentation (in
CAS741) for a partial usability survey.}

\subsection{Automated Testing and Verification Tools}

\wss{What tools are you using for automated testing.  Likely a unit testing
  framework and maybe a profiling tool, like ValGrind.  Other possible tools
  include a static analyzer, make, continuous integration tools, test coverage
  tools, etc.  Explain your plans for summarizing code coverage metrics.
  Linters are another important class of tools.  For the programming language
  you select, you should look at the available linters.  There may also be tools
  that verify that coding standards have been respected, like flake9 for
  Python.}

\wss{If you have already done this in the development plan, you can point to
that document.}

\wss{The details of this section will likely evolve as you get closer to the
  implementation.}

\subsection{Software Validation Plan}

\wss{If there is any external data that can be used for validation, you should
  point to it here.  If there are no plans for validation, you should state that
  here.}

\wss{You might want to use review sessions with the stakeholder to check that
the requirements document captures the right requirements.  Maybe task based
inspection?}

\wss{For those capstone teams with an external supervisor, the Rev 0 demo should 
be used as an opportunity to validate the requirements.  You should plan on 
demonstrating your project to your supervisor shortly after the scheduled Rev 0 demo.  
The feedback from your supervisor will be very useful for improving your project.}

\wss{For teams without an external supervisor, user testing can serve the same purpose 
as a Rev 0 demo for the supervisor.}

\wss{This section might reference back to the SRS verification section.}

\section{System Tests}
This section covers all tests for different areas of the system.

\subsection{Tests for Functional Requirements}

The subsections below covers each major functional of the application, from uploading data to the various outputs of post query analysis. Many of the functions come from the same user flow. By testing the subsections below guarintees that all possible user flows involving the main application functioanlities are working as expected. 

\subsubsection{Data Input and Storage}
The tests below provide a way to evaluate the correctness of data input and storagee for the following functional requirements:
\begin{itemize}
  \item FR-1
  \item FR-2
  \item FR-3
  \item FR-4
\end{itemize}

\begin{enumerate}

\item{FR-ST1}

Control: Manual
					
Initial State: Database is running and ready to intake data
					
Input: Dataset to be stored, in .CSV format
					
Output: Data that is inputted is sent into the system, then labelled and stored successfully

Test Case Derivation: When the data is sent to the system, the data should be stored somewhere and labelled properly before it can be queried.
					
How test will be performed: The test can be performed by sending a test sample of varying sizes to the storage in the system.

\end{enumerate}

\subsubsection{Data Querying and Results}
Th tests below provide a way to evaluate the data querying and visualization process of the system for the following functional requirements:
\begin{itemize}
  \item FR-5
  \item FR-6
  \item FR-7
  \item FR-8
  \item FR-9
\end{itemize}
\begin{enumerate}

  \item{FR-ST2}
  
  Control: Manual
            
  Initial State: The system is not running any jobs, and the user interface is cleared. 
            
  Input: A selection of different combinations of parameters and datasets to be queried on
            
  Output: A human-readable and customizable visualization of correct results corresponding to the selected paramters from the input
  
  Test Case Derivation: The expected output of the system is based on the query parameters selected. A user expects the data analysis to match with what they asked for, and the user is allowed to customize the visuallized data.
            
  How test will be performed: The database will be queried using multiple combinations of parameters, and the results will be compared against the expected output. The outputted visualization will then be tested for customizability.
  
  \end{enumerate}

\subsubsection{Data analysis}
    The tests below provide a way to evaluate the data analysis in the application for the following functional requirements:
    \begin{itemize}
      \item FR-10
      \item FR-11
    \end{itemize}
      \begin{enumerate}
      
        \item{FR-ST3}
        
        Control: Manual
                  
        Initial State: The application's cleared user interface which has not yet been used to query data, with no graph showing yet.
                  
        Input: A combination of parameters to query on for a selected dataset
                  
        Output: A small written human-readable paragraph explaining the input data.
       
        Test Case Derivation: To be able to understand the returned data in ways other than through a graph, a written response gives the user a variety of choices.
                 
        How test will be performed: The website interface will allow the user to pick a written analysis response. The application will look for patterns and trends in the data and will output the findings. 
        
        \end{enumerate}


      \subsubsection{Data Hygiene}
      The tests below provide a way to evaluate how the application maintains the data hygiene of the datasets related to the following functional requirements:
      \begin{itemize}
        \item FR-12
        \item FR-13
      \end{itemize}
        \begin{enumerate}
        
          \item{FR-ST4}
          
          Control: Manual
                    
          Initial State: The application's cleared user interface which has not yet been used to query data, with no graph showing yet.
                    
          Input: Dataset to be stored, in .CSV format
                    
          Output: A log file documenting errors found in the input data and/or removals of missing data.
          
          Test Case Derivation: This is to ensure the efficency of the querying and ensuring the database is only as big as it needs to be. This will also ensure that errors are dealt with by the application and are recorded to document any inconsistancies to increase traceability.
                    
          How test will be performed: After loading in a CSV file with some error in the data. A log file will be generated documenting the error after the application attempts to fix it. 
          
          \end{enumerate}

          \subsubsection{User Access}
          This tests below provide a way to evaluate how the application allows for user login related to the following functional requirements:
          \begin{itemize}
            \item FR-14
          \end{itemize}
            \begin{enumerate}
            
              \item{FR-ST5}
              
              Control: Manual
                        
              Initial State: User interface shows a login page, with no login credentials currently used
                        
              Input: Sample user credentials
                        
              Output: The page redirects to the page designated after login
              
              Test Case Derivation: The system must be able to authenticate users properly, and when authenticated, they should be given access to the application
                        
              How test will be performed: Sample credentials with different combinations of characters will be used to log in to ensure the fields handle credentials correctly.  
              
              \end{enumerate}

  \subsubsection{Data Export}
  The tests below provide a way to evaluate the export of query reports after a session for the following functional requirement:
  \begin{itemize}
    \item FR-15
  \end{itemize}
  \begin{enumerate}
    \item{FR-ST6}
    Control: Manual

    Initial State: User interface after multiple usages of data queries

    Input: User clicking the button for saving or downloading

    Output: Query report will be downloaded to the user's device

    Test Case Derivation: The user needs to be able to get a system generated report of the queries from their session, and be able to save or download that report as needed.

    How Test Will Be Performed: After making multiple queries on the data, the save/download button will be pressed to test functionality.
  \end{enumerate}


\subsection{Tests for Nonfunctional Requirements}

\wss{The nonfunctional requirements for accuracy will likely just reference the
  appropriate functional tests from above.  The test cases should mention
  reporting the relative error for these tests.  Not all projects will
  necessarily have nonfunctional requirements related to accuracy.}

\wss{For some nonfunctional tests, you won't be setting a target threshold for
passing the test, but rather describing the experiment you will do to measure
the quality for different inputs.  For instance, you could measure speed versus
the problem size.  The output of the test isn't pass/fail, but rather a summary
table or graph.}

\wss{Tests related to usability could include conducting a usability test and
  survey.  The survey will be in the Appendix.}

\wss{Static tests, review, inspections, and walkthroughs, will not follow the
format for the tests given below.}

\wss{If you introduce static tests in your plan, you need to provide details.
How will they be done?  In cases like code (or document) walkthroughs, who will
be involved? Be specific.}
\newline
Non-functional requirements will diverge from typical functional requirements
testing. Listed below are different testing methods that will be used throughout
the section.

\noindent \textbf{User Demo Assessment}: We will have users engage in a demo of the application,
and observe how they interact with the application with respect to the
requirements and criteria defined in SRS.


\subsubsection{Look and Feel Requirements}
Look and feel testing is heavily subjective, and will require users to test.
These requirements will be heavily tested with the user demo assessment.

\begin{enumerate}

\item{NFR-LF1 (User Interface)\\}

NFR: LFR-1, LFR-3, LFR-4, LFR-5, LFR-6, LFR-7

Type: User Demo, Manual

Initial State: Fully functional application ready for user interaction, starting
at the login page

Input/Condition: User engagement with application 

Output/Result: Recorded observations of how the user was able to interact with
the system 

How test will be performed: 
\begin{itemize}
  \item User will be given temporary, working credentials 
  \item User will use the app, starting from the login page, and navigate
  through its functionality
  \item An observer will record how long it takes for them to figure out
  functionality and how to navigate the application
  \item Users will also be given a survey about the interface after testing

\end{itemize}
					
\item{NFR-LF2 (Device Compatibility)\\}

NFR: LFR-2

Type: Manual

Initial State: Application running on multiple different devices

Input/Condition: Manual tester's engagement with application

Output/Result: A list of inconsistencies through multiple devices

How test will be performed: 
\begin{itemize}
  \item Each member of the team will use the application across multiple
  platforms, and try to identify any inconsistencies between them
\end{itemize}
\end{enumerate}

\subsubsection{Usability and Humanity Requirements}
Usability is also a subjective area, and thus will also require the use of a
user demo assessment.
\begin{enumerate}

  \item{NFR-UH1 (User Experience)\\}
  
  NFR: UHR-1, UHR-4, UHR-5

  Type: User Demo, Manual

  Initial State: Fully functional application ready for user interaction,
  starting at the login page
  
  Input/Condition: User engagement with application
  
  Output/Result: Recorded observations of how the user was able to interact with
  the system without clicking the help button or asking for outside help*
  
  How test will be performed:
  \begin{itemize}
    \item User will be given temporary, working credentials 
    \item User will use the app, starting from the login page, and navigate
    through its functionality
    \item An observer will record how long it takes for them to figure out
    functionality and how to navigate the application
  \end{itemize}
            
  \item{NFR-UH2 (Language and Localization)\\}
  
  NFR: UHR-2
  
  Type: Manual
  
  Initial State: Fully developed application, starting at the login page
  
  Input/Condition: Manual tester using application
  
  Output/Result: A list of all identified language discrepancies
  
  How test will be performed:
  \begin{itemize}
    \item A manual tester will navigate through each web page to ensure that the
    only language being used is English (US)  
  \end{itemize}

  \item{NFR-UH3 (System Notation)\\}

  NFR: UHR-3

  Type: Manual

  Initial State: Application ready to take in data

  Input/Condition: Sample input data

  Output/Result: Error logs from unrecognized characters, or successful upload
  
  How test will be performed:
  \begin{itemize}
    \item A tester will upload a file that has scientific and mathematical symbols
    \item They will note whether or not the file was uploaded successfully, and if the data was transferred correctly, with the symbols
  \end{itemize}

  \item{NFR-UH4 (Accessibility)\\}

  NFR: UHR-6
  
  Type: Manual
  
  Initial State: Application ready for use
  
  Input/Condition: Tester engagement using third party tools
  
  Output/Result: List of accessibility issues in accordance to WCAG, and
  checklists for if page is screen-readable
  
  How test will be performed:
  \begin{itemize}
    \item A tester launch third party tools (NVDA Reader, SiteImprove)
    \item Third party tools will examine web page and provide results
    \end{itemize}
  \end{enumerate}

\subsubsection{Performance Requirements}
These requirements can be tested in ways similar to functional requirements.
Many of these requirements have thresholds defined, making testing a pass/fail
basis.
\newline \newline
Test Data Generation: Some tests will require a large, dummy set of data to be
used as input in order to test how well the system can handle heavy payloads.
Thus, large test sets of data will be automatically generated accordingly.

\begin{enumerate}
  \item{NFR-P1 (Upload Speed)\\}
  
  NFR: PR-1

  Type: Manual

  Initial State: Application navigated to upload page, ready to upload file
  
  Input/Condition: Sample .CSV files for input
  
  Output/Result: Upload duration
  
  How test will be performed:
  \begin{itemize}
    \item Different .CSV files of varying sizes will be uploaded to the system
    \item The upload functionality will include instructions for the system to
    time how long the upload took
    \item The total duration will be logged on the browser's console.

  \end{itemize}
            
  \item{NFR-P2: (Query Response Time)\\}
  
  NFR: PR-2, PR-4

  Type: Manual

  Initial State: Application navigated to querying page, ready to make query
  
  Input/Condition: Test queries
  
  Output/Result: Query response durations

  How test will be performed:
  \begin{itemize}
    \item A set of parameter/dataset combinations will be used as queries
    \item The duration of each query will be logged and compared to defined
    criteria
  \end{itemize}

  \item{NFR-P3 (Website Response Time)\\}

  NFR: PR-3

  Type: Manual

  Initial State: Application ready to use

  Input/Condition: Tester engagement

  Output/Result: Average response time of buttons on website

  How test will be performed:
  \begin{itemize}
    \item A tester will have a timer
    \item They will time the response time of different buttons, and record it
  \end{itemize}

  \item{NFR-P4 (Data Visualization Speed)\\}

  NFR: PR-5

  Type: Manual

  Initial State: Application navigated to query page, ready to make query to
  generate graphs
 
  Input/Condition: Query parameters
  
  Output/Result: Time taken to generate graphs/visualizations
  
  How test will be performed:
  \begin{itemize}
    \item A set of parameter/dataset combinations will be used as queries 
    \item A tester will manually time how long it takes for a graph to be
    generated, or the system will log it in the console
    \end{itemize}

  \item{NFR-P5 (System Accuracy)\\}

  NFR: PR-6, PR-7, PR-8

  Type: Manual

  Initial State: Application navigated to query page, ready to make query
  
  Input/Condition: Query parameters
 
  Output/Result: A check for precision of numbers in different components of
  system
  
  How test will be performed:
  \begin{itemize}
    \item A set of parameter/dataset combinations will be used as queries 
    \item A tester will manually time how long it takes for a graph to be
    generated, or the system will log it in the console
  \end{itemize}

  \item{NFR-P6 (Robustness - Backend Disruption)\\}

  NFR: PR-9
  
  Type: Manual
  
  Initial State: Application running as normal
  
  Input/Condition: Tester temporarily taking down back end
  
  Output/Result: Error message displayed
  
  How test will be performed:
  \begin{itemize}
    \item The tester will run the application as normal
    \item The tester will then disable back end services and ensure error
    messages are generated
  \end{itemize}

  \item{NFR-P7 (Robustness - Internet Disruption)\\}

  NFR: PR-10
  
  Type: Manual
  
  Initial State: Application running as normal
  
  Input/Condition: Tester temporarily disconnects internet connection

  Output/Result: Previously generated plots and previous queries still working
  
  How test will be performed:
  \begin{itemize}
    \item The tester will run the application as normal
    \item The tester will then disconnect their device from their internet connection
    \item They will then try to load previous queries and previously generated plots
  \end{itemize}

  \item{NFR-P8 (User Capacity)\\}
  
  NFR: PR-11

  Type: Manual
  
  Initial State: Three devices ready to run application
  
  Input/Condition: Multiple different queries run by the different devices
  
  Output/Result: System response time while under load
  
  How test will be performed:
  \begin{itemize}
    \item Multiple devices will make queries simultaneously
    \item Page responsiveness will be measured while system runs multiple queries
  \end{itemize}

  \item{NFR-P9 (Storage Capacity)\\}
  
  NFR: PR-12
  
  Type: Data Generation, Automated
  
  Initial State: A database with a known amount of experiment data
  
  Input/Condition: Large suite of dummy test data
  
  Output/Result: Observations on system health after large payload
  
  How test will be performed:
  \begin{itemize}
    \item Multiple, dummy sets of experiment data will be automatically generated
    \item Using a script, this data will be uploaded into the database
    \item System health will be monitored after upload is complete
  \end{itemize}
  \end{enumerate}

\subsubsection{Operational and Environmental Requirements}
  These tests are to ensure that the application works in the expected
  environment, for the expected users. They will be manual tests done by a
  person to check for an environment's compatibility with the system.

\begin{enumerate}

\item{NFR-OE1 (Operating Environment)\\}

NFR: OER-2, OER-3, OER-4

Type: Manual

Initial State: Application running on a windows device as a web application on a
Chromium based browser.

Input/Condition: Tester engagement

Output/Result: A list of all discovered issues with the application that arise
due to environment compatibility

How test will be performed: 
\begin{itemize}
  \item A tester will run the web application on their Windows device, on Google
  Chrome
  \item They will use the application as normal and try to find any issues that
  are caused due to operating environment
\end{itemize}
					
\item{NFR-OE2 (User Onboarding)\\}

NFR: OER-5

Type: Manual, User Demo

Initial State: Application running and ready for use on home screen

Input/Condition: User engagement

Output/Result: Survey results depicting subjective complexity of onboarding
process

How test will be performed: 
\begin{itemize}
  \item A user will go through the onboarding process
  \item A survey will be given after asking the user how complex they found the onboarding to be
\end{itemize}
\end{enumerate}

\subsubsection{Maintainability and Support Requirements}
These tests aim to ensure that future users of the application will not have any
issues using the application. 

\begin{enumerate}
\item{NFR-MS1 (Interface Intuitiveness)\\}

NFR: MSR-4 

Type: Manual, User Demo

Initial State: Application running, ready for use

Input/Condition: User engagement

Output/Result: Observations on user's ability to complete tasks without support

How test will be performed: 
\begin{itemize}
  \item  A pre-defined list of tasks will be created 
  \item A user will be given this set of tasks to perform on their own
  \item The user will be observed to see if they are able to perform these tasks
  without outside help from those conducting the test 
\end{itemize}
					
\item{NFR-MS2 (Cross-Browser Compatibility)\\}

NFR: MSR-6

Type: Manual

Initial State: Multiple Chromium-based web browsers open

Input/Condition: Tester Engagement

Output/Results: All abnormal behaviour of web pages observed on each different
web browser

How test will be performed: 
\begin{itemize}
  \item A tester will use the application as normal on multiple different
  browsers
  \item They will note down any issues that they find, especially if the issues
  are unique to the usage of a certain browser
\end{itemize}
\end{enumerate}

\subsubsection{Security Requirements}
These security assessments are designed to verify that the application operates
securely within the intended environment and meets the needs of its users. These
evaluations will be conducted manually by a member of the \ref{section:VnV_team}
to ensure that the system adheres to established security standards and practices,
identifying any potential vulnerabilities or misconfigurations within the environment.

\begin{enumerate}

\item{NFR-SR1 (Authentication)\\}
  
  NFR: SR-1
  
  Type: Manual
  
  Initial State: Application login page is displayed.
  
  Input/Condition: Tester attempts to access the application with various credentials.
  
  Output/Result: Access is granted or denied based on the validity of the credentials
  provided.
  
  How test will be performed: 
  \begin{itemize}
    \item The tester will attempt to log into the application using both valid and
    invalid credentials.
    \item Valid credentials will be verified against the application's user database.
    \item Invalid credentials will include common mistakes (e.g., incorrect passwords,
    unregistered usernames) to ensure that the system properly restricts access.
    \item The tester will also verify the application's response to unauthorized access
    attempts, noting any error messages or behavior.
    \item The testing will include attempts to access the application without logging
    in to ensure the login mechanism is enforced.
  \end{itemize}

\item{NFR-SR2 (Permissions)\\}
  
  NFR: SR-2, SR-3
  
  Type: Manual
  
  Initial State: Application logged in with multiple user roles (e.g., admin, editor, viewer).
  
  Input/Condition: Tester interacts with the application using on different user roles.
  
  Output/Result: Access to query or modify data and perform sensitive operations is
  restricted according to user roles.
  
  How test will be performed: 
  \begin{itemize}
    \item The tester will attempt to query or modify the same data in the application
    using each role one by one.
    \item For each role:
    \begin{itemize}
      \item Level 1: Should be able to query, modify and export all data.
      \item Level 2: Should be able to query and modify only specific data
      (as per role permissions), should not be able to export any data.
      \item Level 3: Should be able to query data but not modify it, should not be
      able to export any data.
    \end{itemize}
    \item Verify that appropriate error messages are displayed for unauthorized access
    attempts, and check if these events are logged for security auditing.
  \end{itemize}

\item{NFR-SR3 (Timeout)\\}
  
  NFR: SR-4
  
  Type: Manual
  
  Initial State: User is logged into the application.
  
  Input/Condition: User remains inactive for a specified period.
  
  Output/Result: User is automatically logged out after a predefined period of inactivity.

  How test will be performed: 
  \begin{itemize}
    \item The tester will remain idle (not performing any actions such as clicks, key
    presses, or navigations) for a predetermined period that matches the session timeout
    setting (e.g., 10 minutes).
    \item During the inactivity period, the tester will keep track of the time elapsed
    and ensure no session activity occurs.
    \item After the specified inactivity period, the tester will verify that the session
    times out and the user is automatically logged out.
    \item The user must confirm that the user is redirected to the login page and a message
    is displayed, indicating that the session has timed out due to inactivity.
    \item Attempt to navigate to any application page after the timeout occurs without
    logging back in.
    \item Ensure that access is denied, confirming that the session has ended.
  \end{itemize}

\item{NFR-SR4 (Data Input Integrity)\\}
  
  NFR: SR-5, SR-9
  
  Type: Manual
  
  Initial State: Application is open, ready for data entry and CSV upload.
  
  Input/Condition: Tester submits various data entries, including valid and invalid values,
  and uploads CSV files with both valid and invalid data formats.
  
  Output/Result: All invalid inputs and CSV uploads are rejected, and only valid data entries
  are processed.

  How test will be performed: 
  \begin{itemize}
    \item The tester will input data into different fields that accept data inputed directly by
    users, using a variety of both valid and invalid formats (e.g., date formats, numeric fields,
    text fields).
    \item For valid data, ensure the fields accept data in the correct format (e.g., "DD-MM-YYYY"
    for dates, numerical values for numeric fields).
    \item For invalid data, verify that the application rejects these entries with appropriate
    error messages.
    \item Once that is done, the tester uploads a CSV file containing correctly formatted,
    complete data. Confirm that the application accepts the file and processes the entries.
    \item To test the invalid data, the tester:
    \begin{itemize}
      \item \emph{Incorrect Format}: Upload a CSV file with incorrect or mismatched headers.
      Verify that the application rejects the file and provides a descriptive error.
      \item \emph{Corrupted/Incomplete Data}: Upload a CSV file with missing values,
      invalid characters, or mismatched data types (e.g., text in numeric columns).
      Confirm that the application detects and rejects the file due to validation issues.
      \item \emph{Boundary Testing in CSV}: Test with CSV files containing data at boundary
      limits (e.g., maximum character limits in text fields, maximum/minimum numerical values).
    \end{itemize}
  \end{itemize}

\item{NFR-SR5 (Data Validation)\\}
  
  NFR: SR-6, SR-7, SR-8
  
  Type: Manual
  
  Initial State: Application database contains a set of unique, validated records.
  Application interface is open for data entry, processing, and transfer actions.
  
  Input/Condition: Tester processes and transfers various data entries, including
  attempts to introduce duplicate records and test transfer accuracy.
  
  Output/Result: Duplicate records are detected and prevented, and data accuracy
  is maintained during all transfer operations.

  How test will be performed: 
  \begin{itemize}
    \item The tester will trigger any automated processes (e.g., batch processing,
    import) that might introduce duplicates.
    \item The application should flag and reject duplicate entries, ensuring
    they are not added to the system.
    \item After transfer, perform a consistency check by comparing a subset of
    transferred records with the original database.
    \item As part of the consistency check, ensure that no data is modified unnecessarily.
  \end{itemize}

\item{NFR-SR6 (Data Storage Capacity)\\}
  
  NFR: SR-12
  
  Type: Manual/Automated
  
  Initial State: Database is operational, with storage capacity at or below normal
  usage. Alert system is configured, and the administrator contact information is set
  up to receive notifications.

  Input/Condition: Tester simulates increasing database storage usage to exceed the
  80\% threshold.
  
  Output/Result: System successfully detects when storage usage exceeds 80\% and sends
  a timely alert to administrators.

  How test will be performed: 
  \begin{itemize}
    \item Gradually add data to the database to simulate storage increase. This can
    be done by adding records, uploading large files, or running data generation
    scripts until storage usage surpasses the 80\% threshold.
    \item Verify that the system immediately detects the threshold breach and triggers
    an alert to administrators.
    \item Review the alert content to ensure it provides clear information, including:
    \begin{itemize}
      \item Current storage usage percentage.
      \item Implications of reaching the threshold (e.g., potential performance impact).
      \item Recommended actions for the administrator (e.g., freeing up space or
      provisioning additional storage).
    \end{itemize}
    \item After testing, reduce storage usage (e.g., by deleting test data) to observe
    if the system updates the storage capacity status accordingly.
    \item Ensure that no system interruptions or crashes occur during and after the alert,
    verifying that the system remains functional even when approaching capacity limits.
  \end{itemize}

\item{NFR-SR7 (System Audits)\\}
  
  NFR: SR-13, SR-14, SR-15
  
  Type: Manual/Automated
  
  Initial State: The application is operational, with logging features configured and
  permissions for accessing logs assigned to administrators only.
  
  Input/Condition: Tester performs various access and modification actions within the
  application, then attempts to access the audit logs with both authorized and
  unauthorized user accounts.
  
  Output/Result: All actions are logged with timestamps and user identities, and audit
  logs are accessible only to authorized users, with proper encryption.

  How test will be performed: 
  \begin{itemize}
    \item Tester performs a range of actions within the application, including logging
    in and out of the application, accessing different data records and modifying
    specific data fields.
    \item Verify that an entry is created in the audit log capturing the type of action
    (e.g., login, access, modification), timestamp, and user identity.
    \item Using an authorized account with administrative privileges, retrieve the audit
    logs to verify the following:
    \begin{itemize}
      \item That each performed action is accurately recorded.
      \item That no events are missing, confirming 100\% coverage of all access and
      modification events.
      \item Check for any inconsistencies or inaccuracies in timestamps or user identifiers.
    \end{itemize}
    \item The tester attempts to access the audit logs using a non-administrative (unauthorized)
    account.
    \item The system should deny access to the audit logs, displaying an error or access-restricted
    message.
    \item The tester attempts to modify an audit log entry (if possible) to ensure that the logs are
    tamper-resistant. The system should prevent any unauthorized changes, preserving the integrity of
    the logs.
  \end{itemize}

\item{NFR-SR8 (Intrusion Prevention)\\}
  
  NFR: SR-16
  
  Type: Manual
  
  Initial State: Application login screen is open. Administrator contact information is configured
  to receive security alerts.
  
  Input/Condition: Tester attempts multiple failed logins to trigger the suspicious activity
  detection mechanism.
  
  Output/Result:  Application detects and blocks access after three failed attempts, sends an
  alert to administrators, and locks out the user temporarily.

  How test will be performed: 
  \begin{itemize}
    \item Using an invalid username-password combination, the tester attempts to log in
    repeatedly.
    \item After each failed attempt, verify that the system accurately counts the login failures.
    \item On the fourth failed attempt, the application should block further login attempts for
    that user account or IP address temporarily.
    \item  Confirm that an alert is sent to administrators upon detecting the suspicious activity.
    \item The alert should provide details about the incident, such as:
    \begin{itemize}
      \item IP address and location.
      \item Timestamp of the failed attempts.
      \item Username or account targeted.
    \end{itemize}
    \item Check that each failed login attempt and the subsequent lockout are recorded in the
    audit logs, providing a clear trail for security auditing.
  \end{itemize}

\item{NFR-SR9 (Resource Optimization)\\}
  
  NFR: SR-17
  
  Type: Automated/Manual
  
  Initial State: Application is running under typical workload conditions. Monitoring tools
  and alerts for CPU and memory usage are enabled.
  
  Input/Condition: Tester simulates a high workload to approach system resource limits,
  observing the system's monitoring and optimization response.
  
  Output/Result: The system actively manages CPU and memory usage, preventing overload and
  maintaining performance stability.

  How test will be performed: 
  \begin{itemize}
    \item Gradually increase the workload on the system (e.g., by running intensive processes,
    generating simultaneous user sessions, or processing large data sets) until CPU and memory
    usage approach high utilization levels.
    \item  Observe the system's resource usage in real time to confirm that CPU and memory
    metrics are accurately monitored.
    \item Confirm that the system remains stable, responsive, and does not crash or slow down
    significantly during and after the high-load test.
    \item As resource usage approaches critical levels (e.g., 90\%), verify that the system
    triggers alerts to administrators, notifying them of potential overload risks.
    \item Check that the system dynamically adjusts resource allocation to manage load (e.g.,
    by limiting non-critical processes or optimizing memory usage).
    \item Post this, CPU and memory usage should stabilize below critical thresholds (e.g.,
    preventing usage from exceeding 90\%) due to the system's proactive adjustments.
    \item Check that all significant resource usage events, optimization actions, and alerts
    are logged for auditing and system performance analysis.
  \end{itemize}

\end{enumerate}

\subsection{Traceability Between Test Cases and Requirements}

\begin{longtable}{|p{0.45\linewidth}|p{0.45\linewidth}|}
  \caption{Traceability Matrix for Test Cases and Requirements} \label{table:traceability}\\
  \hline
  \textbf{Req. ID} & \textbf{System Test ID} \\
  \hline
  FR-1 & FR-SLN1\\
  FR-2 & \\
  FR-3 & \\
  FR-4 & \\
  \hline
  FR-5 & FR-SLN2\\
  FR-6 & \\
  FR-7 & \\
  FR-8 & \\
  FR-9 & \\
  \hline
  FR-10 & FR-SL3\\
  FR-11 & \\
  \hline
  FR-12 & FR-SL4\\
  FR-13 & \\
  \hline
  FR-14 & FR-SL5\\
  \hline
  FR-15 & FR-SL6\\
  \hline
  UHR-1 & NFR-UH1\\
  UHR-4 & \\
  UHR-5 & \\
  \hline
  UHR-2 & NFR-UH2\\
  \hline
  UHR-3 & NFR-UH3\\
  \hline
  UHR-6 & NFR-UH4\\
  \hline
  PR-1 & NFR-P1\\
  \hline
  PR-2 & NFR-P2\\
  PR-4 & \\
  \hline
  PR-3 & NFR-P3\\
  \hline
  PR-5 & NFR-P4\\
  \hline
  PR-6 & NFR-P5\\
  PR-7 & \\
  PR-8 & \\
  \hline
  PR-9 & NFR-P6\\
  \hline
  PR-10 & NFR-P7\\
  \hline
  PR-11 & NFR-P8\\
  \hline
  PR-12 & NFR-P9\\
  \hline
  OER-2 & NFR-OE1\\
  OER-3 & \\
  OER-4 & \\
  \hline
  OER-5 & NFR-OE2\\
  \hline
  MSR-4 & NFR-MSR1\\
  \hline
  MSR-6 & NFR-MSR2\\
  \hline
  SR-1 & NFR-SR1\\
  \hline
  SR-2 & NFR-SR2\\
  SR-3 & \\
  \hline
  SR-4 & NFR-SR3\\
  \hline
  SR-5 & NFR-SR4\\
  SR-9 & \\
  \hline
  SR-6 & NFR-SR5\\
  SR-7 & \\
  SR-8 & \\
  \hline
  SR-12 & NFR-SR6\\
  \hline
  SR-13 & NFR-SR7\\
  SR-14 & \\
  SR-15 & \\
  \hline
  SR-16 & NFR-SR8\\
  \hline
  SR-17 & NFR-SR9\\
  \hline

\end{longtable}

\section{Unit Test Description}

\wss{This section should not be filled in until after the MIS (detailed design
  document) has been completed.}

\wss{Reference your MIS (detailed design document) and explain your overall
philosophy for test case selection.}  

\wss{To save space and time, it may be an option to provide less detail in this section.  
For the unit tests you can potentially layout your testing strategy here.  That is, you 
can explain how tests will be selected for each module.  For instance, your test building 
approach could be test cases for each access program, including one test for normal behaviour 
and as many tests as needed for edge cases.  Rather than create the details of the input 
and output here, you could point to the unit testing code.  For this to work, you code 
needs to be well-documented, with meaningful names for all of the tests.}

\subsection{Unit Testing Scope}

\wss{What modules are outside of the scope.  If there are modules that are
  developed by someone else, then you would say here if you aren't planning on
  verifying them.  There may also be modules that are part of your software, but
  have a lower priority for verification than others.  If this is the case,
  explain your rationale for the ranking of module importance.}

\subsection{Tests for Functional Requirements}

\wss{Most of the verification will be through automated unit testing.  If
  appropriate specific modules can be verified by a non-testing based
  technique.  That can also be documented in this section.}

\subsubsection{Module 1}

\wss{Include a blurb here to explain why the subsections below cover the module.
  References to the MIS would be good.  You will want tests from a black box
  perspective and from a white box perspective.  Explain to the reader how the
  tests were selected.}

\begin{enumerate}

\item{test-id1\\}

Type: \wss{Functional, Dynamic, Manual, Automatic, Static etc. Most will
  be automatic}
					
Initial State: 
					
Input: 
					
Output: \wss{The expected result for the given inputs}

Test Case Derivation: \wss{Justify the expected value given in the Output field}

How test will be performed: 
					
\item{test-id2\\}

Type: \wss{Functional, Dynamic, Manual, Automatic, Static etc. Most will
  be automatic}
					
Initial State: 
					
Input: 
					
Output: \wss{The expected result for the given inputs}

Test Case Derivation: \wss{Justify the expected value given in the Output field}

How test will be performed: 

\item{...\\}
    
\end{enumerate}

\subsubsection{Module 2}

...

\subsection{Tests for Nonfunctional Requirements}

\wss{If there is a module that needs to be independently assessed for
  performance, those test cases can go here.  In some projects, planning for
  nonfunctional tests of units will not be that relevant.}

\wss{These tests may involve collecting performance data from previously
  mentioned functional tests.}

\subsubsection{Module ?}
		
\begin{enumerate}

\item{test-id1\\}

Type: \wss{Functional, Dynamic, Manual, Automatic, Static etc. Most will
  be automatic}
					
Initial State: 
					
Input/Condition: 
					
Output/Result: 
					
How test will be performed: 
					
\item{test-id2\\}

Type: Functional, Dynamic, Manual, Static etc.
					
Initial State: 
					
Input: 
					
Output: 
					
How test will be performed: 

\end{enumerate}

\subsubsection{Module ?}

...

\subsection{Traceability Between Test Cases and Modules}

\wss{Provide evidence that all of the modules have been considered.}
				
\bibliographystyle{plainnat}

\bibliography{../../refs/References}

\newpage

\section{Appendix}

This is where you can place additional information.

\subsection{Symbolic Parameters}

The definition of the test cases will call for SYMBOLIC\_CONSTANTS.
Their values are defined in this section for easy maintenance.

\subsection{Usability Survey Questions?}

\wss{This is a section that would be appropriate for some projects.}

\newpage{}
\section*{Appendix --- Reflection}

\wss{This section is not required for CAS 741}

The information in this section will be used to evaluate the team members on the
graduate attribute of Lifelong Learning.

The purpose of reflection questions is to give you a chance to assess your own
learning and that of your group as a whole, and to find ways to improve in the
future. Reflection is an important part of the learning process.  Reflection is
also an essential component of a successful software development process.  

Reflections are most interesting and useful when they're honest, even if the
stories they tell are imperfect. You will be marked based on your depth of
thought and analysis, and not based on the content of the reflections
themselves. Thus, for full marks we encourage you to answer openly and honestly
and to avoid simply writing ``what you think the evaluator wants to hear.''

Please answer the following questions.  Some questions can be answered on the
team level, but where appropriate, each team member should write their own
response:


\begin{enumerate}
  \item What went well while writing this deliverable? 
  \item What pain points did you experience during this deliverable, and how
    did you resolve them?
  \item What knowledge and skills will the team collectively need to acquire to
  successfully complete the verification and validation of your project?
  Examples of possible knowledge and skills include dynamic testing knowledge,
  static testing knowledge, specific tool usage, Valgrind etc.  You should look to
  identify at least one item for each team member.
  \item For each of the knowledge areas and skills identified in the previous
  question, what are at least two approaches to acquiring the knowledge or
  mastering the skill?  Of the identified approaches, which will each team
  member pursue, and why did they make this choice?
\end{enumerate}

\end{document}
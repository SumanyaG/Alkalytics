\documentclass[12pt, titlepage]{article}

\usepackage{booktabs}
\usepackage{array}
\usepackage{caption}
\usepackage{amssymb}
\usepackage{float}
\usepackage{tabularx}
\usepackage{hyperref}
\hypersetup{
    colorlinks,
    citecolor=blue,
    filecolor=black,
    linkcolor=red,
    urlcolor=blue
}
\usepackage[round]{natbib}

%% Comments

\usepackage{color}

\newif\ifcomments\commentstrue %displays comments
%\newif\ifcomments\commentsfalse %so that comments do not display

\ifcomments
\newcommand{\authornote}[3]{\textcolor{#1}{[#3 ---#2]}}
\newcommand{\todo}[1]{\textcolor{red}{[TODO: #1]}}
\else
\newcommand{\authornote}[3]{}
\newcommand{\todo}[1]{}
\fi

\newcommand{\wss}[1]{\authornote{blue}{SS}{#1}} 
\newcommand{\plt}[1]{\authornote{magenta}{TPLT}{#1}} %For explanation of the template
\newcommand{\an}[1]{\authornote{cyan}{Author}{#1}}

%% Common Parts

\newcommand{\progname}{Software Engineering} % PUT YOUR PROGRAM NAME HERE
\newcommand{\authname}{Team 21, Alkalytics
\\ Sumanya Gulati - gulats10
\\ Kate Min - mink9
\\ Jennifer Ye - yej52
\\ Jason Tran - tranj78} % AUTHOR NAMES                  

\usepackage{hyperref}
    \hypersetup{colorlinks=true, linkcolor=blue, citecolor=blue, filecolor=blue,
                urlcolor=blue, unicode=false}
    \urlstyle{same}
                                


\begin{document}

\title{System Verification and Validation Plan for \progname{}} 
\author{\authname}
\date{\today}
	
\maketitle

\pagenumbering{roman}

\section*{Revision History}

\begin{tabularx}{\textwidth}{p{3cm}p{2cm}X}
\toprule {\bf Date} & {\bf Version} & {\bf Notes}\\
\midrule
Date 1 & 1.0 & Notes\\
Date 2 & 1.1 & Notes\\
\bottomrule
\end{tabularx}

~\\
\wss{The intention of the VnV plan is to increase confidence in the software.
However, this does not mean listing every verification and validation technique
that has ever been devised.  The VnV plan should also be a \textbf{feasible}
plan. Execution of the plan should be possible with the time and team available.
If the full plan cannot be completed during the time available, it can either be
modified to ``fake it'', or a better solution is to add a section describing
what work has been completed and what work is still planned for the future.}

\wss{The VnV plan is typically started after the requirements stage, but before
the design stage.  This means that the sections related to unit testing cannot
initially be completed.  The sections will be filled in after the design stage
is complete.  the final version of the VnV plan should have all sections filled
in.}

\newpage

\tableofcontents

\listoftables
\wss{Remove this section if it isn't needed}

\listoffigures
\wss{Remove this section if it isn't needed}

\newpage

\section{Symbols, Abbreviations, and Acronyms}

\renewcommand{\arraystretch}{1.2}
\begin{tabular}{l l} 
  \toprule		
  \textbf{symbol} & \textbf{description}\\
  \midrule 
  T & Test\\
  \bottomrule
\end{tabular}\\

\wss{symbols, abbreviations, or acronyms --- you can simply reference the SRS
  \citep{SRS} tables, if appropriate}

\wss{Remove this section if it isn't needed}

\newpage

\pagenumbering{arabic}

This document ... \wss{provide an introductory blurb and roadmap of the
  Verification and Validation plan}

\section{General Information}

\subsection{Summary}

\wss{Say what software is being tested.  Give its name and a brief overview of
  its general functions.}

\subsection{Objectives}

\wss{State what is intended to be accomplished.  The objective will be around
  the qualities that are most important for your project.  You might have
  something like: ``build confidence in the software correctness,''
  ``demonstrate adequate usability.'' etc.  You won't list all of the qualities,
  just those that are most important.}

\wss{You should also list the objectives that are out of scope.  You don't have 
the resources to do everything, so what will you be leaving out.  For instance, 
if you are not going to verify the quality of usability, state this.  It is also 
worthwhile to justify why the objectives are left out.}

\wss{The objectives are important because they highlight that you are aware of 
limitations in your resources for verification and validation.  You can't do everything, 
so what are you going to prioritize?  As an example, if your system depends on an 
external library, you can explicitly state that you will assume that external library 
has already been verified by its implementation team.}

\subsection{Challenge Level and Extras}

\wss{State the challenge level (advanced, general, basic) for your project.
Your challenge level should exactly match what is included in your problem
statement.  This should be the challenge level agreed on between you and the
course instructor.  You can use a pull request to update your challenge level
(in TeamComposition.csv or Repos.csv) if your plan changes as a result of the
VnV planning exercise.}

\wss{Summarize the extras (if any) that were tackled by this project.  Extras
can include usability testing, code walkthroughs, user documentation, formal
proof, GenderMag personas, Design Thinking, etc.  Extras should have already
been approved by the course instructor as included in your problem statement.
You can use a pull request to update your extras (in TeamComposition.csv or
Repos.csv) if your plan changes as a result of the VnV planning exercise.}

\subsection{Relevant Documentation}

\wss{Reference relevant documentation.  This will definitely include your SRS
  and your other project documents (design documents, like MG, MIS, etc).  You
  can include these even before they are written, since by the time the project
  is done, they will be written.  You can create BibTeX entries for your
  documents and within those entries include a hyperlink to the documents.}

\citet{SRS}

\wss{Don't just list the other documents.  You should explain why they are relevant and 
how they relate to your VnV efforts.}

\section{Plan}
This section outlines the Verification and Validation (VnV) responsibilities of each 
team member and the team supervisor. It presents comprehensive verification plans for the 
Software Requirements Specification (SRS), design documents (including MIS, MG, and System Design),
the VnV plan itself, and the system implementation.\\
\newline Additionally, the automated testing and verification tools intended for use in the VnV process
have been specified. Verification tasks will be carried out by the VnV team in a sequential manner
as each phase progresses, with continuous verification of the SRS and design documents as source code
is developed. Software validation will be conducted in parallel with other VnV activities, 
supporting the Proof of Concept (PoC) and final demonstration phases.

\subsection{Verification and Validation Team}
In this section, Table \ref{table:vnv_team} presents the members of the VnV team designated to carry
out the tasks specified in this document. For each team member, their role in the project's verification
process is outlined, with key details highlighting their respective responsibilities.

\begin{table}[H]
  \centering
  \caption{Validation and Verification Team Members and Responsibilities}
  \label{table:vnv_team}
  \begin{tabularx}{\textwidth}{|>{\centering\arraybackslash}p{0.3\textwidth}|>{\centering\arraybackslash}X|}
      \hline
      \textbf{Team Member} & \textbf{Role and Responsibilities} \\
      \hline
      Bassel Abdelkader & Advisor and one of the primary reviewers of documentation; contributor to the
      validation of user-based testing, providing suggestions and feedback to improve software functionality
      and user experience. \\
      \hline
      Dr. Charles de Lannoy & Advisor and one of the primary reviewers of documentation; contributor to the
      validation of user-based testing, providing suggestions and feedback to improve software functionality
      and user experience. \\
      \hline
      Jason Tran & Review the work of other team members to uphold high standards, provide suggestions for
      improvement, and maintain feedback checklists for each work item, with an emphasis on backend code. \\
      \hline
      Jennifer Ye & Review the work of other team members to uphold high standards, provide suggestions for
      improvement, and maintain feedback checklists for each work item, with an emphasis on frontend code. \\
      \hline
      Kate Min & Review the work of other team members to uphold high standards, provide suggestions for
      improvement, and maintain feedback checklists for each work item, with an emphasis on unit testing. \\
      \hline
      Sumanya Gulati & Review the work of other team members to uphold high standards, provide suggestions
      for improvement, and maintain feedback checklists for each work item, with an emphasis on the documentation
      and VnV Plan. \\
      \hline
      Other Design Teams & Peer reviewers identify issues and offer feedback and suggestions for improving
      documentation. \\
      \hline
  \end{tabularx}
\end{table}

\subsection{SRS Verification Plan}

The Software Requirements Specification (SRS) is essential for defining the project scope and guiding
implementation. To ensure it aligns with project standards and objectives, a rigorous verification
plan will be implemented, incorporating iterative feedback from peers and our assigned Teaching Assistant,
Chris Schankula.\\
\newline
The verification process will begin with an initial review to evaluate the SRS for
clarity, completeness, and alignment with project goals. Feedback will be systematically gathered and
categorized to identify areas needing refinement. Exploratory assessment techniques will allow reviewers
to interact with application prototypes and design mockups, ensuring requirements are realistically
translated into application features.

\subsubsection{Functional Testing}
Functional testing based on the SRS will validate application behavior, confirming that each function
performs according to specified requirements. The feedback will be integrated into the SRS, leading to
revisions that enhance clarity and accuracy. A comprehensive final review will ensure all feedback
has been effectively addressed, solidifying the SRS as a true reflection of the intended system.

\subsubsection{Formal and Ad-hoc Reviews}
The verification approach includes formal and ad-hoc feedback methods. Formal reviews will create
updated checklists based on existing ones and grading feedback, while review meetings with
the supervisor will identify mistakes and suggest improvements. Ad-hoc peer reviews from other design
teams will also provide valuable insights. Key verification checks will ensure the adequacy,
feasibility, and verifiability of requirements, along with traceability to use cases.

\subsubsection{Checklist}
An initial verification plan checklist will be maintained and updated over time, focusing on detailed
descriptions, relevance, traceability, verifiability, and feasibility of requirements, as well as
closing all reviewer-identified issues. By leveraging diverse insights, the SRS will be refined
to facilitate a smooth transition into the development phase, laying a strong foundation for the
project's success.\\
\newline
The following checklist has been created as an initial draft that will be iteratively updated as SRS
reviews are completed over time:
\begin{itemize}
  \item[$\square$] Does each functional requirement have a clear and precise description?
  \item[$\square$] Are the rationales for each requirement clearly articulated?
  \item[$\square$] Is there a defined fit criterion for each requirement that specifies success criteria?
  \item[$\square$] Are all functional requirements complete and unambiguous?
  \item[$\square$] Are all requirements consistent with one another, avoiding conflicts?
  \item[$\square$] Are all dependencies on external systems or components identified and addressed?
\end{itemize}

\subsection{Design Verification Plan}
This section delineates the strategies and procedures the team will employ to verify the
correctness and reliability of the design of the Alkalytics application. This plan will
serve as a guideline during the testing phase to ensure that the design aligns with the
intended requirements and effectively mitigates potential hazards identified by the VnV team.

\subsubsection{Document Review}
After the completion of the initial draft of the design documentation (including the Management
Guide (MG), Management Information System (MIS), and System Design documents), each member of
the testing team will conduct a comprehensive review before submission. The objectives of this
review process are to:
\begin{itemize}
  \item Ensure that the system design aligns with all functional and non-functional requirements.
  \item Assess the accuracy of the documentation in describing the intended functionality and
  behavior of the system.
  \item Record and report any design elements that deviate from specified requirements for
  further discussion.
\end{itemize}

\subsubsection{Review Meetings}
A structured review meeting will be conducted with the supervisors once the design documents
are completed. Additionally, peer reviews from classmates will provide critical suggestions for
improvement.

\subsubsection{Code Conformity Verification}
The team will verify that the code adheres to SOLID design principles, ensuring modular and
maintainable code structures.

\subsubsection{Formal Team Review}
A formal review with team members will occur after initial document creation, allowing for
reflection on the design prior to final review. Checklists will be utilized to compare the
design documents against the Software Requirements Specification (SRS) post-verification.

\subsubsection{Checklist}
This checklist will be updated as reviews progress to ensure comprehensive verification of
the design documents:
\begin{itemize}
  \item[$\square$] Are all requirements (functional and non-functional) traceable to at least one
  implementing module in the MG?
  \item[$\square$] Have all issues raised by reviewers been addressed and resolved?
  \item[$\square$] Do all modules and components conform to the SOLID design principles?
  \item[$\square$] Are all modules assigned unambiguous tasks with well-defined inputs and outputs?
  \item[$\square$] Is the design reflective of the database structure and functionality required for
  the application?
\end{itemize}

\subsection{Verification and Validation Plan Verification Plan}
The Verification and Validation (V\&V) plan, i.e., this document, must also have
a verification plan to ensure its correctness, completeness, and feasibility.
The following methods will be applied to verify the document:
\begin{itemize}
  \item \textbf{Internal document review}: All team members review each section
  to ensure quality and provide feedback/suggestions for improvement.
  \item \textbf{Peer review by classmates}: Another team reviews the contents of
  the V\&V plan and provide feedback/suggestions for improvement.
  \item \textbf{Feedback integration}: The team refines the V\&V plan
  appropriately after reviewing feedback received from internal reviews, peer
  reviews, and the grading Teaching Assistant (TA).
  \item \textbf{Mutation testing}: The team performs mutation testing by
  injecting mutations (i.e., faults) into the tests to evaluate whether the
  tests can detect the mutant and verify if its behaviour is the expected
  outcome.
\end{itemize}
The following checklist serves as a guide for verifying the V\&V plan:
\begin{itemize}
  \item Each verification plan is complete, feasible, and unambiguous.
  \item Roles and responsibilities for verification are explicitly and clearly
  defined.
  \item Testing plans cover all defined requirements.
  \item Each test specifies clear inputs, expected outputs, and criteria for
  pass/fail.
  \item Test procedures are well-described.
\end{itemize}

\subsection{Implementation Verification Plan}

\wss{You should at least point to the tests listed in this document and the unit
  testing plan.}

\wss{In this section you would also give any details of any plans for static
  verification of the implementation.  Potential techniques include code
  walkthroughs, code inspection, static analyzers, etc.}

\wss{The final class presentation in CAS 741 could be used as a code
walkthrough.  There is also a possibility of using the final presentation (in
CAS741) for a partial usability survey.}

\subsection{Automated Testing and Verification Tools}

\wss{What tools are you using for automated testing.  Likely a unit testing
  framework and maybe a profiling tool, like ValGrind.  Other possible tools
  include a static analyzer, make, continuous integration tools, test coverage
  tools, etc.  Explain your plans for summarizing code coverage metrics.
  Linters are another important class of tools.  For the programming language
  you select, you should look at the available linters.  There may also be tools
  that verify that coding standards have been respected, like flake9 for
  Python.}

\wss{If you have already done this in the development plan, you can point to
that document.}

\wss{The details of this section will likely evolve as you get closer to the
  implementation.}

\subsection{Software Validation Plan}

\wss{If there is any external data that can be used for validation, you should
  point to it here.  If there are no plans for validation, you should state that
  here.}

\wss{You might want to use review sessions with the stakeholder to check that
the requirements document captures the right requirements.  Maybe task based
inspection?}

\wss{For those capstone teams with an external supervisor, the Rev 0 demo should 
be used as an opportunity to validate the requirements.  You should plan on 
demonstrating your project to your supervisor shortly after the scheduled Rev 0 demo.  
The feedback from your supervisor will be very useful for improving your project.}

\wss{For teams without an external supervisor, user testing can serve the same purpose 
as a Rev 0 demo for the supervisor.}

\wss{This section might reference back to the SRS verification section.}

\section{System Tests}

\wss{There should be text between all headings, even if it is just a roadmap of
the contents of the subsections.}

\subsection{Tests for Functional Requirements}

\wss{Subsets of the tests may be in related, so this section is divided into
  different areas.  If there are no identifiable subsets for the tests, this
  level of document structure can be removed.}

\wss{Include a blurb here to explain why the subsections below
  cover the requirements.  References to the SRS would be good here.}

\subsubsection{Area of Testing1}

\wss{It would be nice to have a blurb here to explain why the subsections below
  cover the requirements.  References to the SRS would be good here.  If a section
  covers tests for input constraints, you should reference the data constraints
  table in the SRS.}
		
\paragraph{Title for Test}

\begin{enumerate}

\item{test-id1\\}

Control: Manual versus Automatic
					
Initial State: 
					
Input: 
					
Output: \wss{The expected result for the given inputs.  Output is not how you
are going to return the results of the test.  The output is the expected
result.}

Test Case Derivation: \wss{Justify the expected value given in the Output field}
					
How test will be performed: 
					
\item{test-id2\\}

Control: Manual versus Automatic
					
Initial State: 
					
Input: 
					
Output: \wss{The expected result for the given inputs}

Test Case Derivation: \wss{Justify the expected value given in the Output field}

How test will be performed: 

\end{enumerate}

\subsubsection{Area of Testing2}

...

\subsection{Tests for Nonfunctional Requirements}

\wss{The nonfunctional requirements for accuracy will likely just reference the
  appropriate functional tests from above.  The test cases should mention
  reporting the relative error for these tests.  Not all projects will
  necessarily have nonfunctional requirements related to accuracy.}

\wss{For some nonfunctional tests, you won't be setting a target threshold for
passing the test, but rather describing the experiment you will do to measure
the quality for different inputs.  For instance, you could measure speed versus
the problem size.  The output of the test isn't pass/fail, but rather a summary
table or graph.}

\wss{Tests related to usability could include conducting a usability test and
  survey.  The survey will be in the Appendix.}

\wss{Static tests, review, inspections, and walkthroughs, will not follow the
format for the tests given below.}

\wss{If you introduce static tests in your plan, you need to provide details.
How will they be done?  In cases like code (or document) walkthroughs, who will
be involved? Be specific.}

\subsubsection{Area of Testing1}
		
\paragraph{Title for Test}

\begin{enumerate}

\item{test-id1\\}

Type: Functional, Dynamic, Manual, Static etc.
					
Initial State: 
					
Input/Condition: 
					
Output/Result: 
					
How test will be performed: 
					
\item{test-id2\\}

Type: Functional, Dynamic, Manual, Static etc.
					
Initial State: 
					
Input: 
					
Output: 
					
How test will be performed: 

\end{enumerate}

\subsubsection{Area of Testing2}

...

\subsection{Traceability Between Test Cases and Requirements}

\wss{Provide a table that shows which test cases are supporting which
  requirements.}

\section{Unit Test Description}

\wss{This section should not be filled in until after the MIS (detailed design
  document) has been completed.}

\wss{Reference your MIS (detailed design document) and explain your overall
philosophy for test case selection.}  

\wss{To save space and time, it may be an option to provide less detail in this section.  
For the unit tests you can potentially layout your testing strategy here.  That is, you 
can explain how tests will be selected for each module.  For instance, your test building 
approach could be test cases for each access program, including one test for normal behaviour 
and as many tests as needed for edge cases.  Rather than create the details of the input 
and output here, you could point to the unit testing code.  For this to work, you code 
needs to be well-documented, with meaningful names for all of the tests.}

\subsection{Unit Testing Scope}

\wss{What modules are outside of the scope.  If there are modules that are
  developed by someone else, then you would say here if you aren't planning on
  verifying them.  There may also be modules that are part of your software, but
  have a lower priority for verification than others.  If this is the case,
  explain your rationale for the ranking of module importance.}

\subsection{Tests for Functional Requirements}

\wss{Most of the verification will be through automated unit testing.  If
  appropriate specific modules can be verified by a non-testing based
  technique.  That can also be documented in this section.}

\subsubsection{Module 1}

\wss{Include a blurb here to explain why the subsections below cover the module.
  References to the MIS would be good.  You will want tests from a black box
  perspective and from a white box perspective.  Explain to the reader how the
  tests were selected.}

\begin{enumerate}

\item{test-id1\\}

Type: \wss{Functional, Dynamic, Manual, Automatic, Static etc. Most will
  be automatic}
					
Initial State: 
					
Input: 
					
Output: \wss{The expected result for the given inputs}

Test Case Derivation: \wss{Justify the expected value given in the Output field}

How test will be performed: 
					
\item{test-id2\\}

Type: \wss{Functional, Dynamic, Manual, Automatic, Static etc. Most will
  be automatic}
					
Initial State: 
					
Input: 
					
Output: \wss{The expected result for the given inputs}

Test Case Derivation: \wss{Justify the expected value given in the Output field}

How test will be performed: 

\item{...\\}
    
\end{enumerate}

\subsubsection{Module 2}

...

\subsection{Tests for Nonfunctional Requirements}

\wss{If there is a module that needs to be independently assessed for
  performance, those test cases can go here.  In some projects, planning for
  nonfunctional tests of units will not be that relevant.}

\wss{These tests may involve collecting performance data from previously
  mentioned functional tests.}

\subsubsection{Module ?}
		
\begin{enumerate}

\item{test-id1\\}

Type: \wss{Functional, Dynamic, Manual, Automatic, Static etc. Most will
  be automatic}
					
Initial State: 
					
Input/Condition: 
					
Output/Result: 
					
How test will be performed: 
					
\item{test-id2\\}

Type: Functional, Dynamic, Manual, Static etc.
					
Initial State: 
					
Input: 
					
Output: 
					
How test will be performed: 

\end{enumerate}

\subsubsection{Module ?}

...

\subsection{Traceability Between Test Cases and Modules}

\wss{Provide evidence that all of the modules have been considered.}
				
\bibliographystyle{plainnat}

\bibliography{../../refs/References}

\newpage

\section{Appendix}

This is where you can place additional information.

\subsection{Symbolic Parameters}

The definition of the test cases will call for SYMBOLIC\_CONSTANTS.
Their values are defined in this section for easy maintenance.

\subsection{Usability Survey Questions?}

\wss{This is a section that would be appropriate for some projects.}

\newpage{}
\section*{Appendix --- Reflection}

\wss{This section is not required for CAS 741}

The information in this section will be used to evaluate the team members on the
graduate attribute of Lifelong Learning.

The purpose of reflection questions is to give you a chance to assess your own
learning and that of your group as a whole, and to find ways to improve in the
future. Reflection is an important part of the learning process.  Reflection is
also an essential component of a successful software development process.  

Reflections are most interesting and useful when they're honest, even if the
stories they tell are imperfect. You will be marked based on your depth of
thought and analysis, and not based on the content of the reflections
themselves. Thus, for full marks we encourage you to answer openly and honestly
and to avoid simply writing ``what you think the evaluator wants to hear.''

Please answer the following questions.  Some questions can be answered on the
team level, but where appropriate, each team member should write their own
response:


\begin{enumerate}
  \item What went well while writing this deliverable? 
  \item What pain points did you experience during this deliverable, and how
    did you resolve them?
  \item What knowledge and skills will the team collectively need to acquire to
  successfully complete the verification and validation of your project?
  Examples of possible knowledge and skills include dynamic testing knowledge,
  static testing knowledge, specific tool usage, Valgrind etc.  You should look to
  identify at least one item for each team member.
  \item For each of the knowledge areas and skills identified in the previous
  question, what are at least two approaches to acquiring the knowledge or
  mastering the skill?  Of the identified approaches, which will each team
  member pursue, and why did they make this choice?
\end{enumerate}

\end{document}
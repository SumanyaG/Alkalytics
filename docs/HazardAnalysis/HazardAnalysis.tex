\documentclass{article}

\usepackage{booktabs}
\usepackage{tabularx}
\usepackage{hyperref}

\hypersetup{
    colorlinks=true,       % false: boxed links; true: colored links
    linkcolor=red,          % color of internal links (change box color with linkbordercolor)
    citecolor=green,        % color of links to bibliography
    filecolor=magenta,      % color of file links
    urlcolor=cyan           % color of external links
}

\title{Hazard Analysis\\\progname}

\author{\authname}

\date{}

\input{../Comments}
%% Common Parts

\newcommand{\progname}{Software Engineering} % PUT YOUR PROGRAM NAME HERE
\newcommand{\authname}{Team 21, Alkalytics
\\ Sumanya Gulati - gulats10
\\ Kate Min - mink9
\\ Jennifer Ye - yej52
\\ Jason Tran - tranj78} % AUTHOR NAMES                  

\usepackage{hyperref}
    \hypersetup{colorlinks=true, linkcolor=blue, citecolor=blue, filecolor=blue,
                urlcolor=blue, unicode=false}
    \urlstyle{same}
                                


\begin{document}

\maketitle
\thispagestyle{empty}

~\newpage

\pagenumbering{roman}

\begin{table}[hp]
\caption{Revision History} \label{TblRevisionHistory}
\begin{tabularx}{\textwidth}{llX}
\toprule
\textbf{Date} & \textbf{Developer(s)} & \textbf{Change}\\
\midrule
Date1 & Name(s) & Description of changes\\
Date2 & Name(s) & Description of changes\\
... & ... & ...\\
\bottomrule
\end{tabularx}
\end{table}

~\newpage

\tableofcontents

~\newpage

\pagenumbering{arabic}

\wss{You are free to modify this template.}

\section{Introduction}

A hazard is any property or condition that has the potential to cause harm. This
document serves as a hazard analysis for the application revolving around the
capstone project "Alkalytics". This project aims to aids in the data management
and analysis of an ocean alkalinity enhancement experiment. This document
identifies the components of the system, and then the possible software hazards
in these components, as well as ways to mitigate the risks they impose.
\section{Scope and Purpose of Hazard Analysis}

\wss{You should say what \textbf{loss} could be incurred because of the
hazards.}

\section{System Boundaries and Components}

\wss{Dividing the system into components will help you brainstorm the hazards.
You shouldn't do a full design of the components, just get a feel for the major
ones.  For projects that involve hardware, the components will typically include
each individual piece of hardware.  If your software will have a database, or an
important library, these are also potential components.}

\section{Critical Assumptions}
The following are assumptions made about the software of the system. 
\begin{itemize}
    \item The data provided to this system is validated and correctly formatted
    before system ingestion
    \begin{itemize}
        \item Errors can occur from badly formatted or invalid incoming data,
        which is not something the system can control
    \end{itemize}
    \item The user of this application is not intentionally trying to misuse it
    \begin{itemize}
        \item This assumption mitigates the risk of someone intentionally
        damaging the system
    \end{itemize}
    \item Internet connection and server infrastructure will always be
    available, and will not suddenly go down and compromise the system
    \begin{itemize}
        \item This assumption mitigates the risk of a failing internet
        connection or fragile server infrastructure interrupting the system
    \end{itemize}
    \item Users using this application understand how to use the application,
    whether through documentation or a tutorial
    \begin{itemize}
        \item This assumptions ensures that user errors caused by lack of
        knowledge do not occur
    \end{itemize}
    \item The system is regularly maintained for security and bug fixes
    \begin{itemize}
        \item This assumption prevents any threats to the system due to poor
        maintenance
    \end{itemize}
    \item The system is scalable and possesses enough computational resources to
    handle any volume of data
    \begin{itemize}
        \item This assumption mitigates the risk of the system failing due to
        lack of resources
    \end{itemize}
\end{itemize}

\section{Failure Mode and Effect Analysis}

\wss{Include your FMEA table here. This is the most important part of this document.}
\wss{The safety requirements in the table do not have to have the prefix SR.
The most important thing is to show traceability to your SRS. You might trace to
requirements you have already written, or you might need to add new
requirements.}
\wss{If no safety requirement can be devised, other mitigation strategies can be
entered in the table, including strategies involving providing additional
documentation, and/or test cases.}

\section{Safety and Security Requirements}

\wss{Newly discovered requirements.  These should also be added to the SRS.  (A
rationale design process how and why to fake it.)}

\section{Roadmap}

\wss{Which safety requirements will be implemented as part of the capstone timeline?
Which requirements will be implemented in the future?}

\newpage{}

\section*{Appendix --- Reflection}

\wss{Not required for CAS 741}

\begin{enumerate}
    \item What went well while writing this deliverable? \newline \newline
        When writing this deliverable, it was simple in a sense where many of
        the topics and ideas presented here were talked about before hand with
        both the team and the supervisors. It was more of a matter of expanding
        on ideas and topics in further detail to ensure that the problem and
        goals were aligned with previous conversations. This allowed the writing
        process to be more time consuming than actually being difficult. 
    \item What pain points did you experience during this deliverable, and how
    did you resolve them? \newline \newline
        The biggest pain point during this deliverable was figuring out how to
        properly explain the problem and goals. Many if not all ideas presented
        were simply ideas. It was quite difficult to be able to not only explain
        the goals but to also connect it back to why they are important to have
        in this project. Displaying the understanding of how pieces fit together
        and explaining why certain things are in scope was the most challenging
        part of this deliverable. 
    \item How did you and your team adjust the scope of your goals to ensure
    they are suitable for a Capstone project (not overly ambitious but also of
    appropriate complexity for a senior design project)? \newline \newline
        As a team, this was very much a concern in the beginning. This resulted
        in our team setting up strong measures of communication. Before even
        starting on this deliverable we made sure to go through each section
        with the supervisor. We had discussions throughout the meeting where the
        team brings a more technical point of view to aid in the supervisor's
        needs and wants. It was through this process that we did not have to
        adjust the scope of the groups goals much as we started with the a
        strong foundation and understanding with all parties involved.  
\end{enumerate}  

\begin{enumerate}
    \item What went well while writing this deliverable? 
    \item What pain points did you experience during this deliverable, and how
    did you resolve them?
    \item Which of your listed risks had your team thought of before this
    deliverable, and which did you think of while doing this deliverable? For
    the latter ones (ones you thought of while doing the Hazard Analysis), how
    did they come about?
    \item Other than the risk of physical harm (some projects may not have any
    appreciable risks of this form), list at least 2 other types of risk in
    software products. Why are they important to consider?
\end{enumerate}

\end{document}
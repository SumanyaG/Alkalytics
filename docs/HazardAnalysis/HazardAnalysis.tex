\documentclass{article}

\usepackage{booktabs}
\usepackage{tabularx}
\usepackage{hyperref}
\usepackage{longtable}
\usepackage{geometry}  % For no margins
\usepackage{pdflscape}    
\hypersetup{
    colorlinks=true,       % false: boxed links; true: colored links
    linkcolor=red,          % color of internal links (change box color with linkbordercolor)
    citecolor=green,        % color of links to bibliography
    filecolor=magenta,      % color of file links
    urlcolor=cyan           % color of external links
}


\title{Hazard Analysis\\\progname}

\author{\authname}

\date{}

\input{../Comments}
%% Common Parts

\newcommand{\progname}{Software Engineering} % PUT YOUR PROGRAM NAME HERE
\newcommand{\authname}{Team 21, Alkalytics
\\ Sumanya Gulati - gulats10
\\ Kate Min - mink9
\\ Jennifer Ye - yej52
\\ Jason Tran - tranj78} % AUTHOR NAMES                  

\usepackage{hyperref}
    \hypersetup{colorlinks=true, linkcolor=blue, citecolor=blue, filecolor=blue,
                urlcolor=blue, unicode=false}
    \urlstyle{same}
                                


\begin{document}

\maketitle
\thispagestyle{empty}

~\newpage

\pagenumbering{roman}

\begin{table}[hp]
\caption{Revision History} \label{TblRevisionHistory}
\begin{tabularx}{\textwidth}{llX}
\toprule
\textbf{Date} & \textbf{Developer(s)} & \textbf{Change}\\
\midrule
25 October 2024 & Kate Min & Add sections 2 and 7\\
25 October 2024 & Jason Tran & Add sections 5 and 6\\
25 October 2024 & Jennifer Ye & Add sections 1 and 4\\
26 October 2024 & Sumanya Gulati & Add sections 3 and Appendix - Reflection\\
\bottomrule
\end{tabularx}
\end{table}

~\newpage

\tableofcontents

~\newpage

\pagenumbering{arabic}

\section{Introduction}

A hazard is any property or condition that has the potential to cause harm. This
document serves as a hazard analysis for the application revolving around the
capstone project ``Alkalytics". This project aims to aids in the data management
and analysis of an ocean alkalinity enhancement experiment. This document
identifies the components of the system, and then the possible software hazards
in these components, as well as ways to mitigate the risks they impose.

\section{Scope and Purpose of Hazard Analysis}

The purpose of the hazard analysis is to identify potential hazards and its
causes, assess the effects, and set mitigation strategies to eliminate or lessen
the risk of the hazard. It is important to consider all possible hazards
associated with the Alkalytics project and its components, as the following
losses could be incurred from an insufficient hazard assessment:
\begin{itemize}
    \item Data loss: The data gets unintentionally lost, deleted, or corrupted.
    \item Data integrity: The data is not complete, accurate, nor correct which
    can lead to computational errors.
    \item System availability: Users are unable to access the system due to
    factors such as unstable internet connection or server crashes.
    \item Security: If there are no proper measures for authentication,
    unauthorized access to the data or user information may occur.
\end{itemize}

\section{System Boundaries and Components}
The system boundaries are used to define what is within the scope of the hazard analysis,
in essence, where the system interacts with external systems, users and hardware. Some key
boundaries for Alkalytics include:
\begin{itemize}
  \item \textbf{User Interaction Boundaries}:
    \begin{enumerate}
      \item Interactions happen through the frontend interface, where users authenticate, 
      input data, and retrieve visualizations and analytics.
      \item Limit interactions in terms of user access levels, as only authenticated and authorized users
      are allowed to view or modify certain data.
    \end{enumerate}
  \item \textbf{Data Boundaries}:
    \begin{enumerate}
      \item Alkalytics handles data import (CSV files), processing, analysis, and export, requiring clear 
      boundaries on data validity and format conformity.
      \item The system is also bounded by its data storage capabilities, including access controls, data integrity 
      safeguards, and secure handling of sensitive information.
    \end{enumerate}
  \item \textbf{Network and External Connectivity Boundaries}:
    \begin{enumerate}
      \item External connections, like internet-based server access and database queries, form a boundary where system 
      security, speed, and availability must be managed.
      \item Network performance, internet stability, and server downtime impact the accessibility and reliability of the system.
    \end{enumerate}
\end{itemize}

The system can be broken down into the following major components:
\begin{enumerate}
  \item \textbf{Authentication System}: Ensures that only authorized users can access the Alkalytics platform and limits actions 
  based on user permissions.
  \item \textbf{Comma-Separated Values (CSV) Data Migration Module}: Handles the import of data from CSV files into the database,
  including checks for data format, completeness, and correctness.
  \item \textbf{Data Visualization Module}: Generates graphs and visual representations of data, requiring accuracy in rendering and 
  efficient performance.
  \item \textbf{Query Functionality}: Processes user queries to retrieve data from the database, ensuring that correct data is returned promptly.
  \item \textbf{Data Export Module}: Allows users to export data in CSV for external use, requiring accuracy and completeness.
  \item \textbf{Backend Database}: Stores and retrieves all user and application data, requiring reliable connection management, data integrity, and sufficient storage.
  \item \textbf{Frontend Interface}: Provides users with access to Alkalytics' functionalities through a web-based UI, handling data input validation and responsiveness.
  \item \textbf{Error Tracking System}: Logs and categorizes system errors for troubleshooting and performance monitoring.
  \item \textbf{Machine Learning Analysis Module}: Processes data to provide predictive insights or data analyses, requiring well-trained models and accurate data handling.
\end{enumerate}

\section{Critical Assumptions}

The following are assumptions made about the software of the system. 
\begin{itemize}
    \item The data provided to this system is validated and correctly formatted
    before system ingestion
    \begin{itemize}
        \item Errors can occur from badly formatted or invalid incoming data,
        which is not something the system can control
    \end{itemize}
    \item The user of this application is not intentionally trying to misuse it
    \begin{itemize}
        \item This assumption mitigates the risk of someone intentionally
        damaging the system
    \end{itemize}
    \item Internet connection and server infrastructure will always be
    available, and will not suddenly go down and compromise the system
    \begin{itemize}
        \item This assumption mitigates the risk of a failing internet
        connection or fragile server infrastructure interrupting the system
    \end{itemize}
    \item Users using this application understand how to use the application,
    whether through documentation or a tutorial
    \begin{itemize}
        \item This assumptions ensures that user errors caused by lack of
        knowledge do not occur
    \end{itemize}
    \item The system is regularly maintained for security and bug fixes
    \begin{itemize}
        \item This assumption prevents any threats to the system due to poor
        maintenance
    \end{itemize}
    \item The system is scalable and possesses enough computational resources to
    handle any volume of data
    \begin{itemize}
        \item This assumption mitigates the risk of the system failing due to
        lack of resources
    \end{itemize}
\end{itemize}

\section{Failure Mode and Effect Analysis}
\newgeometry{margin=2cm}
\begin{landscape}
    \begin{longtable}{|p{2.5cm}|p{4cm}|p{4cm}|p{4cm}|p{4cm}|p{1.25cm}|p{1cm}|}
        \hline
        \textbf{Component} & \textbf{Failure Modes} & \textbf{Effects of
        Failure} & \textbf{Causes of Failure} & \textbf{Recommended Action} &
        \textbf{SR} & \textbf{Ref} \\
        \hline
        \endfirsthead
        
        \hline
        \textbf{Component} & \textbf{Failure Modes} & \textbf{Effects of
        Failure} & \textbf{Causes of Failure} & \textbf{Recommended Action} &
        \textbf{SR} & \textbf{Ref} \\
        \hline
        \endhead
        
        \hline
        \endfoot
        Authentication & Unauthorized user access & Data breach, loss of data
        integrity & Weak security protocols & Strengthen authentication
        mechanisms & SR-1, SR-2 & H1-1 \\
        \cline{2-7}
        \texttt{} & User unable to log in & Loss of productivity & System
        downtime, credential errors & Ensure high system uptime and credential
        recovery & FR-14, SR-1, SR-4 & H1-2 \\
        \cline{1-7}
        CSV Data Migration & Data not uploaded to the database & Loss of data
        availability & Incorrect file format or server issue & Validate file
        format and ensure server uptime & FR-3, FR-4, SR-9 & H2-1 \\
        \cline{2-7}
        \texttt{} & Data partially uploaded & Incomplete data leads to incorrect
        analysis & Timeout during upload, corrupted file & Implement
        error-checking during upload & FR-3, FR-2 & H2-2 \\
        \cline{2-7}
        \texttt{} & Duplicate data entries & Conflicting results in data
        analysis & No validation for duplicates & Add duplicate data detection
        and rejection logic & SR-5, FR-4, SR-7 & H2-3 \\
        \cline{1-7}
        Data Visualization & Incorrect graph rendering & Misleading or
        inaccurate data interpretation & Inaccurate parameter selection &
        Improve input validation, real-time graph updates & FR-8, FR-9 & H3-1 \\
        \cline{2-7}
        \texttt{} & Slow graph rendering & Poor user experience & Large dataset
        or inefficient plotting algorithm & Optimize graph rendering speed &
        PR-5, PR-2 & H3-2 \\
        \cline{1-7}
        Query Functionality & Data not returned or delayed & User frustration,
        inability to retrieve data & Database connection or query input error &
        Optimize query performance and error handling & FR-5, FR-6, PR-2 & H4-1
        \\
        \cline{2-7}
        \texttt{} & Incorrect data returned & Incorrect conclusions from user &
        Misconfigured query logic & Validate query structure and results & PR-6,
        FR-5 & H4-2 \\
        \cline{2-7}
        \texttt{} & Query results outdated & Decisions based on stale data &
        Data not refreshed in database & Implement data refresh strategy &
        PR-14, FR-6 & H4-3 \\
        \cline{1-7}
        Data Export & CSV export generates corrupted files & Data cannot be used
        in external systems & Incorrect formatting logic & Implement robust
        export validation & FR-15, PR-6, SR-3 & H5-1 \\
        \cline{2-7}
        \texttt{} & Export missing data & Partial data exported, incomplete
        reports & Timeout during export or data truncation & Ensure export
        process handles large data volumes & FR-15, PR-8 & H5-2 \\
        \cline{2-7}
        \texttt{} & Session timeout too short & User repeatedly logged out,
        inconvenience & Incorrect session configuration & Increase session
        timeout settings & FR-14, SR-16, SR-4 & H5-3 \\
        \cline{1-7}
        Backend Database & Database connection lost & Data retrieval fails,
        analysis halted & Network issues, server downtime & Improve database
        fault-tolerance and backups & PR-9, PR-10 & H6-1 \\
        \cline{2-7}
        \texttt{} & Data corruption during storage & Loss of data integrity &
        Incorrect write operations or hardware failure & Implement database
        checksums and backups & SR-6, SR-7 & H6-2 \\
        \cline{2-7}
        \texttt{} & Insufficient storage space & Application crashes or stops
        accepting data & Lack of storage planning & Increase storage capacity
        and monitor usage & PR-12, PR-13 & H6-3 \\
        \cline{1-7}
        Frontend Interface & UI unresponsive & Poor user experience, tasks
        uncompleted & JavaScript errors, resource overload, slow internet &
        Debug UI code, efficient resource management & FR-7, PR-3 & H7-1 \\
        \cline{2-7}
        \texttt{} & Elements not displayed correctly & Confusion, incorrect
        actions performed & Browser compatibility issues & Test for
        compatibility across browsers & LFR-1, OER-3 & H7-2 \\
        \cline{2-7}
        \texttt{} & Data input fields allow invalid entries & Corrupted data
        entered into system & No input validation & Enforce input validation on
        frontend & FR-1, FR-4, SR-7 & H7-3 \\
        \cline{1-7}
        Performance & Slow page load times & User frustration, reduced
        productivity & Unoptimized frontend/backend code & Optimize code and
        database queries & PR-3, PR-5 & H8-1 \\
        \cline{2-7}
        \texttt{} & High CPU/memory usage & System instability, crashes &
        Inefficient data processing or memory leaks & Implement memory
        management and resource monitoring & PR-4, PR-2 & H8-2 \\
        \cline{1-7}
        Error Tracking & Errors not logged & Difficulty identifying and
        resolving issues & Lack of error handling in code & Implement detailed
        error logging & FR-12, FR-13 & H9-1 \\
        \cline{2-7}
        \texttt{} & Logged errors not displayed to user & Users unaware of
        issues & Incomplete error-handling UI & Display errors to user,
        troubleshooting tips & MSR-5, PR-9 & H9-2 \\
        \cline{2-7}
        \texttt{} & Errors logged but not categorized & Troubleshooting becomes
        complex & Poor error categorization & Create detailed error categories
        and log structure & MSR-5, FR-12 & H9-3 \\
        \cline{1-7}
        Machine Learning Analysis & Incorrect data prediction & Misleading
        trends, faulty decisions & Inaccurate algorithm configuration & Improve
        machine learning validation steps & FR-11 & H10-1 \\
        \cline{2-7}
        \texttt{} & Model training incomplete & Model unable to provide accurate
        predictions & Insufficient data or faulty training process & Ensure
        ample, clean training data & FR-11, PR-8 & H10-2 \\
        \cline{2-7}
        \texttt{} & Training data overfitting & Model provides unreliable
        results & Model too tightly fitted to training data & Implement
        regularization techniques & FR-11 & H10-3 \\
        
    \end{longtable}
\end{landscape}

\restoregeometry

\section{Safety and Security Requirements}
New safety and security requirements have been discovered and will be integrated
within the SRS document. Note that the entire requirement codes have been
changed with the new additions having expanded information, and the previous
requirements displaying new codes.

\subsection{Access Requirements}
\textbf{SR-1.} Access to the application must be restricted to authorized
  personnel, with an authentication mechanism. \ \\
\ \\
\textbf{SR-2.} Only authenticated users should have the ability to query or
  modify the data, and each user’s access must be limited to their capabilities
  within the application. \ \\
  \ \\
\textbf{SR-3.} The application must restrict sensitive operations (e.g., data export)
  to authorized personnel only.
    \begin{itemize}
      \item \textit{Rationale:} Prevents unauthorized users from exporting or
      sharing sensitive data, protecting data integrity.
      \item \textit{Fit Criterion:} Only authorized users must be able to
      perform sensitive operations like data export.
      \item \textit{Traceability:} FR-15, SR-2.
    \end{itemize}
\ \\
\textbf{SR-4.} The application must enforce session timeout and automatic logouts
after a period of inactivity.
  \begin{itemize}
    \item \textit{Rationale:} Protects the application from unauthorized access if
    users leave their session unattended.
    \item \textit{Fit Criterion:} Sessions must time out and log users off
    automatically after a specified inactivity period.
    \item \textit{Traceability:} FR-14, SR-1, SR-4.
  \end{itemize}

\subsection{Integrity Requirements}
\textbf{SR-5.} The application must validate data inputs to ensure they conform
  to expected formats and values before they are processed. \ \\
\ \\
\textbf{SR-6.} The application must not modify the data unnecessarily through
  its transfer process. \ \\
\ \\
\textbf{SR-7.} The application must ensure that any data processed or
  transferred is free from duplication or inconsistencies. \ \\
\ \\
\textbf{SR-8.} The application must have safeguards in place to maintain the
  accuracy of the transferred data. \ \\
  \ \\
\textbf{SR-9.} The application must validate CSV data thoroughly before upload to
prevent corrupted or incomplete data entries.
  \begin{itemize}
    \item \textit{Rationale:} Ensures that only valid, complete, and accurate
    data enters the application to prevent faulty analysis.
    \item \textit{Fit Criterion:} The application shall reject any data that
    does not meet the validation criteria.
    \item \textit{Traceability:} FR-3, FR-4.
  \end{itemize}

\subsection{Privacy Requirements}
\textbf{SR-10.} All personal information related to experimental participants or
  stakeholders, if applicable, must be anonymized and handled in accordance with
  relevant privacy laws and regulations. \ \\
\ \\
\textbf{SR-11.} The application must restrict data sharing with external parties
  unless expressly authorized by stakeholders, and users must be fully informed
  about the privacy policies. \ \\
  \ \\
  \textbf{SR-12.} The application must monitor database storage capacity and alert
  administrators when thresholds are reached to prevent application crashes.
    \begin{itemize}
      \item \textit{Rationale:} Ensures the application continues operating smoothly
      by addressing storage limits proactively.
      \item \textit{Fit Criterion:} The application shall send alerts when storage
      capacity exceeds 80\% usage.
      \item \textit{Traceability:} PR-12, PR-13.
    \end{itemize}

\subsection{Audit Requirements}
\textbf{SR-13.} The application must maintain a comprehensive audit trail,
  logging all access and modification events, including timestamps and
  identities of users performing actions. \ \\
\ \\
\textbf{SR-14.} Audit logs must be securely stored and accessible only by
  authorized personnel. \ \\
  \ \\
  \textbf{SR-15.} The application must display real-time 
  error logs to users to enhance troubleshooting when applicable.
    \begin{itemize}
      \item \textit{Rationale:} Ensures users are informed about application issues
      and can take corrective action promptly.
      \item \textit{Fit Criterion:} All errors must be logged and displayed
      clearly to users in real-time.
      \item \textit{Traceability:} FR-12, MSR-5.
    \end{itemize}
  

\subsection{Immunity Requirements}
\textbf{SR-16.} The application must have proactive measures to detect and
  mitigate suspicious activities, such as repeated unauthorized access attempts,
  ensuring the application remains secure at all times. \ \\
  \ \\
  \textbf{SR-17.} Real-time monitoring and optimization of application resources must
  be implemented to avoid crashes due to resource overload.
    \begin{itemize}
      \item \textit{Rationale:} Prevents application downtime by ensuring efficient
      use of CPU and memory.
      \item \textit{Fit Criterion:} The application must manage memory and CPU usage
      dynamically to avoid overloads.
      \item \textit{Traceability:} PR-4, PR-9.
    \end{itemize}

\section{Roadmap}
The following table outlines a proposed roadmap of when each safety requirement
will be implemented within the capstone timeline and justifications.

\begin{longtable}[c]{|m{2cm}|m{2.6cm}|m{2cm}|m{3.8cm}|}
    \hline
    \textbf{Stage} & \textbf{Req. Category} & \textbf{Req. ID(s)} &
    \textbf{Rationale} \\
    \hline
    \endhead
    PoC Demo (Nov 11) & Access & N/A & The PoC plan will not consider user
    access features at this time.\\
    \cline{2-4}
    & Integrity & SR-5, SR-6, SR-7, SR-8, SR-9 & The database must adhere to
    these requirements for a successful PoC.\\
    \cline{2-4}
    & Privacy & N/A & Since the PoC plan will\\
    & Audit & N/A & only have the database\\
    & Immunity & N/A & these requirements are not applicable.\\
    \hline
    Rev0 Demo (Feb 3) & Access & SR-1 & User authentication should be
    implemented during front-end development.\\
    \cline{2-4}
    & Integrity & N/A & The crucial integrity requirements will have already
    been implemented by the PoC demo.\\
    \cline{2-4}
    & Privacy & SR-10, SR-11 & At this point there will be user access, thus
    these requirements must be implemented.\\
    \cline{2-4}
    & Audit & N/A & These requirements\\
    & Immunity & N/A & are not high-priority.\\
    \hline
    Rev1 Final Demo (Mar 24) & Access & SR-2, SR-3 & User access must be
    extended to permissions and capabilities prior to release.\\
    \cline{2-4}
    & Integrity & N/A & The crucial integrity requirements will have already
    been implemented by the PoC demo.\\
    \cline{2-4}
    & Privacy & SR-12 & This requirement is necessary for system availability
    and robustness to extend past capstone.\\
    \cline{2-4}
    & Audit & SR-15 & Client and users must be informed about system issues and
    be able to troubleshoot even without the original development team.\\
    \cline{2-4}
    & Immunity & SR-17 & This requirement is necessary for system availability
    and robustness to extend past capstone.\\
    \hline
    Future considerations & Access & SR-4 & This requirement is out of scope for
    the project timeline.\\
    \cline{2-4}
    & Integrity & SR-11 & This requirement is out of scope for the project
    timeline.\\
    \cline{2-4}
    & Privacy & N/A & All privacy requirements have been covered.\\
    \cline{2-4}
    & Audit & SR-13, SR-14 & These requirements would be nice-to-haves but are
    not essential for the project.\\
    \cline{2-4}
    & Immunity & SR-16 & This requirement is out of scope for the project
    timeline.\\
    \hline
    \caption{Roadmap of the implementation of the safety and security requirements.}
\end{longtable}

\newpage{}

\section*{Appendix --- Reflection}

\begin{enumerate}
    \item \textbf{What went well while writing this deliverable?}
    
    Writing a comprehensive Software Requirements Specification (SRS) right before this milestone
    helped tremendously because as a team, we had been nudged into considering multiple aspects
    of the project we had not thought of earlier. All this research and decision-making ensured
    that coming into this milestone, we had most of the relevant things figured out prior to writing
    this documentation.
    \newline
    \item \textbf{What pain points did you experience during this deliverable, and how
    did you resolve them?}
    
    One major pain point we experienced was the lack of time we could dedicate to this milestone as
    a team. Since this milestone was due the week after reading week when most of us had 2 or more than 2 midterms along
    with multiple other assignment deadlines during the same week, it was harder to coordinate times to meet up and get a lot of the work done
    a couple of days before the deadline, which is what we ideally like to do. As an instance, due to the time constraints, 
    we did not have enough time to get our Pull Requests (PRs) reviewed and approved by all the team members before merging.\\
    \newline
    Additionally, since most of the sections are a lot more interwoven as compared to prior milestones (which although connected,
    could be written somewhat independently), it was harder to divide the sections. This ties back to the aforementioned 
    time-contraints and we believe that if we had more time to meet up as a team, the division of tasks would have been easier
    to manage.  We resolved this by effectively communicating with each other over text instead.
    \newline
    \item \textbf{Which of your listed risks had your team thought of before this
    deliverable, and which did you think of while doing this deliverable? For
    the latter ones (ones you thought of while doing the Hazard Analysis), how
    did they come about?}

    Our team had thought of a majority of the risks before this deliverable. The ones that we had not
    considered include unresponsiveness of the frontend, performance related failures and error tracking related hazards.
    These came about after we had identified all the system boundaries and when we focused on the frontend, performance as well as error tracking 
    components in particular and tried to consider all the potential ways these component can fail.
    \newline
    \item \textbf{Other than the risk of physical harm (some projects may not have any
    appreciable risks of this form), list at least 2 other types of risk in
    software products. Why are they important to consider?}

    Beyond the risk of physical harm, software products may have the following types of risks:
    \begin{enumerate}
      \item \textbf{Security Risks}: These involve vulnerabilities that can be exploited, leading to unauthorized access, data breaches
      or malicious attacks. For projects that are data centric like ours, these risks are critical because they can lead to the loss of
      sensitive information.
      \item \textbf{Operational Risks}: These risks arise from failures in the software's functionality, reliability, or performance under 
      real-world conditions. Operational risks are important to consider because any failure in the software’s expected behavior can disrupt 
      business operations, cause customer dissatisfaction, lead to unexpected maintenance costs, or more.
    \end{enumerate}
\end{enumerate}

\end{document}
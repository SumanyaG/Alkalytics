% THIS DOCUMENT IS FOLLOWS THE VOLERE TEMPLATE BY Suzanne Robertson and James Robertson
% ONLY THE SECTION HEADINGS ARE PROVIDED
%
% Initial draft from https://github.com/Dieblich/volere
%
% Risks are removed because they are covered by the Hazard Analysis
\documentclass[12pt]{article}

\usepackage{booktabs}
\usepackage{tabularx}
\usepackage{hyperref}
\hypersetup{
    bookmarks=true,         % show bookmarks bar?
      colorlinks=true,      % false: boxed links; true: colored links
    linkcolor=red,          % color of internal links (change box color with linkbordercolor)
    citecolor=green,        % color of links to bibliography
    filecolor=magenta,      % color of file links
    urlcolor=cyan           % color of external links
}

\newcommand{\lips}{\textit{Insert your content here.}}

%% Comments

\usepackage{color}

\newif\ifcomments\commentstrue %displays comments
%\newif\ifcomments\commentsfalse %so that comments do not display

\ifcomments
\newcommand{\authornote}[3]{\textcolor{#1}{[#3 ---#2]}}
\newcommand{\todo}[1]{\textcolor{red}{[TODO: #1]}}
\else
\newcommand{\authornote}[3]{}
\newcommand{\todo}[1]{}
\fi

\newcommand{\wss}[1]{\authornote{blue}{SS}{#1}} 
\newcommand{\plt}[1]{\authornote{magenta}{TPLT}{#1}} %For explanation of the template
\newcommand{\an}[1]{\authornote{cyan}{Author}{#1}}

%% Common Parts

\newcommand{\progname}{Software Engineering} % PUT YOUR PROGRAM NAME HERE
\newcommand{\authname}{Team 21, Alkalytics
\\ Sumanya Gulati - gulats10
\\ Kate Min - mink9
\\ Jennifer Ye - yej52
\\ Jason Tran - tranj78} % AUTHOR NAMES                  

\usepackage{hyperref}
    \hypersetup{colorlinks=true, linkcolor=blue, citecolor=blue, filecolor=blue,
                urlcolor=blue, unicode=false}
    \urlstyle{same}
                                


\begin{document}

\title{Software Requirements Specification for \progname: subtitle describing software} 
\author{\authname}
\date{\today}
	
\maketitle

~\newpage

\pagenumbering{roman}

\tableofcontents

~\newpage

\section*{Revision History}

\begin{tabularx}{\textwidth}{p{3cm}p{2cm}X}
\toprule {\textbf{Date}} & {\textbf{Version}} & {\textbf{Notes}}\\
\midrule
09-28-24 & 1.0 & Rough draft of sections 1,2,10,24\\
Date 2 & 1.1 & Notes\\
\bottomrule
\end{tabularx}

~\\

~\newpage
\section{Purpose of the Project}
\subsection{User Business}
\lips
\subsection{Goals of the Project}
This project aims to provide a unique solution to a data management and analysis
problem. The project this solution is dedicated to supporting is an ocean
alkalinity enhancement experiment process. \newline
This platform will be able to consolidate and organize the data from the
experiments with proper labeling acorss all given data sets allowing for a
centeralize method of data storage that is both scalable and maintanable. On the
platform users can request to see certain data points from anywhere from the
inputted data sets given any specified order. Once the data is returned to the
user the platform can show inter-parameter comparability to better aid data
analysis. This comparability acts as a starting point to analysis. This will all
be presented in a web interface where all the user functions will be displayed
and can be shared among those involved in the experiemnt.   
\section{Stakeholders}
Anyone that has interest in the system 
end user 
indirect user
client 
different users have different requirments, different viewpoints 
\subsection{Client}
Dr. Charles De Lannoy serves as the main client for this project as he is the
lead supervisor of the reseach study. Bassel Abdelkader is another client of
this project as he is the person that works directly with the research data. One
of his responsibilities is to record the resperiment data and upload them to
their current data storage system, Micorsoft Excel. 
\subsection{Customer}
Although this project is a taliored solution to one research study, its
application can be extended to any other situation where large sets of data is involved. 
This could be shared among other reseachers to aid in their data managment and analysis as well. 
\subsection{Other Stakeholders}
Current students and members of the lab working on the study is can also be
considered stakeholders for the same reasons as the clients. However, since they
will only be working with the study for a short amount of time without daily or
consistant itneraction, they do not serve as a main stakeholder. The founder of
the study, who is currently funding the research project is another stakeholder.
However, since they do not work directly with the processes of the study rather
oversee the process, they may not have strong interest in the details of the solution. 
\subsection{Hands-On Users of the Project}
Dr. Charles De Lannon, Bassel Abdelkader and the students involved in the
experiemnt process are all hands-on users of the project. They will be the ones
who will be interacting directly with the project. 
\subsection{Personas}
\begin{itemize}
  \item John Doe is an 23 year old McMaster undergraduate student who has a
  research position on the ocean alkalinity research project. They have been
  tasksed to aid the experiment data collection proces. After being told that
  the data is being stored in a master Excel file, they find that is it hard to
  use. Being an engineering student without much experience with Excel, they
  struggle to find the data they want. Inputting data is still a managble
  process but they find themselves to be spending a lot of time looking at Excel
  documentation which they find frustrating as that time could be allocated to
  being more productive during the school term. Although, they want to a better
  way to manage the data, they know that it is not up to their decision on what
  tools are being used but suggested that there could be another better solution
  to use. 
  \item Dr.\ Carly Kelvon is a 60 years old professor at a university and
  is working on her own research project for over five years. She has gathered
  lots of data and thankfully she has always been great at Excel. However, other
  the last two years she has found that Excel is becoming less sustainable. The
  queries are a lot slower and sifting through pages and pages of data is
  wasiting a lot of her time. She sees this more evidently through those that
  work along side her as they are also facing the same struggles with even less
  Excel experience as her. She wants to find a more scalable solution but she
  fears that her lack of digital knowledge will do her more harm than good, as a
  result she fears that if she introduces a new platform to serve her needs
  better that she will find it hard and confusing to use. 
  \item Dr.\ Alex Stark is a 30 year old associate professor who has recently
  gotten funding for his innovative research idea and has been dedicating all
  his time on prefecting its methodology. It has only been one year since his
  research started but had recently found a great application of his ideas to
  reach far more people and be more impactful that he had originally thought.
  But with his current data managment set up, he quickly realises that it is not
  sustainable. He finds that there are many other solutions on the market but
  they do not exactly meet his needs and cost a lot more than what he can spend
  on a tool. He decided that the best way is to create his own tool but lacks
  the software knowledge to create something stable and realiable. 
\end{itemize}
    
\subsection{Priorities Assigned to Users}
\lips
\subsection{User Participation}
\lips
\subsection{Maintenance Users and Service Technicians}
\lips

\section{Mandated Constraints}
\subsection{Solution Constraints}
\lips
\subsection{Implementation Environment of the Current System}
\lips
\subsection{Partner or Collaborative Applications}
\lips
\subsection{Off-the-Shelf Software}
\lips
\subsection{Anticipated Workplace Environment}
\lips
\subsection{Schedule Constraints}
\lips
\subsection{Budget Constraints}
\lips
\subsection{Enterprise Constraints}
\lips

\section{Naming Conventions and Terminology}
\subsection{Glossary of All Terms, Including Acronyms, Used by Stakeholders
involved in the Project}
\lips

\section{Relevant Facts And Assumptions}
\subsection{Relevant Facts}
\lips
\subsection{Business Rules}
\lips
\subsection{Assumptions}
\lips

\section{The Scope of the Work}
\subsection{The Current Situation}
\lips
\subsection{The Context of the Work}
\lips
\subsection{Work Partitioning}
\lips
\subsection{Specifying a Business Use Case (BUC)}
\lips

\section{Business Data Model and Data Dictionary}
\subsection{Business Data Model}
\lips
\subsection{Data Dictionary}
\lips

\section{The Scope of the Product}
\subsection{Product Boundary}
\lips
\subsection{Product Use Case Table}
\lips
\subsection{Individual Product Use Cases (PUC's)}
\lips

\section{Functional Requirements}
\subsection{Functional Requirements}
\lips

\section{Look and Feel Requirements}
\subsection{Appearance Requirements}
\begin{itemize}
  \item [AR1.]The website should have a simple and organized layout, with clearly defined
  sections where all major functions should be easily accessible and viewable
  \item [AR2.]The website shall be responsive on all computer and laptop screens aside from mobile screens
  \item [AR3.] The website's functions and buttons shall be properly labeled
  where no button is ambiguous to users
  \item [AR4.]The produced plot from the data shall be properly labeled
\end{itemize}

\subsection{Style Requirements}
\begin{itemize}
  \item [SR1.] All icons on the website must be in the artistic style
  \item [SR2.] All colours must match the theme of the website
  \item [SR3.] All fonts are to be consistant throughout the website 

\end{itemize}

\section{Usability and Humanity Requirements}
\subsection{Ease of Use Requirements}
\lips
\subsection{Personalization and Internationalization Requirements}
\lips
\subsection{Learning Requirements}
\lips
\subsection{Understandability and Politeness Requirements}
\lips
\subsection{Accessibility Requirements}
\lips

\section{Performance Requirements}
\subsection{Speed and Latency Requirements}
\lips
\subsection{Safety-Critical Requirements}
\lips
\subsection{Precision or Accuracy Requirements}
\lips
\subsection{Robustness or Fault-Tolerance Requirements}
\lips
\subsection{Capacity Requirements}
\lips
\subsection{Scalability or Extensibility Requirements}
\lips
\subsection{Longevity Requirements}
\lips

\section{Operational and Environmental Requirements}
\subsection{Expected Physical Environment}
\lips
\subsection{Wider Environment Requirements}
\lips
\subsection{Requirements for Interfacing with Adjacent Systems}
\lips
\subsection{Productization Requirements}
\lips
\subsection{Release Requirements}
\lips

\section{Maintainability and Support Requirements}
\subsection{Maintenance Requirements}
\lips
\subsection{Supportability Requirements}
\lips
\subsection{Adaptability Requirements}
\lips

\section{Security Requirements}
\subsection{Access Requirements}
\lips
\subsection{Integrity Requirements}
\lips
\subsection{Privacy Requirements}
\lips
\subsection{Audit Requirements}
\lips
\subsection{Immunity Requirements}
\lips

\section{Cultural Requirements}
\subsection{Cultural Requirements}
\lips

\section{Compliance Requirements}
\subsection{Legal Requirements}
\lips
\subsection{Standards Compliance Requirements}
\lips

\section{Open Issues}
\lips

\section{Off-the-Shelf Solutions}
\subsection{Ready-Made Products}
\lips
\subsection{Reusable Components}
\lips
\subsection{Products That Can Be Copied}
\lips

\section{New Problems}
\subsection{Effects on the Current Environment}
\lips
\subsection{Effects on the Installed Systems}
\lips
\subsection{Potential User Problems}
\lips
\subsection{Limitations in the Anticipated Implementation Environment That May
Inhibit the New Product}
\lips
\subsection{Follow-Up Problems}
\lips

\section{Tasks}
\subsection{Project Planning}
\lips
\subsection{Planning of the Development Phases}
\lips

\section{Migration to the New Product}
\subsection{Requirements for Migration to the New Product}
\lips
\subsection{Data That Has to be Modified or Translated for the New System}
\lips

\section{Costs}
\lips
\section{User Documentation and Training}
\subsection{User Documentation Requirements}
User documentation will cover both the front-end and back-end features and functionalities.
\subsubsection{Back-End}
The backend user documentaion will include all the API end points that are created. \newline
It wil show the base URL for the end points along with descriptions of each
along with its required parameters and its types. These parameter will be those
in the path, query and body. The response from the API call will inlcude what
the expected body will be along with any error handeling. With this being a
small project, a rate limit section may be included to ensure that it does not
exceed the number of API calls. Versioning and support resources will be added
for any additonal references missing. \newline

\subsubsection{Front-End}
The front end documentaion serves less as for developers but for the end user.
This documentation will include how to add new attributes to the data set, how
to request data plots and any major functionality features. 

\subsection{Training Requirements}
No training is required
\section{Waiting Room}
\lips

\section{Ideas for Solution}
\lips

\newpage{}
\section*{Appendix --- Reflection}

The information in this section will be used to evaluate the team members on the
graduate attribute of Lifelong Learning.  Please answer the following questions:

\begin{enumerate}
  \item What knowledge and skills will the team collectively need to acquire to
  successfully complete this capstone project?  Examples of possible knowledge
  to acquire include domain specific knowledge from the domain of your
  application, or software engineering knowledge, mechatronics knowledge or
  computer science knowledge.  Skills may be related to technology, or writing,
  or presentation, or team management, etc.  You should look to identify at
  least one item for each team member.
  \item For each of the knowledge areas and skills identified in the previous
  question, what are at least two approaches to acquiring the knowledge or
  mastering the skill?  Of the identified approaches, which will each team
  member pursue, and why did they make this choice?
\end{enumerate}

\end{document}
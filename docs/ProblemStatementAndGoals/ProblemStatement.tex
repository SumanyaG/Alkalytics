\documentclass{article}

\usepackage{tabularx}
\usepackage{booktabs}

\title{Problem Statement and Goals\\\progname}

\author{\authname}

\date{}

%% Comments

\usepackage{color}

\newif\ifcomments\commentstrue %displays comments
%\newif\ifcomments\commentsfalse %so that comments do not display

\ifcomments
\newcommand{\authornote}[3]{\textcolor{#1}{[#3 ---#2]}}
\newcommand{\todo}[1]{\textcolor{red}{[TODO: #1]}}
\else
\newcommand{\authornote}[3]{}
\newcommand{\todo}[1]{}
\fi

\newcommand{\wss}[1]{\authornote{blue}{SS}{#1}} 
\newcommand{\plt}[1]{\authornote{magenta}{TPLT}{#1}} %For explanation of the template
\newcommand{\an}[1]{\authornote{cyan}{Author}{#1}}

%% Common Parts

\newcommand{\progname}{Software Engineering} % PUT YOUR PROGRAM NAME HERE
\newcommand{\authname}{Team 21, Alkalytics
\\ Sumanya Gulati - gulats10
\\ Kate Min - mink9
\\ Jennifer Ye - yej52
\\ Jason Tran - tranj78} % AUTHOR NAMES                  

\usepackage{hyperref}
    \hypersetup{colorlinks=true, linkcolor=blue, citecolor=blue, filecolor=blue,
                urlcolor=blue, unicode=false}
    \urlstyle{same}
                                


\begin{document}

\maketitle

\begin{table}[hp]
\caption{Revision History} \label{TblRevisionHistory}
\begin{tabularx}{\textwidth}{llX}
\toprule
\textbf{Date} & \textbf{Developer(s)} & \textbf{Change}\\
\midrule
09-19-2024 & Jennifer Ye & Inital Rough Work \\
Date2 & Name(s) & Description of changes\\
... & ... & ...\\
\bottomrule
\end{tabularx}
\end{table}

\section{Problem Statement}

\wss{You should check your problem statement with the
\href{https://github.com/smiths/capTemplate/blob/main/docs/Checklists/ProbState-Checklist.pdf}
{problem statement checklist}.} 

\wss{You can change the section headings, as long as you include the required
information.}

\subsection{Problem}

\subsection{Inputs and Outputs}

\wss{Characterize the problem in terms of ``high level'' inputs and outputs.  
Use abstraction so that you can avoid details.}

\subsection{Stakeholders}

\subsection{Environment}

\wss{Hardware and software environment}

\section{Goals}
This project has a total of seven goals.
\subsection*{Consolidated Organized Accurate Data with proper labels}
The raw data input for this project is a csv file exported from Excel. With hundreds of experiments,
where each has ten or more attributes will account for thousands of data points. As the study gets larger, the current data collection format is not sustainable.
As such it is a goal to be able to migrate all the existing data to another solution where the solution is not only scalable but also extendable.
To migrate all the data also entails that the data is properly labelled as it had been originally.


\subsection*{Querying from Multiple Sets}
The experiment has multiple pages of Exec sheets that data is recorded on. Although the data is separated and organised across sheets, to be able to analyse the data queries will need to be made.
This project should be able to pull data across sheets quickly and efficiently. Currently, retrieving the data is manual, slow and tedious.
This project strives to make data retrieval easy and efficient.  


\subsection*{Inter-Parameter Comparability}
To further aid the data analysis of the experiment data, the product should be able to compare multiple parameters.
This could be a simple binary comparison of two attributes or more, but also narrowing down the search windows.
For example, the user could specify the time frame of when the experiments were done. Within that time frame, they might want to compare three different attributes from those experiences.
One of the goals for this project is to be able to make the above example possible.


\subsection*{Web Interface}
With a web interface, the user can easily interact with the backend processes.
This interface should be user friendly and displays all the functions that are needed to input data, and return the appropriate data.
The returned data should be given in a readable way.


\subsection*{Flexibility in Expansion (i.e., adding additional parameters)}
Although a big portion of this project is to migrate the existing data into a better solution, the product should also be
built with the possibility of expansion. As the experiment grows, there could be a new parameter to be recorded.
There could also be the possibility for this tool to be used in other experiments that do not follow the same format. The product should still be able to handle those new changes, among others.
It is our goal to be able to create a tool to be used as a general data recording tool.


\subsection*{Visual Plot of Specified Parameters + Results}
The product should be able to use the data it retrieves to interpret it in a visual manner.
This product must be able to plot the data in the desired format of the user to then be exported.
As the data will not be completely useful if it is not readable or understandable to the user.

\section{Stretch Goals}
\subsection*{Automatically Download Data from Data Collection Device}
The current state of the experiment requires a data collection device. This device only holds a certain number of experiment trials.
This data must also be manually exported at the end of each day. If the device is powered off or hits the limit of total possible recorded experiments, all the data will be lost.
As a stretch goal, the product could be connected to this device and upload the data either when it finishes collecting data per experiment or once it hits a limit.
This is a stretch goal as it requires the team to go through how the device actually records the data.


\subsection*{Dynamic Dashboard to Generate Comparison Reports}
As an extension of one of the main project goals, the visual plot can be in the form of a dashboard with a small written conclusion of the data requested instead of just the plotted graph.
The choice of visual plot can also be given to the user. The dashboard could change to dynamically to newly added information, user interactions and removals
without having to regenerate the entire plot from scratch.  


\subsection*{Mobile Development and Accessibility}
This project is intended to have a website interface. As a stretch goal, the project could be extended to a mobile app for
easy access and reachability. The website and mobile app could also incorporate more accessibility features for those with disabilities.
This may be but not limited to colour contrasts, screen readers, big fonts and responsiveness to those who have trouble seeing.
 
\subsection*{Machine Learning Analysis and Projections}
One major stretch goal is to incorporate an artificial intelligence machine learning to data analysis.
In hopes to make data analysis easier for the user and more adaptable to any possible analysis.
By introducing an machine learning algorithm, it would be able to automatically query the data, customise that query,
analyse the data and create a customizable dashboard board as well based on any sort of prompt. The machine learning 
aspect can also automatically return conclusions based on prompts

\section{Challenge Level and Extras}

\wss{State your expected challenge level (advanced, general or basic).  The
challenge can come through the required domain knowledge, the implementation or
something else.  Usually the greater the novelty of a project the greater its
challenge level.  You should include your rationale for the selected level.
Approval of the level will be part of the discussion with the instructor for
approving the project.  The challenge level, with the approval (or request) of
the instructor, can be modified over the course of the term.}

\wss{Teams may wish to include extras as either potential bonus grades, or to
make up for a less advanced challenge level.  Potential extras include usability
testing, code walkthroughs, user documentation, formal proof, GenderMag
personas, Design Thinking, etc.  Normally the maximum number of extras will be
two.  Approval of the extras will be part of the discussion with the instructor
for approving the project.  The extras, with the approval (or request) of the
instructor, can be modified over the course of the term.}

\newpage{}

\section*{Appendix --- Reflection}

\wss{Not required for CAS 741}

The purpose of reflection questions is to give you a chance to assess your own
learning and that of your group as a whole, and to find ways to improve in the
future. Reflection is an important part of the learning process.  Reflection is
also an essential component of a successful software development process.  

Reflections are most interesting and useful when they're honest, even if the
stories they tell are imperfect. You will be marked based on your depth of
thought and analysis, and not based on the content of the reflections
themselves. Thus, for full marks we encourage you to answer openly and honestly
and to avoid simply writing ``what you think the evaluator wants to hear.''

Please answer the following questions.  Some questions can be answered on the
team level, but where appropriate, each team member should write their own
response:


\begin{enumerate}
    \item What went well while writing this deliverable? 
    \item What pain points did you experience during this deliverable, and how
    did you resolve them?
    \item How did you and your team adjust the scope of your goals to ensure
    they are suitable for a Capstone project (not overly ambitious but also of
    appropriate complexity for a senior design project)?
\end{enumerate}  

\end{document}
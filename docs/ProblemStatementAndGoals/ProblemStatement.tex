\documentclass{article}

\usepackage{tabularx}
\usepackage{booktabs}

\title{Problem Statement and Goals\\\progname}

\author{\authname}

\date{}

\input{../Comments}
%% Common Parts

\newcommand{\progname}{Software Engineering} % PUT YOUR PROGRAM NAME HERE
\newcommand{\authname}{Team 21, Alkalytics
\\ Sumanya Gulati - gulats10
\\ Kate Min - mink9
\\ Jennifer Ye - yej52
\\ Jason Tran - tranj78} % AUTHOR NAMES                  

\usepackage{hyperref}
    \hypersetup{colorlinks=true, linkcolor=blue, citecolor=blue, filecolor=blue,
                urlcolor=blue, unicode=false}
    \urlstyle{same}
                                


\begin{document}

\maketitle

\begin{table}[hp]
\caption{Revision History} \label{TblRevisionHistory}
\begin{tabularx}{\textwidth}{llX}
\toprule
\textbf{Date} & \textbf{Developer(s)} & \textbf{Change}\\
\midrule
09-19-2024 & Jennifer Ye & Inital Rough Work \\
09-23-2024 & Jennifer Ye & Final Edits After Team Meet\\
02-22-2025 & Jennifer Ye, Kate Min & Incorporated Feedback\\ 
\bottomrule
\end{tabularx}
\end{table}



\section{Problem Statement}
This project aims to aid in the data management and analysis of an ocean
alkalinity enhancement experiment process.

\subsection{Background}
This project aims to aid in the data management and analysis of an ocean
alkalinity enhancement experiment process. Rising global temperatures are
impacting the pH balance of the oceans, which in turn affects their natural
ability to absorb carbon dioxide (CO\textsubscript{2}) from the atmosphere. The
research is working towards a scalable process to capture CO\textsubscript{2}
emissions using Bipolar Membrane Electrodialysis (BMED) technology. This process
separates a salt solution into distinct acid and base components, with the base
solution meant to be returned to the ocean.

\noindent The goal of the experimental study is to increase the ocean's pH
levels to enhance its ability to absorb more CO\textsubscript{2}, helping to
reduce its warming effect and mitigate global temperature rise. Currently, the
production process for generating this base is on a small scale and is
still being refined. A significant number of experiments and data collection
would be required to scale it up globally, necessitating a more expansive
production operation. Optimization of the experimental data is critical to
improve process efficiency. This presents a big data software challenge,
requiring the ability to identify and fine-tune specific parameters to achieve
ideal conditions.

\subsection{Problem}
Currently, all experiment data is stored in separate Excel spreadsheets, with
one file for each day's worth of experiments. This method of data collection and
storage is becoming increasingly inefficient and unsustainable as the study
grows. The research team is forced to manually sift through the spreadsheets to
be able to plot and analyze any results. In some instances, they must also
modify certain data values, which can only be done locating the correct file and
searching through it. This is an extremely tedious process, which will only
become more prevalent with time as the experiments continue to run. The team
aims to develop a solution to efficiently store and manage the large volume of
data being collected. This solution must ensure that all data is properly
labelled, organized, searchable, and have the ability to compare datasets for
efficient analysis.

\subsection{Inputs and Outputs}
The data is being stored in Excel spreadsheets. However, the data can be
exported to a Comma Separated Values (CSV) file. This will be the main input for
the Alkalytics project. The BMED device automatically exports the collected data
as a CSV file. There are no other input file types.
\newline
\newline
There are two possible outputs for this project, one of which could be an
exportable CSV file of the desired data, or a visual plot of the desired data.
The choice of output will depend on the user and their specific needs.


\subsection{Stakeholders}
The ongoing experimental study is currently of small scale. \newline
The main stakeholders currently are:
\begin{itemize}
    \item \textbf{Dr.\ Charles de Lannoy}: The project supervisor.\newline
        Dr.\ de Lannoy is a chemical engineering professor at McMaster
        University and leads the \emph{de Lannoy Lab}, the research
        conducting the ocean alkalinity enhancement experiments.
    \item \textbf{Bassel Abdelkader}: The secondary supervisor.\newline
        A Postdoctoral researcher working alongside Dr.\ de Lannoy to run the
        experiments and manage the data.
    \item \textbf{Current students/members of the lab} \newline
        Student researchers involved in the experimental study may participate
        for a limited time. Although they still interact with the experimental
        data, they may not be considered a primary stakeholder.
    \item \textbf{Lab funding sources} \newline
        Organizations or individuals funding the research project. They may be
        interested in new tools being developed to enhance the data management
        and analysis.
\end{itemize}

\subsection{Environment}
This software will be compatible with any browser on a computer that is
connected to the internet. This software will also be compatible with Windows 10
and 11. There is no hardware component.


\section{Goals}
This project has a total of six goals.
\subsection{Consolidate Organized Accurate Data with Proper Labels}
The raw data input for this project is a CSV file exported from Excel. With
hundreds of experiments, each containing ten or more attributes, a single
dataset can contain over thousands of recorded values. As the study expands,
the current data format becomes unmaintainable. Therefore, one of the primary
goals is to migrate all the existing data to a solution that is both scalable
and extendable. This migration must ensure that the data is properly labelled
according to its original strcuture. This goal can be measured by the percentage
of data successfully imported and accurately labelled.

\subsection{Querying from Multiple Sets}
Information about experiments is stored in a single spreadsheet, with each
record having its own dedicated raw data file. Although the data is organized
within these files, it is difficult to find and query multiple datasets
efficiently. The only method the research team currently has to achieve this is
manually finding the datasheet filename with the corresponding experiment date.
This project should be able to pull data across sheets quickly and efficiently.
This goal can be measured by the time it takes for a query to complete.

\subsection{Inter-Parameter Comparability}
To improve the data analysis, the product should allow for comparisons of
multiple parameters. This could include a simple comparison between two or more
attributes, or narrowing down the search window. One of the goals for this
project is to provide users with the control over the data they wish to view,
and the ability to define the specific  ranges for that data. This goal can be
measured by the number of parameters that can be searched for/compared at once.

\subsection{Web Interface}
The product should have a web interface to enable the user to interact with the
backend processes easily.
This interface should be user-friendly and display all necessary functions for
data input and retrieval. The returned data should be presented in a readable
format. This goal can be measured by how many of the required tasks it can
successfully perform, and how intuitive it is for the user.

\subsection{Flexibility in Expansion (i.e.\ adding additional parameters)}
Although one of the core goals of this project is to migrate the existing data
to a better solution, the product should be able to support future expansion. As
the experiment evolves, the research team may identify new parameters that may
need to be recorded, or they may want this tool to accomodate other experiments
that do not follow the same format. The product should be adaptable to handle
such new changes and serve as a general data recording tool. This goal can be
measured by How well the product handles unconventional data.

\subsection{Visual Plot of Specified Parameters and Results}
The product should be capable of generating visual representations of the
retrieved data. Users should be allowed to specify the desired format for these
visualizations, which should also ideally be exportable. The data must be
presented in a way that is clear and easy to understand for it to be meaningful
to the user. This goal can be measured by the quality of the generated
visualizations.

\section{Stretch Goals}
Stretch goals show what additional features that could be implemented once the
primary goals of the project are achieved. There is a total of four stretch goals
for this project.
\subsection{Automatically Download Data from Data Collection Device}
The current experiment setup uses a BMED device that collects the raw
experiment data. This device only stores a certain number of experiment trials,
and the data must also be manually exported daily. If the data is not exported
within a week, it will be overwritten. As a stretch goal, the product could be
integrated with this device to automatically upload the data either after each
experiment or when a predetermined limit is reached.
This is a stretch goal as it requires a deeper understanding from the developing
team of how the device records the data. This goal can be measured by the
frequency of successful automatic downloads and by the amount of data retained
without loss.

\subsection{Dynamic Dashboard to Generate Comparison Reports}
As an extension of one of the main project goals, the visual plots can be
enhanced with a dynamic dashboard that includes a brief written summary of the
requested data requested along with the plotted graph. Users could also be given
the ability to select the type of visual plot. The dashboard could dynamically
adapt to additional information, user interactions and removals without having
to regenerate the entire plot. This goal can be measured by how well the plots
change dynamically.

\subsection{Mobile Development and Accessibility}
This project is intended to have a website interface. As a stretch goal, the
project could be extended to include a mobile app for improved ease of access
and accessability. Both the website and mobile app could incorporate more
specific accessibility features, such as high-contrast colour schemes, screen
reader compatability, larger fonts and responsiveness for users with visual
impairments. This goal can be measured by how many of these extra features are
successfully implemented.
 
\subsection{Machine Learning Analysis and Projections}
One major stretch goal is to integrate artificial intelligence into the data
analysis. By incorporating a machine learning algorithm, the system could
automatically query data, customise queries, analyze the data, and generate a
customizable dashboard based on user prompts.
The machine learning component could also be trained to provide predictions
based on previous analyses or specific user prompts.

\section{Challenge Level and Extras}

The assigned challenge level for this project is general. This project does
not have a direct need for extended research on the development side, however,
features and abilities required to develop the product may have its own
challenges.

\noindent There are two extra components the team has chosen to complete for this
project. 
\begin{itemize}
    \item Usability Testing
    \item User documentation
\end{itemize}
These extras may be subject to change. However, the team has determined that
they will help to ensure the project progresses smoothly and keep stakeholders
informed about its status, development, and usage.

\newpage{}

\section*{Appendix --- Reflection}

\begin{enumerate}
    \item What went well while writing this deliverable? \newline \newline
        When writing this deliverable, it was simple in a sense where many of
        the topics and ideas presented here were talked about before hand with
        both the team and the supervisors. It was more of a matter of expanding
        on ideas and topics in further detail to ensure that the problem and
        goals were aligned with previous conversations. This allowed the writing
        process to be more time consuming than actually being difficult. 
    \item What pain points did you experience during this deliverable, and how
    did you resolve them? \newline \newline
        The biggest pain point during this deliverable was figuring out how to
        properly explain the problem and goals. Many if not all ideas presented
        were simply ideas. It was quite difficult to be able to not only explain
        the goals but to also connect it back to why they are important to have
        in this project. Displaying the understanding of how pieces fit together
        and explaining why certain things are in scope was the most challenging
        part of this deliverable. 
    \item How did you and your team adjust the scope of your goals to ensure
    they are suitable for a Capstone project (not overly ambitious but also of
    appropriate complexity for a senior design project)? \newline \newline
        As a team, this was very much a concern in the beginning. This resulted
        in our team setting up strong measures of communication. Before even
        starting on this deliverable we made sure to go through each section
        with the supervisor. We had discussions throughout the meeting where the
        team brings a more technical point of view to aid in the supervisor's
        needs and wants. It was through this process that we did not have to
        adjust the scope of the groups goals much as we started with the a
        strong foundation and understanding with all parties involved.  
\end{enumerate}  

\end{document}


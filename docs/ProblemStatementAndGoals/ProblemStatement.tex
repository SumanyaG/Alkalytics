\documentclass{article}

\usepackage{tabularx}
\usepackage{booktabs}

\title{Problem Statement and Goals\\\progname}

\author{\authname}

\date{}

%% Comments

\usepackage{color}

\newif\ifcomments\commentstrue %displays comments
%\newif\ifcomments\commentsfalse %so that comments do not display

\ifcomments
\newcommand{\authornote}[3]{\textcolor{#1}{[#3 ---#2]}}
\newcommand{\todo}[1]{\textcolor{red}{[TODO: #1]}}
\else
\newcommand{\authornote}[3]{}
\newcommand{\todo}[1]{}
\fi

\newcommand{\wss}[1]{\authornote{blue}{SS}{#1}} 
\newcommand{\plt}[1]{\authornote{magenta}{TPLT}{#1}} %For explanation of the template
\newcommand{\an}[1]{\authornote{cyan}{Author}{#1}}

%% Common Parts

\newcommand{\progname}{Software Engineering} % PUT YOUR PROGRAM NAME HERE
\newcommand{\authname}{Team 21, Alkalytics
\\ Sumanya Gulati - gulats10
\\ Kate Min - mink9
\\ Jennifer Ye - yej52
\\ Jason Tran - tranj78} % AUTHOR NAMES                  

\usepackage{hyperref}
    \hypersetup{colorlinks=true, linkcolor=blue, citecolor=blue, filecolor=blue,
                urlcolor=blue, unicode=false}
    \urlstyle{same}
                                


\begin{document}

\maketitle

\begin{table}[hp]
\caption{Revision History} \label{TblRevisionHistory}
\begin{tabularx}{\textwidth}{llX}
\toprule
\textbf{Date} & \textbf{Developer(s)} & \textbf{Change}\\
\midrule
09-19-2024 & Jennifer Ye & Inital Rough Work \\
Date2 & Name(s) & Description of changes\\
... & ... & ...\\
\bottomrule
\end{tabularx}
\end{table}


\section{Problem Statement}
This project aims to aid in the data management and analysis of an ocean
alkalinity enhancement and bipolar membrane electrodialysis experiment process.


\subsection{Background}
The goal is to develop a scalable process to capture CO\textsubscript{2} using a
combination of electric fields and membranes. The experiment process creates a
very dilute base using this very efficient method. Despite being
thermodynamically efficient, the low concentration limits practical
applications. However, as earth's temperature rises, so does the ocean's. This
affects its ability to absorb CO\textsubscript{2}. However, the resulting
product of this process can be used to increase the CO\textsubscript{2}
absorption by adding the very dilute base from the experiment back into the
ocean. This could be a great additional solution to the battle of climate
change. This would require a massive volume of solution to have a significant
impact.

\subsection{Problem}
As this process is still on a small scale, it is currently being perfected which
means many more experiments must be done to be able to bring it to a global
scale. The process has many niche technical challenges which requires high
precision for success. Optimization of data is critical to improve process
efficiency. This is a big data software problem requiring the ability to find
and fine-tune specific parameters.
\newline
As such, this team aims to provide a solution to be able to hold and manage
large amounts of data being collected in the ongoing experimental system. The
data must be labeled, organized and have the ability to search and compare
across datasets for details efficiently.
\newline

\subsection{Inputs and Outputs}
Currently the data is being stored across Excel sheets. However, the data can be
exported to a csv file. This will be the main input for the Alkalytic project.
The data collection device will also export the data as a csv file. There are no
other input file types.
\newline
There are two possible outputs of this project. The exact choice will depend on
the users' needs. The output could be an exportable csv file of the desired
data. It could be a visual plot based on the data. Both are outputs, but which
one will be outputed is up to the user


\subsection{Stakeholders}
Since this experiment study is still on-going, it is of small scale. \newline
The main stakeholders currently are:
\begin{itemize}
    \item Dr.\ Charles De Lannoy
    \item Bassel Abdelkader
    \item Current students/members of the lab working on the study
    \item The founder of the study
\end{itemize}

\subsection{Environment}
This software will be compatible with any browser with a computer that is
connected to the internet. This software will also be compatible with Windows 10
and 11. There is no hardware component.


\section{Goals}
This project has a total of seven goals.
\subsection*{Consolidated Organized Accurate Data with proper labels}
The raw data input for this project is a csv file exported from Excel. With
hundreds of experiments, where each has ten or more attributes will account for
thousands of data points. As the study gets larger, the current data collection
format is not sustainable. As such it is a goal to be able to migrate all the
existing data to another solution where the solution is not only scalable but
also extendable. To migrate all the data also entails that the data is properly
labelled as it had been originally.

\subsection*{Querying from Multiple Sets}
The experiment has multiple pages of Exec sheets that data is recorded on.
Although the data is separated and organised across sheets, to be able to
analyse the data queries will need to be made. This project should be able to
pull data across sheets quickly and efficiently. Currently, retrieving the data
is manual, slow and tedious. This project strives to make data retrieval easy
and efficient.  

\subsection*{Inter-Parameter Comparability}
To further aid the data analysis of the experiment data, the product should be
able to compare multiple parameters. This could be a simple binary comparison of
two attributes or more, but also narrowing down the search windows. For example,
the user could specify the time frame of when the experiments were done. Within
that time frame, they might want to compare three different attributes from
those experiences. One of the goals for this project is to be able to make the
above example possible.

\subsection*{Web Interface}
With a web interface, the user can easily interact with the backend processes.
This interface should be user friendly and displays all the functions that are
needed to input data, and return the appropriate data. The returned data should
be given in a readable way.

\subsection*{Flexibility in Expansion (i.e.\ adding additional parameters)}
Although a big portion of this project is to migrate the existing data into a
better solution, the product should also be built with the possibility of
expansion. As the experiment grows, there could be a new parameter to be
recorded. There could also be the possibility for this tool to be used in other
experiments that do not follow the same format. The product should still be able
to handle those new changes, among others. It is our goal to be able to create a
tool to be used as a general data recording tool.

\subsection*{Visual Plot of Specified Parameters + Results}
The product should be able to use the data it retrieves to interpret it in a
visual manner. This product must be able to plot the data in the desired format
of the user to then be exported. As the data will not be completely useful if it
is not readable or understandable to the user.

\section{Stretch Goals}
\subsection*{Automatically Download Data from Data Collection Device}
The current state of the experiment requires a data collection device. This
device only holds a certain number of experiment trials. This data must also be
manually exported at the end of each day. If the device is powered off or hits
the limit of total possible recorded experiments, all the data will be lost. As
a stretch goal, the product could be connected to this device and upload the
data either when it finishes collecting data per experiment or once it hits a
limit. This is a stretch goal as it requires the team to go through how the
device actually records the data.

\subsection*{Dynamic Dashboard to Generate Comparison Reports}
As an extension of one of the main project goals, the visual plot can be in the
form of a dashboard with a small written conclusion of the data requested
instead of just the plotted graph. The choice of visual plot can also be given
to the user. The dashboard could change to dynamically to newly added
information, user interactions and removals without having to regenerate the
entire plot from scratch.  

\subsection*{Mobile Development and Accessibility}
This project is intended to have a website interface. As a stretch goal, the
project could be extended to a mobile app for easy access and reachability. The
website and mobile app could also incorporate more accessibility features for
those with disabilities. This may be but not limited to colour contrasts, screen
readers, big fonts and responsiveness to those who have trouble seeing.
 
\subsection*{Machine Learning Analysis and Projections}
One major stretch goal is to incorporate an artificial intelligence machine
learning to data analysis. In hopes to make data analysis easier for the user
and more adaptable to any possible analysis. By introducing an machine learning
algorithm, it would be able to automatically query the data, customise that
query, analyse the data and create a customizable dashboard board as well based
on any sort of prompt. The machine learning aspect can also automatically return
conclusions based on prompts

\section{Challenge Level and Extras}

The asigned challange level for this project is general. As this project does not have any direct need for research, 
the features and abilities needed to be developed may still be more complicated than aniticiapted. 
As such, there are three extra components the team has chosen to take up for this project. 
\begin{itemize}
    \item Usability Testing
    \item User documentation
\end{itemize}
This is not a final list of extras, however currently as a team we decided that
these extras will not only benifet us as a team with ensuring the project is
continuously moving forward but also ensuring that the stakeholders are kept in
the know of how the project is progrssing

\newpage{}

\section*{Appendix --- Reflection}

The purpose of reflection questions is to give you a chance to assess your own
learning and that of your group as a whole, and to find ways to improve in the
future. Reflection is an important part of the learning process.  Reflection is
also an essential component of a successful software development process.  

Reflections are most interesting and useful when they're honest, even if the
stories they tell are imperfect. You will be marked based on your depth of
thought and analysis, and not based on the content of the reflections
themselves. Thus, for full marks we encourage you to answer openly and honestly
and to avoid simply writing ``what you think the evaluator wants to hear.''

Please answer the following questions.  Some questions can be answered on the
team level, but where appropriate, each team member should write their own
response:



\end{document}


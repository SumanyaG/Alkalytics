\documentclass{article}

\usepackage{tabularx}
\usepackage{booktabs}

\title{Problem Statement and Goals\\\progname}

\author{\authname}

\date{}

%% Comments

\usepackage{color}

\newif\ifcomments\commentstrue %displays comments
%\newif\ifcomments\commentsfalse %so that comments do not display

\ifcomments
\newcommand{\authornote}[3]{\textcolor{#1}{[#3 ---#2]}}
\newcommand{\todo}[1]{\textcolor{red}{[TODO: #1]}}
\else
\newcommand{\authornote}[3]{}
\newcommand{\todo}[1]{}
\fi

\newcommand{\wss}[1]{\authornote{blue}{SS}{#1}} 
\newcommand{\plt}[1]{\authornote{magenta}{TPLT}{#1}} %For explanation of the template
\newcommand{\an}[1]{\authornote{cyan}{Author}{#1}}

%% Common Parts

\newcommand{\progname}{Software Engineering} % PUT YOUR PROGRAM NAME HERE
\newcommand{\authname}{Team 21, Alkalytics
\\ Sumanya Gulati - gulats10
\\ Kate Min - mink9
\\ Jennifer Ye - yej52
\\ Jason Tran - tranj78} % AUTHOR NAMES                  

\usepackage{hyperref}
    \hypersetup{colorlinks=true, linkcolor=blue, citecolor=blue, filecolor=blue,
                urlcolor=blue, unicode=false}
    \urlstyle{same}
                                


\begin{document}

\maketitle

\begin{table}[hp]
\caption{Revision History} \label{TblRevisionHistory}
\begin{tabularx}{\textwidth}{llX}
\toprule
\textbf{Date} & \textbf{Developer(s)} & \textbf{Change}\\
\midrule
09-19-2024 & Jennifer Ye & Inital Rough Work \\
09-23-2024 & Jennifer Ye & Final Edits After Team Meet\\
02-22-2025 & Jennifer Ye, Kate Min & Incorporated Feedback\\ 
\bottomrule
\end{tabularx}
\end{table}



\section{Problem Statement}
This project aims to aid in the data management and analysis of an ocean
alkalinity enhancement experiment process.

\subsection{Background}
The goal is to develop a scalable process to capture CO\textsubscript{2} using a
combination of electric fields and membranes. The experiment process creates a
very dilute base using a very efficient generation method. Although this process
is thermodynamically efficient, the low concentration limits practical
applications. As earth's temperature rises, so does the ocean's, affects its
ability to absorb CO\textsubscript{2}. However, the resulting product of this
process can be used to increase the CO\textsubscript{2} absorption by adding the
very dilute base from the experiment back into the ocean. This could be a great
additional solution to the battle of climate change but will require a massive
volume of solution to have a significant impact.
This project aims to aid in the data management and analysis of an ocean
alkalinity enhancement experiment process.

\subsection{Background}
The goal is to develop a scalable process to capture CO\textsubscript{2} using a
combination of electric fields and membranes. The experiment process creates a
very dilute base using a very efficient generation method. Although this process
is thermodynamically efficient, the low concentration limits practical
applications. As earth's temperature rises, so does the ocean's, affects its
ability to absorb CO\textsubscript{2}. However, the resulting product of this
process can be used to increase the CO\textsubscript{2} absorption by adding the
very dilute base from the experiment back into the ocean. This could be a great
additional solution to the battle of climate change but will require a massive
volume of solution to have a significant impact.

\subsection{Problem}
As this process is still on a small scale, it is currently being perfected which
means many more experiments must be done to be able to bring it to a global
scale. The process has many niche technical challenges which requires high
precision for success. Optimization of data is critical to improve process
efficiency. This is a big data software problem requiring the ability to find
and fine-tune specific parameters. \newline \newline
Currently, all the experiment data is stored in many Excel files, one for each
day's worth of experiments. This is a very inefficient way of collecting and
storing data. As the study gets bigger and bigger, the amount of data is
becoming less manageable. The study also needs to draw conclusions from the
data. Currently, they need to manually pick through the data and either plot it
or draw their own conclusions. Changing certain data points will also require
someone to find the correct data sheet and sift through all the data points to
find the one to be edited. All of this is a very tedious process, which will
exponentially increase with time as the experiment gets bigger. The team aims to
provide a solution to store and manage large amounts of data being collected in
the ongoing experimental system. The data must be labeled, organized, and have
the ability to search and compare across datasets for details efficiently.
The team aims to provide a solution to store and manage large amounts of data
being collected in the ongoing experimental system. The data must be labelled,
organized and have the ability to search and compare across datasets for details
efficiently.

\subsection{Inputs and Outputs}
Currently the data is being stored across Excel sheets. However, the data can be
exported to a Comma Separated Values (CSV) file. This will be the main input for
the Alkalytic project. The data collection device will also export the data as a
CSV file. There are no other input file types.
\newline
\newline
There are two possible outputs from this project. The exact choice will depend
on the users' needs. The output could be an exportable CSV file of the desired
data. It could also be a visual plot of desired data. Both are outputs of this
project, but which one will be produced is decided by the user.


\subsection{Stakeholders}
Since this experiment study is still on-going, it is of small scale. \newline
The main stakeholders currently are:
\begin{itemize}
    \item Dr.\ Charles De Lannoy \newline
        The project supervisor. He is the one who discovered the alkalinity
        enhancement experiment process and is working on expanding the scale. 
    \item Bassel Abdelkader \newline
        A Doctor of Philosophy student who is working along side Dr.\ Charles De
        Lannoy in running the experiments and analysis. 
    \item Current students/members of the lab working on the study \newline
        There may be new students getting involved with the study project.
        However, they may be there for a short period of time. Although they
        will still be interacting with the experimental data, they may not be a
        primary stakeholder. 
    \item The founder of the study \newline
        The founder of the study is the one who is currently funding the
        research project. They will be interested in any new tools being used to
        further enhance the analysis or data management process.
\end{itemize}

\subsection{Environment}
This software will be compatible with any browser on a computer that is
connected to the internet. This software will also be compatible with Windows 10
and 11. There is no hardware component.


\section{Goals}
This project has a total of six goals.
\subsection{Consolidate Organized Accurate Data with Proper Labels}
The raw data input for this project is a CSV file exported from Excel. With
hundreds of experiments, where each has ten or more attributes will account for
thousands of data points. As the study gets larger, the current data collection
format is not sustainable. As such it is a goal to be able to migrate all the
existing data to another solution where the solution is not only scalable but
also extendable. To migrate all the data also means that the data is properly
labelled as it had been originally. This goal can be measured by a percentage of
data properly imported and labelled.

\subsection{Querying from Multiple Sets}
The experiment has multiple pages of Excel sheets that data is recorded on.
Although the data is separated and organised across sheets, in order to analyse
the data, queries will need to be made. This project should be able to pull data
across sheets quickly and efficiently. Currently, retrieving the data is manual,
slow and tedious. This project strives to make data retrieval easy and
efficient. This goal can be measured by the time it takes for a query to
complete.  

\subsection{Inter-Parameter Comparability}
To further aid the data analysis of the experiment data, the product should be
able to compare multiple parameters. This could be a simple comparison of two
attributes or more, but also narrowing down the search windows.  One of the
goals for this project is to facilitate control over the data to be seen, as
well as the ranges for that data. This goal can be measured by the number of
parameters that can be search/compared at once.

\subsection{Web Interface}
A web interface allows the user to easily interact with the backend processes.
This interface should be user friendly and displays all the functions that are
needed to input data, and return the appropriate data. The returned data should
be given in a readable format. This goal can be measured by how many of the
required tasks it can successfully perform, and if they are clear for the user.

\subsection{Flexibility in Expansion (i.e.\ adding additional parameters)}
Although a big portion of this project is to migrate the existing data into a
better solution, the product should also be built with the possibility of
expansion. As the experiment grows, there could be a new parameter to be
recorded. There could also be the possibility for this tool to be used in other
experiments that do not follow the same format. The product should still be able
to handle those new changes, among others. It is our goal to be able to create a
tool to be used as a general data recording tool. This goal can be measured by
how well the application adapts to unconvential data.

\subsection{Visual Plot of Specified Parameters and Results}
The product should be able to use the data it retrieves and interpret it in a
visual manner. This product must be able to plot the data in the desired format
of the user, which can then be exported. The data will not be useful if it is
not readable or understandable to the user. This goal can be measured by the
quality of the generated visualizations.

\section{Stretch Goals}
Stretch goals show what additional features that could be implemented as the
main goals of the project are achieved. There is a total of four stretch goals
for this project.
\subsection{Automatically Download Data from Data Collection Device}
The current state of the experiment requires a data collection device. This
device only holds a certain number of experiment trials and the data must also
be manually exported at the end of each day. If the recorded data is not
exported within a week, all the data will be overwritten. As a stretch goal, the
product could be connected to this device and upload the data either when it
finishes collecting data per experiment or once it hits a predetermined limit.
This is a stretch goal as it requires the team to gain a deeper understanding of
how the device records the data. This goal can be measured by how often data is
successfully automatically downloaded, or by how little data is missing.

\subsection{Dynamic Dashboard to Generate Comparison Reports}
As an extension of one of the main project goals, the visual plot can be in the
form of a dashboard with a small written conclusion of the data requested
instead of just the plotted graph. The choice of visual plot can also be given
to the user. The dashboard could dynamically adapt to additional information,
user interactions and removals without having to regenerate the entire plot from
scratch. This goal can be measured by how well the plots changes dynamically.

\subsection{Mobile Development and Accessibility}
This project is intended to have a website interface. As a stretch goal, the
project could be extended to a mobile app for easy access and reachability. The
website and mobile app could also incorporate more accessibility features for
those with disabilities. This may be but not limited to colour contrasts, screen
readers, big fonts and responsiveness to those who have trouble seeing. This can
be measured by how many of these extra features are successfully implemented.
 
\subsection{Machine Learning Analysis and Projections}
One major stretch goal is to incorporate artificial intelligence into the data
analysis. In hopes to make data analysis easier for the user and more adaptable
to any possible analysis. By introducing a machine learning algorithm, it would
be able to automatically query the data, customise that query, analyse the data
and create a customizable dashboard board as well based on any sort of prompt.
The machine learning aspect can also automatically return conclusions based on
prompts.

\section{Challenge Level and Extras}

The assigned challange level for this project is general. As this project does
not have any direct need for research on the development side, the features and
abilities needed to develop may still be more complicated than aniticipated. 

\noindent There are two extra components the team has chosen to take up for this
project. 
\begin{itemize}
    \item Usability Testing
    \item User documentation
\end{itemize}
This is list of extras is subject to change. However the team has decided that
these extras will not only benefit us as a team with ensuring the project is
continuously moving forward but also ensuring that the stakeholders are aware of
how the project is progressing.

\newpage{}

\section*{Appendix --- Reflection}

\begin{enumerate}
    \item What went well while writing this deliverable? \newline \newline
        When writing this deliverable, it was simple in a sense where many of
        the topics and ideas presented here were talked about before hand with
        both the team and the supervisors. It was more of a matter of expanding
        on ideas and topics in further detail to ensure that the problem and
        goals were aligned with previous conversations. This allowed the writing
        process to be more time consuming than actually being difficult. 
    \item What pain points did you experience during this deliverable, and how
    did you resolve them? \newline \newline
        The biggest pain point during this deliverable was figuring out how to
        properly explain the problem and goals. Many if not all ideas presented
        were simply ideas. It was quite difficult to be able to not only explain
        the goals but to also connect it back to why they are important to have
        in this project. Displaying the understanding of how pieces fit together
        and explaining why certain things are in scope was the most challenging
        part of this deliverable. 
    \item How did you and your team adjust the scope of your goals to ensure
    they are suitable for a Capstone project (not overly ambitious but also of
    appropriate complexity for a senior design project)? \newline \newline
        As a team, this was very much a concern in the beginning. This resulted
        in our team setting up strong measures of communication. Before even
        starting on this deliverable we made sure to go through each section
        with the supervisor. We had discussions throughout the meeting where the
        team brings a more technical point of view to aid in the supervisor's
        needs and wants. It was through this process that we did not have to
        adjust the scope of the groups goals much as we started with the a
        strong foundation and understanding with all parties involved.  
\end{enumerate}  

\end{document}


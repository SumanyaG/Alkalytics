\documentclass{article}

\usepackage{booktabs}
\usepackage{tabularx}
\usepackage{array}

\title{Development Plan\\\progname}

\author{\authname}

\date{}

%% Comments

\usepackage{color}

\newif\ifcomments\commentstrue %displays comments
%\newif\ifcomments\commentsfalse %so that comments do not display

\ifcomments
\newcommand{\authornote}[3]{\textcolor{#1}{[#3 ---#2]}}
\newcommand{\todo}[1]{\textcolor{red}{[TODO: #1]}}
\else
\newcommand{\authornote}[3]{}
\newcommand{\todo}[1]{}
\fi

\newcommand{\wss}[1]{\authornote{blue}{SS}{#1}} 
\newcommand{\plt}[1]{\authornote{magenta}{TPLT}{#1}} %For explanation of the template
\newcommand{\an}[1]{\authornote{cyan}{Author}{#1}}

%% Common Parts

\newcommand{\progname}{Software Engineering} % PUT YOUR PROGRAM NAME HERE
\newcommand{\authname}{Team 21, Alkalytics
\\ Sumanya Gulati - gulats10
\\ Kate Min - mink9
\\ Jennifer Ye - yej52
\\ Jason Tran - tranj78} % AUTHOR NAMES                  

\usepackage{hyperref}
    \hypersetup{colorlinks=true, linkcolor=blue, citecolor=blue, filecolor=blue,
                urlcolor=blue, unicode=false}
    \urlstyle{same}
                                


\begin{document}

\maketitle

\begin{table}[hp]
\caption{Revision History} \label{TblRevisionHistory}
\begin{tabularx}{\textwidth}{llX}
\toprule
\textbf{Date} & \textbf{Developer(s)} & \textbf{Change}\\
\midrule
09-20-2024 & Kate Min & Added Team Meeting Plan\\
09-21-2024 & Kate Min & Added Team Communication Plan and Roles\\
09-22-2024 & Kate Min & Added Team Charter, License Information, and
Introduction\\
09-23-2024 & Jason Tran & Added Workflow Plan, Project Decomposition and Scheduling, Expected Technology and
Coding Standard\\
09-23-2024 & Sumanya Gulati & Added Proof of Concept Demonstration Plan\\
09-24-2024 & Sumanya Gulati & Added Appendix - Reflection\\
... & ... & ...\\
\bottomrule
\label{table:1}
\end{tabularx}
\end{table}

\newpage{}

\noindent This document outlines Team Alkalytics' project development
plan.\newline

\noindent Administrative details such as copyright and Intellectual Property
(IP) agreements, the team's meeting schedule, communication methods, and
expected roles are thoroughly covered.\newline

\noindent This is followed by details about the team's planned workflow, project
deliverables, scheduling, a Proof of Concept (POC) demonstration plan, and an
expected technology stack for the project and relevant coding standards.\newline

\noindent The appendices include a reflection on the writing process of this
document as well as a team charter that outlines a few ground rules and
expectations for the team.

\section{Confidential Information?}

There is no confidential information to protect with respect to this project.

\section{IP to Protect}

There is IP to protect with respect to this project. Currently, the type of
agreement that the team may be required to sign is in discussion between the
primary supervisor, Dr.\ Charles de Lannoy, and the Capstone course instructor,
Dr.\ Spencer Smith. Once the agreement has been decided and signed, this section
will be updated to reflect the terms of the agreement.

\section{Copyright License}

The project will be adopting the license decided by Dr.\ de Lannoy and Dr.\
Smith. As this is still in discussion, the project will temporarily adopt the
standard MIT license, provided
\href{https://github.com/SumanyaG/Alkalytics/blob/main/LICENSE}{here}. Once the
appropriate license for the project has been decided, this section and the
linked license file in the repository will be updated to reflect the proper
license agreement.

\section{Team Meeting Plan}
This section provides detail on the team's internal and external meeting plans.
\subsection{Weekly Meetings}
The team will have weekly check-ins on Fridays from 3:30PM to 4:30PM, either
in-person on campus or virtually, depending on member's availabilities. Virtual
meetings will be hosted on Microsoft Teams. Additional meetings will be
scheduled as needed.\newline

\noindent The current meeting chair should open a GitHub issue prior to a
meeting. During a meeting, the chair will go over each item on a prepared agenda
and all members participate in the discussion. The notetaker will record meeting
minutes and post them as a comment under the GitHub issue after the meeting, as
well as a summary of task assignments.

\subsection{Supervisor Meetings}
The team will have biweekly or monthly meetings with the primary supervisor,
Dr.\ Charles de Lannoy, depending on his availability. The team will use his
Outlook calendar to arrange appropriate meeting dates. These meetings will be
held either in his office or virtually on Microsoft Teams.\newline

\noindent Weekly meetings with the secondary supervisor, Bassel Abdelkader will
be on Tuesdays from 3:30PM to 4:30PM, as he works closely with the data that
must be integrated into the project. These meetings will be virtual due to his
schedule.

\section{Team Communication Plan}

The team uses an Instagram group chat for regular communication and a Microsoft
Teams group chat for drafting emails and sharing resources.\newline

\noindent Communication with supervisors will take place through email or
scheduled meetings. The team liaison will be responsible for sending the emails
and should Carbon Copy (CC) all team members.\newline

\noindent All project-related communication will be done via GitHub issues. Each
issue will track a specific task and assign team members. Every pull request
should link to its corresponding issue(s), and feedback is expected from all
team members when a pull request is created.

\section{Team Member Roles}

The team plans to rotate through the following roles between every stage,
defined in Table \ref{table:2}:

\begin{itemize}
  \item \textbf{Meeting chair}: The meeting chair should prepare an agenda for
  each meeting and open an issue for the meeting. The issue should contain an
  attendance tracker, the agenda, and other topics that may need to be
  discussed.
  \item \textbf{Notetaker}: The notetaker will record meeting minutes during a
  meeting and post it as a comment under the meeting issue, along with a summary
  of task assignments.
  \item \textbf{Primary reviewer for pull requests}: The primary reviewer will
  give the official ``approval'' for pull requests. All members should review
  pull requests and provide feedback, but the primary reviewer will make the
  final decision to ensure a smooth workflow.
\end{itemize}

\noindent Once implementation begins, all members are expected to contribute as
developers. However, the team anticipates appointing technical leads based on
the stage of development and the members' skillsets, described below:

\begin{itemize}
  \item Sumanya has experience in data analytics and working with large
  datasets, and will also serve as the main point of communication between the
  team and the project's stakeholders.
  \item Kate has experience in web development and programming in Python, as
  well as proficiency in several Python libraries that may need to be used.
  \item Jason has experience in MongoDB and machine learning, as well as
  proficiency in full-stack development.
  \item Jennifer has experience working with large datasets and employing data
  visualization tools like PowerBI, along with a strong background in web
  development.
\end{itemize}

\section{Workflow Plan}

During each milestone, tasks will be delegated to the developers from the
project board. To start developing, they will first create dedicated branches
for their work and will develop independently, first by pulling changes from the
main branch and then following the project's coding standards. Once their work
is complete, they will submit a pull request for review, where it will be
tested, reviewed by peers, and merged upon approval. This workflow ensures that
development progresses smoothly, with ownership, and consistent quality.

\subsection{Development with Git}

Developers will adhere to the following convention and process.

\subsubsection{Branches \& Commits}

Each project milestone will be managed using a single overviewing pre-production
branch. Developers will do \texttt{git checkout -B "branch-name"} off this
branch to create their respective dedicated work branch to make commits,
ensuring that progress is isolated and well-documented for each task.

\begin{itemize}
    \item \textbf{Branch Naming Convention}: \texttt{branchType/branchDesc}
        \begin{itemize}
            \item Types:
                \begin{itemize}
                    \item \texttt{documentation}: new documentation
                    \item \texttt{frontend-feat}: new high-level frontend
                    implementation
                    \item \texttt{backend-feat}: new high-level backend
                    implementation
                    \item \texttt{improvement}: an upgrade to an existing
                    feature
                    \item \texttt{bugfix}: a solution to an error
                \end{itemize}
            \item Example: \texttt{documentation/hazard-analysis}
        \end{itemize}
    \item \textbf{Commit Naming Convention}: \texttt{commitType(directory):
    desc}
        \begin{itemize}
            \item Types:
                \begin{itemize}
                    \item \texttt{feat}: new feature added
                    \item \texttt{fix}: a solution to an error
                    \item \texttt{imp}: an upgrade to an existing feature
                    \item \texttt{doc}: any documentation changes
                \end{itemize}
            \item Example: \texttt{fix(src): fix plot flickering during
            rerender}
        \end{itemize}
\end{itemize}

\subsubsection{Merging Strategy}

Before a developer merges their individual branch into the milestone branch via
a pull request, all commits within an overviewing directory will be squashed
into a single commit. Creating a clean, consolidated history for the developer’s
branch. However, when merging the milestone (pre-production) branch into the
main release, individual commits will be preserved to maintain ownership of
changes. Throughout the project, developers will use rebase and merge to keep a
clean Git history, resolving conflicts locally before merging.

\begin{itemize}
    \item \textbf{Pull Requests}:  Is a request to merge code changes from
    one branch into another. Each pull request requires the approval of at least two
    reviewers before merging.
    \begin{itemize}
        \item \textbf{Code Review}: Reviewers must check for code quality,
        adherence to the coding standard, and functional correctness before
        approving a pull request. Any issues or discussions must be contained
        within the pull request for tracking.
    \end{itemize}
\end{itemize}

\subsubsection{Continuous Integration (CI)}

The team use automated tests and linting for both frontend, backend, and
documentation (LaTeX). These will run on each pull request upon creation and must pass
before requesting review from other developers.

\subsection{Issue Management}

Bugs and complications are inevitable during the development process, so it is
best practice to address them in a standardized manner. Upon discovering an
issue, developers are expected to open an \texttt{issue} on GitHub Projects and
follow the following procedure:

\subsubsection{Template}

Create an \texttt{issue} on the project board with a title, description, labels,
links, and any relevant information. Furthermore, categorize the issue to the
respective milestone and assign it accordingly.

\begin{itemize}
    \item \textbf{Milestone}: Each stage is associated with a milestone. Issues
    will be added to the respective milestones based on the scope.
    \item \textbf{Assignee}: The developer responsible for addressing the issue.
    \item \textbf{Title}: A short, high-level description of the issue.
    \item \textbf{Description}: A detailed description of the issue.
    \item \textbf{Labels}: Categorize using label(s) (e.g., \texttt{bug},
    \texttt{feature}, \texttt{enhancement}, and etc). Priority labels will
    indicate urgency and can be used at the team’s discretion.
    \item \textbf{Links}: Link related issues if relevant to enable tracking.
\end{itemize}


\section{Project Decomposition and Scheduling}

The project is organized around key milestones with overviewing stages (refer to Table \ref{table:2}) in which
the team will rotate roles and responsibilities to allow dynamic and fluid
progression. This will constantly use the team’s best strength and the rotation
of the roles of team member will be at the team’s discretion.

\newpage

\begin{table}[htbp]
    \centering
    \begin{tabular}{|l|l|l|}
    \hline
    \textbf{Stage} & \textbf{Milestone} & \textbf{Deadline} \\
    \hline
    Stage 1 & Problem Statement, POC Plan, Development Plan & Sept 24 \\
    \texttt{} & Requirements Document Revision 0 & Oct 9 \\
    \hline
    Stage 2 & Hazard Analysis & Oct 23 \\
    \texttt{} & V\&V Plan Revision 0 & Nov 1 \\
    \hline
    Stage 3 & POC Demonstration & Nov 11 - 22 \\
    \hline
    Stage 4 & Design Document Revision 0 & Jan 15 \\
    \hline
    Stage 5 & Revision 0 Demonstration & Feb 3 - 14\\
    \hline
    Stage 6 & V\&V Report Revision 0 & Mar 7 \\
    \hline
    Stage 7 & Final Demonstration (Revision 1) & Mar 24 - 30\\
    \texttt{} & EXPO Demonstration & Apr TBD \\
    \texttt{} & Final Documentation (Revision 1) & Apr 2 \\
    \hline
    \end{tabular}
    \caption{Project Decomposition and Deadlines}
    \label{table:2}
\end{table}

\noindent Code development will start in Stage 2 once foundational requirements are
concrete and will continue until the final demonstration.

\subsection{GitHub Projects Usage}

The project board will follow a Kanban style with columns for \texttt{To-Do},
\texttt{In Progress}, \texttt{Stuck}, \texttt{Under Review}, and \texttt{Done}.
Tasks are divided into milestones and assigned to respective team members. Each
task or issue will be represented as a card on the board, moving through the
columns as work progresses.

\section{Proof of Concept Demonstration Plan}

Currently, all the data collected from the various experiments is downloaded from the 
apparatus in the form of Comma-Separated Values (CSV) files. These files are then manually
copied onto an existing Excel template where faulty data is deleted and the rest of the data points
are labelled, filtered and then analysed. As the number of experiments and the data collected
from each experiment increase in size, this model of storing and analysing data has
become unsustainable.\\
\newline
\noindent Based on the goals outlined in section 2 of the Problem Statement and Goals document,
the following key risks have been identified that must be mitigated in the early development
stage to demonstrate the feasibility of this project:
\begin{itemize}
  \item \textbf{Data Migration}: As described in section 2.1, it is critical to ensure that all existing 
  data is migrated to a scalable and extendable database without losing integrity, labelling or structuring.
  \item \textbf{Performance in Querying}: As described in section 2.2, it must be ensured that the database can 
  efficiently handle queries to avoid bottlenecks that could affect system usability.
  \item \textbf{Parameter Comparability}: To accomplish the analysis of the consolidated datapoints, it is essential
  that the database can facilitate complex inter-parameter comparisons, as defined in section 2.3.
\end{itemize}

\noindent Since the remaining goals (2.4 to 2.7) are built upon the assumption of a functioning database that 
allows efficient querying of the data, it must be demonstrated through the Proof of Concept (PoC) demonstration that the aforementioned 
risks can be mitigated and that the project can be completed on schedule.\\
\newline
\noindent The success of the PoC Demonstration and thus, the feasibility of the project will be determined based 
on the following criteria:
\begin{itemize}
  \item Ensuring that 100\% of the existing data has been migrated without loss or error to the new database.
  \item Demonstrating that querying of data from across multiple tables can be completed within a reasonable amount of time.
  \item Demonstrating that all existing inter-parameter comparisons in the Excel templates have been replicated in the database.
\end{itemize}

\section{Expected Technology}
Table \ref{table:3} shows the expected technology for the development of this project.
\begin{table}[htbp]
    \centering
    \begin{tabularx}{\textwidth}{|l|>{\raggedright\arraybackslash}X|}  
    \hline
    \textbf{Technology} & \textbf{Purpose} \\
    \hline
    Visual Studio & Integrated development environment for developing code \\
    \hline
    GitHub & For hosting repositories, collaboration, and project tracking \\
    \hline
    Git & Version control system for managing changes and branches \\
    \hline
    TypeScript & Typed superset of JavaScript used for building front-end \\
    \hline
    Python & General-purpose programming language for backend \\
    \hline
    MongoDB & NoSQL database for managing and storing datasets \\
    \hline
    \end{tabularx}
    \caption{Expected Technology Overview}
    \label{table:3}
\end{table}

\noindent The following, Table \ref{table:4}, shows potential but expected technology to be used and are subject
to change according to the team’s discretion at any stage of the development.
\begin{table}[htbp]
    \centering
    \begin{tabularx}{\textwidth}{|l|>{\raggedright\arraybackslash}X|}  
    \hline
    \textbf{Technology} & \textbf{Purpose} \\
    \hline
    React & JavaScript library for building dynamic user interfaces \\
    \hline
    Apollo Client & GraphQL client for managing queries and local state \\
    \hline
    Material-UI & Component library for creating user interfaces \\
    \hline
    GraphQL & Query language for APIs \\
    \hline
    Graphene & Python library for implementing GraphQL APIs \\
    \hline
    PyMongo & Python library for interacting with MongoDB \\
    \hline
    Pandas & Data analysis and manipulation library, used for CSV optimization
    \\
    \hline
    ESLint & JavaScript linting tool for enforcing coding standards \\
    \hline
    Flake8 & Python linting tool for ensuring code standards \\
    \hline
    \end{tabularx}
    \caption{Potential Technology Overview}
    \label{table:4}
\end{table}



\section{Coding Standard}

The team will adhere to the coding standard in Table \ref{table:5} and the appropriate linter will be in use.
\begin{table}[htbp]
    \centering
    \begin{tabularx}{\textwidth}{|l|>{\raggedright\arraybackslash}X|}
    \hline
    \textbf{Category} & \textbf{Details} \\
    \hline
    Frontend & Adhere to \href{https://standardjs.com/}{StandardJS} for
    TypeScript coding standard. \\
    \hline
    Backend & Adhere to
    \href{https://www.python.org/dev/peps/pep-0008/}{PEP 8} for Python coding
    standards. \\
    \hline
    \end{tabularx}
    \caption{Coding Standards}
    \label{table:5}
\end{table}


\newpage{}

\section*{Appendix --- Reflection}

\begin{enumerate}
    \item \textbf{Why is it important to create a development plan prior to starting the
    project?}\\
    \newline
    It is important to create a development plan prior to starting the project because
    it forces the team to have detailed conversations about the various aspects of the project.\\
    \newline
    Although our team spent a considerable amount of time discussing our experiences and the
    technologies we thought would work the best for this project (taking into account the preferences
    of our supervisors), creating a detailed development plan compelled us to decide on not just the 
    technologies but also the QA factors that we had not considered discussing such as branch and commit 
    naming conventions, coding standards and more.\\
    \newline
    Having a detailed conversation with our supervisors while coming up with a PoC Demonstration
    Plan allowed us to communicate our goals more precisely and set clear expectations in terms of timelines
    and first steps.
    \item \textbf{In your opinion, what are the advantages and disadvantages of using
    CI/CD?}\\
    \newline
    So far, our team has only implemented CI for the documentation parts of the project. This means that
    all the checks created in GitHub Actions currently only relate to LaTeX syntax checks, pdf compilation
    and other such conflicts. We plan on incorporating linters and other checks in our CI process over the
    next couple of weeks.\\
    \newline
    The initial process of figuring out how CI can be used for our project and setting it up was painstakingly
    arduous. The learning curve that came with adopting the CI checks seemed comepletely unnecessary at times
    especially because the checks take on average 10-12 minutes to complete.\\
    \newline
    With that said, however, we do forsee the advantages of using CI. We have already run into a
    couple of issues with naming conventions of the commits and branches so having a linter in place that can
    flag style violations and auto-fix formatting issues to make the codebase consistent would be very useful.
    \item \textbf{What disagreements did your group have in this deliverable, if any,
    and how did you resolve them?}\\
    \newline
    Our group did not face any conflicts or disputes. Most of our disagreements were related to writing styles
    (using third person as opposed to first person or avoiding repetition of words) and naming conventions (following
    the naming convention for all the branch names and the commit messages).\\
    \newline
    To be fair, it is an exaggeration to classify these issues as `disagreements' because for the naming convention 
    issue, it was mostly the novelty of these naming conventions that caused confusion. Once Jason (who set up CI and
    suggested the convention style) explained how to do it in detail, we did not run into the issue again.\\
    \newline
    As for the incompatible writing styles -  our team decided to review all the PRs over a 2-hour call where the writing
    style and formatting (indenting versus using newlines) discrepancies were brought up. All the team members were
    receptive to the feedback and worked together to incorporate the suggestions. In instances where we had to make
    decisions related to formatting, we held an informal vote and chose to proceed with the decided formatting style
    for all our documents to avoid confusion and maintain consistency.
\end{enumerate}

\newpage{}

\section*{Appendix --- Team Charter \cite{ref1}}

\subsection*{External Goals}

By the end of Capstone, our team wishes to deliver a project that is impactful
and useful beyond graduation. We seek the personal satisfaction that comes from
knowing the product could contribute to improvements in climate change. By
achieving these goals, we thereby anticipate and hope for an excellent grade in
the Capstone course.

\subsection*{Attendance}

This section lays out ground rules and expectations regarding team member
attendance.

\subsubsection*{Expectations}

All members are expected to attend all weekly check-in meetings, meetings with
our supervisors, and maintain punctuality. We will not require full attendance
at every Capstone lecture or tutorial, as long as at least one member from the
team attends. All forms of attendance will be logged through GitHub Issues.

\subsubsection*{Acceptable Excuse}

It will not be acceptable for a member to miss a meeting or a deadline without
any prior communication. Missing a few meetings will be acceptable as long as
the member communicates their absence to the rest of the team at least 1-2 hours
in advance. There will be no acceptable excuses for missing a deadline unless
they have an emergency. The guidelines for such a situation are outlined in the
following section.

\subsubsection*{In Case of Emergency}

If a team member has an emergency and cannot finish their work or meet a
deadline, they are expected to communicate this as early as possible. The rest
of the team will discuss potential accommodations and ensure that all tasks are
completed by the deadline.

\subsection*{Accountability and Teamwork}

This section covers expectations for the team's collaboration efforts.

\subsubsection*{Quality} 

All members should prepare for weekly check-ins and meetings by doing the
following:

\begin{itemize}
  \item[(a)] Review the agenda prepared by the meeting chair in advance, and
  \item[(b)] Prepare a short progress report and any questions/concerns that
  were not already mentioned in the agenda.
\end{itemize}

\noindent Each member is expected to put in consistent effort and deliver
high-quality work that meets or exceeds the expectations described in the
applicable rubric. Code should be well-structured, well-documented, and adhere
to the coding standards outlined in Section 11 of this document.

\subsubsection*{Attitude}

All members are encouraged to share ideas, ask for feedback/help, and
collaborate harmoniously. Feedback should always be clear, concise, and
respectful. We do not necessitate a Code of Conduct, but if issues begin to
arise over the course of the project, we may consider standardizing one as part
of a conflict resolution plan.

\subsubsection*{Stay on Track}

We will track everything using GitHub and GitHub Projects. We have a Kanban
board to track attendance and tasks for deliverables as GitHub Issues, and other
metrics will be collected by the methods provided in the
\href{https://github.com/SumanyaG/Alkalytics/tree/main/docs/projMngmnt}{projMngmnt}
folder of the project repository (e.g., performance metrics).\newline

\noindent If uneven contribution becomes a pattern or a team member’s quality of
work degrades over time, the team will try to resolve the issue internally by
bringing it to that member’s attention. Uneven contribution may present itself
in the form of fewer commits or fewer tasks completed despite being assigned,
etc. If no improvement is seen after two internal discussions, the issue will be
escalated to our assigned Teaching Assistant (TA). If there is still no
improvement after the TA's involvement, it will be brought to the instructor's
attention.

\subsubsection*{Team Building}

We plan on building team cohesion through our interactions outside of the
Capstone project. As we all have similar class schedules, we have numerous
opportunities to interact and work with each other in environments that differ
from the Capstone project. 

\subsubsection*{Decision Making} 

Decisions are to be made through a majority vote and should only be made after
the team has had a thorough discussion. If there is a tie, we may approach our
supervisors for their input if it is relevant for them.

\newpage{}

\begin{thebibliography}{1}
  \bibitem{ref1} ``Senior Capstone Design Team Charter 2018'' [Online], Shiley
  School of Engineering, University of Portland, Oregon, 2018. Available:
  \url{https://engineering.up.edu/industry_partnerships/files/team-charter.pdf}
\end{thebibliography}

\end{document}
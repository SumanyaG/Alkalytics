\documentclass{article}

\usepackage{booktabs}
\usepackage{tabularx}

\title{Development Plan\\\progname}

\author{\authname}

\date{}

%% Comments

\usepackage{color}

\newif\ifcomments\commentstrue %displays comments
%\newif\ifcomments\commentsfalse %so that comments do not display

\ifcomments
\newcommand{\authornote}[3]{\textcolor{#1}{[#3 ---#2]}}
\newcommand{\todo}[1]{\textcolor{red}{[TODO: #1]}}
\else
\newcommand{\authornote}[3]{}
\newcommand{\todo}[1]{}
\fi

\newcommand{\wss}[1]{\authornote{blue}{SS}{#1}} 
\newcommand{\plt}[1]{\authornote{magenta}{TPLT}{#1}} %For explanation of the template
\newcommand{\an}[1]{\authornote{cyan}{Author}{#1}}

%% Common Parts

\newcommand{\progname}{Software Engineering} % PUT YOUR PROGRAM NAME HERE
\newcommand{\authname}{Team 21, Alkalytics
\\ Sumanya Gulati - gulats10
\\ Kate Min - mink9
\\ Jennifer Ye - yej52
\\ Jason Tran - tranj78} % AUTHOR NAMES                  

\usepackage{hyperref}
    \hypersetup{colorlinks=true, linkcolor=blue, citecolor=blue, filecolor=blue,
                urlcolor=blue, unicode=false}
    \urlstyle{same}
                                


\begin{document}

\maketitle

\begin{table}[hp]
\caption{Revision History} \label{TblRevisionHistory}
\begin{tabularx}{\textwidth}{llX}
\toprule
\textbf{Date} & \textbf{Developer(s)} & \textbf{Change}\\
\midrule
09-20-2024 & Kate Min & Added Team Meeting Plan\\
09-21-2024 & Kate Min & Added Team Communication Plan and Roles\\
09-22-2024 & Kate Min & Added Team Charter, License Information, and Introduction\\
... & ... & ...\\
\bottomrule
\end{tabularx}
\end{table}

\newpage{}

\noindent This document outlines Team Alkalytics' project development plan.\newline

\noindent Administrative details such as copyright and Intellectual
Property (IP) agreements, the team's meeting schedule, communication methods,
and expected roles are thoroughly covered.\newline

\noindent This is followed by details about the team's planned workflow, project
deliverables, scheduling, a Proof of Concept (POC) demonstration plan, and an
expected technology stack for the project and relevant coding standards.\newline

\noindent The appendices include a reflection on the writing process of this document as
well as a team charter that outlines a few ground rules and expectations for 
the team.

\section{Confidential Information?}

There is no confidential information to protect with respect to this project.

\section{IP to Protect}

There is IP to protect with respect to this project. Currently, the type of
agreement that the team may be required to sign is in discussion between the
primary supervisor, Dr.\ Charles de Lannoy, and the Capstone course instructor,
Dr.\ Spencer Smith. Once the agreement has been decided and signed, this section
will be updated to reflect the terms of the agreement.

\section{Copyright License}

The project will be adopting the license decided by Dr.\ de Lannoy and Dr.\ Smith. As this is
still in discussion, the project will temporarily adopt the standard MIT license, provided
\href{https://github.com/SumanyaG/Alkalytics/blob/main/LICENSE}{here}. Once the appropriate license for the project has been decided, this section and the linked
license file in the repository will be updated to reflect the proper license
agreement.

\section{Team Meeting Plan}
This section provides detail on the team's internal and external meeting plans.
\subsection{Weekly Meetings}
The team will have weekly check-ins on Fridays from 3:30PM to 4:30PM, either in-person 
on campus or virtually, depending on member's availabilities. Virtual meetings 
will be hosted on Microsoft Teams. Additional meetings will be scheduled as needed.\newline

\noindent The current meeting chair should open a GitHub issue prior to a meeting. During a 
meeting, the chair will go over each item on a prepared agenda and all members 
participate in the discussion. The notetaker will record meeting minutes and post 
them as a comment under the GitHub issue after the meeting, as well as a summary of
task assignments.

\subsection{Supervisor Meetings}
The team will have biweekly or monthly meetings with the primary supervisor, Dr.\ Charles
de Lannoy, depending on his availability. We will be using his Outlook calendar to
arrange appropriate meeting dates. These meetings will be held either in his
office or virtually on Microsoft Teams.\newline

\noindent Weekly meetings with the secondary supervisor, Bassel Abdelkader will be on Tuesdays from 3:30PM to 4:30PM, as he works closely with the data 
we need to integrate into the project. These meetings will be virtual due
to his schedule.

\section{Team Communication Plan}

The team uses an Instagram group chat for regular communication and a Microsoft Teams group chat for drafting emails and sharing resources.\newline

\noindent Communication with supervisors will take place through email or scheduled meetings.
The team liaison will be responsible for sending the emails and should Carbon Copy (CC) all team members.\newline

\noindent All project-related communication will be done via GitHub issues. Each issue will track a specific task and assign team members. Every pull request should link to its 
corresponding issue(s), and feedback is expected from all team members when a pull request is created.

\section{Team Member Roles}

The team plans to rotate through the following roles between every stage, defined
in Table 2:

\begin{itemize}
  \item \textbf{Meeting chair}: The meeting chair should prepare an agenda for each
  meeting and open an issue for the meeting. The issue should contain an attendance
  tracker, the agenda, and other topics that may need to be discussed.
  \item \textbf{Notetaker}: The notetaker will record meeting minutes during a 
  meeting and post it as a comment under the meeting issue, along with a summary of 
  task assignments.
  \item \textbf{Primary reviewer for pull requests}: The primary reviewer will give
  the official ``approval'' for pull requests. All members should review pull requests
  and provide feedback, but the primary reviewer will make the final decision to
  ensure a smooth workflow.
\end{itemize}

\noindent Once implementation begins, all members are expected to contribute as developers.
However, the team anticipates appointing technical leads based on the stage of development and the members' skillsets, 
described below:

\begin{itemize}
  \item Sumanya has experience in data analytics and working with large datasets,
  and will also serve as the main point of communication between the team and the project's
  stakeholders.
  \item Kate has experience in web development and programming in Python, as well
  as proficiency in several Python libraries that may need to be used.
  \item Jason has experience in MongoDB and machine learning, as well as proficiency
  in full-stack development.
  \item Jennifer has experience working with large datasets and employing data
  visualization tools like PowerBI, along with a strong background in web development.
\end{itemize}

\section{Workflow Plan}

\begin{itemize}
	\item How will you be using git, including branches, pull request, etc.?
	\item How will you be managing issues, including template issues, issue
	classification, etc.?
  \item Use of CI/CD
\end{itemize}

\section{Project Decomposition and Scheduling}

\begin{itemize}
  \item How will you be using GitHub projects?
  \item Include a link to your GitHub project
\end{itemize}

\wss{How will the project be scheduled?  This is the big picture schedule, not
details. You will need to reproduce information that is in the course outline
for deadlines.}

\section{Proof of Concept Demonstration Plan}

What is the main risk, or risks, for the success of your project?  What will you
demonstrate during your proof of concept demonstration to convince yourself that
you will be able to overcome this risk?

\section{Expected Technology}

\wss{What programming language or languages do you expect to use?  What external
libraries?  What frameworks?  What technologies.  Are there major components of
the implementation that you expect you will implement, despite the existence of
libraries that provide the required functionality.  For projects with machine
learning, will you use pre-trained models, or be training your own model?  }

\wss{The implementation decisions can, and likely will, change over the course
of the project.  The initial documentation should be written in an abstract way;
it should be agnostic of the implementation choices, unless the implementation
choices are project constraints.  However, recording our initial thoughts on
implementation helps understand the challenge level and feasibility of a
project.  It may also help with early identification of areas where project
members will need to augment their training.}

Topics to discuss include the following:

\begin{itemize}
\item Specific programming language
\item Specific libraries
\item Pre-trained models
\item Specific linter tool (if appropriate)
\item Specific unit testing framework
\item Investigation of code coverage measuring tools
\item Specific plans for Continuous Integration (CI), or an explanation that CI
  is not being done
\item Specific performance measuring tools (like Valgrind), if
  appropriate
\item Tools you will likely be using?
\end{itemize}

\wss{git, GitHub and GitHub projects should be part of your technology.}

\section{Coding Standard}

\wss{What coding standard will you adopt?}

\newpage{}

\section*{Appendix --- Reflection}

\wss{Not required for CAS 741}

The purpose of reflection questions is to give you a chance to assess your own
learning and that of your group as a whole, and to find ways to improve in the
future. Reflection is an important part of the learning process.  Reflection is
also an essential component of a successful software development process.  

Reflections are most interesting and useful when they're honest, even if the
stories they tell are imperfect. You will be marked based on your depth of
thought and analysis, and not based on the content of the reflections
themselves. Thus, for full marks we encourage you to answer openly and honestly
and to avoid simply writing ``what you think the evaluator wants to hear.''

Please answer the following questions.  Some questions can be answered on the
team level, but where appropriate, each team member should write their own
response:


\begin{enumerate}
    \item Why is it important to create a development plan prior to starting the
    project?
    \item In your opinion, what are the advantages and disadvantages of using
    CI/CD?
    \item What disagreements did your group have in this deliverable, if any,
    and how did you resolve them?
\end{enumerate}

\newpage{}

\section*{Appendix --- Team Charter \cite{ref1}}

\subsection*{External Goals}

By the end of Capstone, our team wishes to deliver a project that is impactful
and useful beyond graduation. We seek the personal satisfaction that comes from
knowing the product could contribute to improvements in climate change. By achieving 
these goals, we thereby anticipate and hope for an excellent grade in the Capstone
course.

\subsection*{Attendance}

This section lays out ground rules and expectations regarding team member attendance.

\subsubsection*{Expectations}

All members are expected to attend all weekly check-in meetings, meetings with our
supervisors, and maintain punctuality. We will not require full attendance at every
Capstone lecture or tutorial, as long as at least one member from the team attends.
All forms of attendance will be logged through GitHub Issues.

\subsubsection*{Acceptable Excuse}

It will not be acceptable for a member to miss a meeting or a deadline without any
prior communication. Missing a few meetings will be acceptable as long as the member
communicates their absence to the rest of the team at least 1-2 hours in advance.
There will be no acceptable excuses for missing a deadline unless they have an 
emergency. The guidelines for such a situation are outlined in the following section.

\subsubsection*{In Case of Emergency}

If a team member has an emergency and cannot finish their work or meet a deadline,
they are expected to communicate this as early as possible. The rest of the team will
discuss potential accommodations and ensure that all tasks are completed by the
deadline.

\subsection*{Accountability and Teamwork}

This section covers expectations for the team's collaboration efforts.

\subsubsection*{Quality} 

All members should prepare for weekly check-ins and meetings by doing the following:

\begin{itemize}
  \item[(a)] Review the agenda prepared by the meeting chair in advance, and
  \item[(b)] Prepare a short progress report and any questions/concerns that were not
  already mentioned in the agenda.
\end{itemize}

\noindent Each member is expected to put in consistent effort and deliver high-quality work
that meets or exceeds the expectations described in the applicable rubric. Code
should be well-structured, well-documented, and adhere to the coding standards
outlined in Section 11 of this document.

\subsubsection*{Attitude}

All members are encouraged to share ideas, ask for feedback/help, and collaborate
harmoniously. Feedback should always be clear, concise, and respectful. We do not
necessitate a Code of Conduct, but if issues begin to arise over the course of the
project, we may consider standardizing one as part of a conflict resolution plan.

\subsubsection*{Stay on Track}

We will track everything using GitHub and GitHub Projects. We have a Kanban board to
track attendance and tasks for deliverables as GitHub Issues, and other
metrics will be collected by the methods provided in the \href{https://github.com/SumanyaG/Alkalytics/tree/main/docs/projMngmnt}{projMngmnt}
folder of the project repository (e.g., performance metrics).\newline

\noindent If uneven contribution becomes a pattern or a team member’s quality of work degrades
over time, the team will try to resolve the issue internally by bringing it to that
member’s attention. Uneven contribution may present itself in the form of fewer
commits or fewer tasks completed despite being assigned, etc. If no improvement is
seen after two internal discussions, the issue will be escalated to our assigned
Teaching Assistant (TA). If there is still no improvement after the TA's involvement,
it will be brought to the instructor's attention.

\subsubsection*{Team Building}

We plan on building team cohesion through our interactions outside of the Capstone
project. As we all have similar class schedules, we have numerous opportunities to
interact and work with each other in environments that differ from the Capstone
project. 

\subsubsection*{Decision Making} 

Decisions are to be made through a majority vote and should only be made after the
team has had a thorough discussion. If there is a tie, we may approach our
supervisors for their input if it is relevant for them.

\newpage{}

\begin{thebibliography}{1}
  \bibitem{ref1} "Senior Capstone Design Team Charter 2018" [Online], Shiley School of Engineering, University of Portland, Oregon, 2018. Available: \url{https://engineering.up.edu/industry_partnerships/files/team-charter.pdf}
\end{thebibliography}

\end{document}
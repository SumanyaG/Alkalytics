\documentclass[12pt, titlepage]{article}

\usepackage[parfill]{parskip}
\usepackage{float}
\usepackage{booktabs}
\usepackage{tabularx}
\usepackage{hyperref}
\hypersetup{ colorlinks, citecolor=black, filecolor=black, linkcolor=red,
    urlcolor=blue }
\usepackage[round]{natbib}

%% Comments

\usepackage{color}

\newif\ifcomments\commentstrue %displays comments
%\newif\ifcomments\commentsfalse %so that comments do not display

\ifcomments
\newcommand{\authornote}[3]{\textcolor{#1}{[#3 ---#2]}}
\newcommand{\todo}[1]{\textcolor{red}{[TODO: #1]}}
\else
\newcommand{\authornote}[3]{}
\newcommand{\todo}[1]{}
\fi

\newcommand{\wss}[1]{\authornote{blue}{SS}{#1}} 
\newcommand{\plt}[1]{\authornote{magenta}{TPLT}{#1}} %For explanation of the template
\newcommand{\an}[1]{\authornote{cyan}{Author}{#1}}

%% Common Parts

\newcommand{\progname}{Software Engineering} % PUT YOUR PROGRAM NAME HERE
\newcommand{\authname}{Team 21, Alkalytics
\\ Sumanya Gulati - gulats10
\\ Kate Min - mink9
\\ Jennifer Ye - yej52
\\ Jason Tran - tranj78} % AUTHOR NAMES                  

\usepackage{hyperref}
    \hypersetup{colorlinks=true, linkcolor=blue, citecolor=blue, filecolor=blue,
                urlcolor=blue, unicode=false}
    \urlstyle{same}
                                


\begin{document}

\title{Verification and Validation Report: \progname} 
\author{\authname}
\date{\today}
	
\maketitle

\pagenumbering{roman}

\section{Revision History}

\begin{tabularx}{\textwidth}{p{3cm}p{2cm}X} \toprule {\bf Date} & {\bf Version}
& {\bf Notes}\\
\midrule
4 March 2025 & 1.0 & Initial Draft\\
10 March 2025 & 1.1 & Add Results\\
\bottomrule
\end{tabularx}

~\newpage

\section{Symbols, Abbreviations and Acronyms}

\renewcommand{\arraystretch}{1.2}
\begin{tabular}{l l} 
  \toprule		
  \textbf{symbol} & \textbf{description}\\
  \midrule
  CSV & Comma-Separated Values\\
  FR & Functional Requirement\\
  LFR & Look and Feel Requirement\\
  MG & Module Guide\\
  MSR & Maintainability and Support Requirement\\
  NFR & Nonfunctional Requirement\\
  OER & Operational and Environmental Requirement\\
  PR & Performance Requirement\\
  SR & Security Requirement\\
  SRS & Software Requirements Specification\\
  ST & System Test\\
  UHR & Usability and Humanity Requirement\\
  VnV & Verification and Validation\\
  WCAG & Web Content Accessibility Guidelines\\
  \bottomrule
\end{tabular}\\

\newpage

\tableofcontents

\listoftables %if appropriate

\newpage

\pagenumbering{arabic}

\section{Overview}
This document provides a comprehensive summary of the Verification and
Validation (VnV) process for the Alkalytics project. It includes reports and
analysis of system testing and unit testing results, along with traceability
between test cases, the software modules in the
\href{https://github.com/SumanyaG/Alkalytics/blob/main/docs/Design/MG.pdf}{Module
Guide} (MG), and the requirements outlined in the
\href{https://github.com/SumanyaG/Alkalytics/blob/main/docs/SRS/SRS.pdf}{Software
Requirements Specification} (SRS). Tests and test procedures are derived from the \href{https://github.com/SumanyaG/Alkalytics/blob/main/docs/VnVPlan/VnVPlan.pdf}{VnV Plan} document. Additionally, it highlights changes made to the
implementation based on testing outcomes and user feedback.

\section{Functional Requirements Evaluation}
This section outlines the manual system tests that were run and their sub-
sequent results and outputs for all listed functional requirements.
\subsection{Data Input and Storage}
\begin{enumerate}
  \item{\textbf{FR-ST1}} \label{FR:ST1}

Control: Manual

Initial State: Database is running and ready to intake data

Input: \texttt{SAMPLE\_.CSV\_FILE}

\begin{footnotesize}
  *Note: Actual experiment data cannot be provided publicly.
\end{footnotesize}

Expected Output: The inputted data is stored in the system with proper labels
and without any data loss or errors

Actual Output: Data successfully stored in the database with correct key-value
pairs; no data loss or errors were detected.

Result: \textcolor{green}{Pass}
%if fail, explain.

\item{\textbf{FR-ST1.1}} \label{FR:ST1.1}

Control: Manual

Initial State: Database is running and ready to intake data

Input: \texttt{SAMPLE\_DATA\_POINT}: \texttt{\{NEW\_EXPERIMENT\_ROW\}}

\begin{footnotesize}
  *Note: Actual experiment data cannot be provided publicly.
\end{footnotesize}

Expected Output: The inputted data is stored in the system with proper labels
and without any data loss or errors.

Actual Output: The inputted data is stored in the system under the correct label
and without any data loss or errors.

Result: \textcolor{green}{Pass}
%if fail, explain actual output.
\end{enumerate}

\subsection{Data Querying and Results}
\begin{enumerate}
\item{\textbf{FR-ST2}} \label{FR:ST2}

Control: Manual

Initial State: The system is not running any jobs, and the user interface is
cleared.

Input: \texttt{SAMPLE\_PARAMETERS}: \{Graph: Line, Experiment Dates: 2024-11-05,
Parameter Type: Data Sheet, X Parameter: C1 Cond, Y Parameter: C1 Temp, X Axis
Label: C1 Cond, Y Axis Label: C1 Temp, Graph Title: 2024-11-05 C1 Cond vs C1
Temp \}

Expected Output: A human-readable and customizable visualization of correct
results corresponding to the selected parameters from the input

Actual Output: Line graph visual displaying the data points from the data sheet
with the axis custom labeled and graph title. A line of best fit is also
displayed on the graph.

Result: \textcolor{green}{Pass}
%if fail, explain actual output.
\end{enumerate}

\subsection{Data Analysis}
\begin{enumerate}
\item{\textbf{FR-ST3}} \label{FR:ST3}

Control: Manual

Initial State: The application's cleared user interface which has not yet been
used to query data, with no graph showing yet.

Input: \texttt{SAMPLE\_PARAMETERS}: \{Graph: Scatter, Experiment Dates: 2024-09-26,
Parameter Type: Data Sheet, X Parameter: C1 pH, Y Parameter: C1 Temp, X Axis
Label: C1 pH, Y Axis Label: C1 Temp\}

Expected Output: A small written human-readable paragraph explaining the input
data.

Actual Output: A linear regression line is plotted with the resulting graph
output, along with information about how well the regression line fits with the
data.

Result: Neutral. The actual output does not match the original expected output,
however, the linear regression line provides a clear trend analysis and has been
determined to be more useful than a text-based analysis.
%if fail, explain actual output.
\end{enumerate}

\subsection{Data Hygiene}
\begin{enumerate}
  \item{\textbf{FR-ST4}}; Stretch Goal Requirement \label{FR:ST4}

Control: Manual

Initial State: The application's cleared user interface which has not yet been
used to query data, with no graph showing yet.

Input: \texttt{SAMPLE\_.CSV\_FILE}

Expected Output: A log file documenting errors found in the input data and/or
removals of missing data.

Actual Output: Feature not implemented; as this is a stretch goal, the
functionality is currently not available.

Result: \textcolor{red}{Fail}

\end{enumerate}

\subsection{User Access}
\begin{enumerate}
\item{\textbf{FR-ST5}} \label{FR:ST5}

Control: Manual

Initial State: User interface shows a login page, with no login credentials
currently used

Input: \texttt{SAMPLE\_VALID\_EMAIL, SAMPLE\_VALID\_PASSWORD}
                        
Expected Output: The page redirects to the page designated after login

Actual Output: Once credentials are submitted, the page redirects to the main
home page of the application.

Result: \textcolor{green}{Pass}
\end{enumerate}

\subsection{Data Export}
\begin{enumerate}
  \item{\textbf{FR-ST6}}; Stretch Goal Requirement \label{FR:ST6}

Control: Manual

Initial State: User interface after multiple usages of data queries

Input: \texttt{USER ACTION => BUTTON CLICK}

Expected Output: Query report will be downloaded to the user's device

Actual Output: Feature not implemented; as this is a stretch goal, the
functionality is currently not available.

Result: \textcolor{red}{Fail}
\end{enumerate}

\section{Nonfunctional Requirements Evaluation}
This section outlines the manual system tests that were run and their sub-
sequent results and outputs for all listed non-functional requirements.
\subsection{Look and Feel} \label{NFR-LF}
\begin{enumerate}
\item{\textbf{NFR-LF1}} \label{NFR:LF1}

Type: User Demo, Manual

Initial State: Fully functional application ready for user interaction, starting
at the login page

Input/Condition: User engagement with application

Expected Output: 90\% of responses in the Navigation and Ease of Use section of
the usability survey are at least 'Neither easy nor difficult'; 85\% of users
navigate the entire application within 10 minutes.

Actual Output: \newline
The following answers are based on the Usability Survey provided in the \href{https://github.com/SumanyaG/Alkalytics/blob/main/docs/VnVPlan/VnVPlan.pdf}
{Verification and Validation Plan}. The test was conducted on 2 users.
\begin{itemize}
  \item Navigation and Ease of Use
  \begin{enumerate}
    \item \textbf{User 1:} \emph{Very Easy}, \textbf{User 2:} \emph{Somewhat Easy}
    \item \textbf{User 1:} \emph{Somewhat Easy}, \textbf{User 2:} \emph{Somewhat Easy}
  \end{enumerate}
  Calculated percentage of user responses that are at least `Neither easy nor difficult' = 100\%.
  \item Time to navigate the application end to end: \textbf{User 1:} 7.63 minutes, \textbf{User 2:} 8.5 minutes.
  
  Calculated percentage of users that are able to navigate the application in under 10 minutes = 100\%.
\end{itemize}

Result: \textcolor{green}{Pass}

\item{\textbf{NFR-LF2}} \label{NFR:LF2}

Type: Manual

Initial State: Application running on a 15" laptop, 24" monitor, standard smart
phone and/or tablet

Input/Condition: Manual tester's engagement with application

Expected Output: Number of inconsistencies found across all devices $\leq$ 10.

Actual Output: \newline
\textbf{Dashbaord}: 2 inconsistancies on mobile, 1 inconsistancy on tablet. Not
full screen cuts off half way on both tablet and mobile. Table quick view is not
mobile compatible as the table is squished and not readable.\newline
\textbf{Query Page}: 4 inconsistancies on mobile, 1 inconsistancy on tablet. The
table is not horizontally scrollable on mobile view as such cannot see the
search bar and column drop down. Also cannot type in the equations field to
apply them to the table. On tablet view there is abit of the table that is cut
off on the right. \newline
\textbf{Upload Page}: No inconsistancies.\newline
\textbf{Graphs Page}: 1 inconsistancy on mobile. The modal is cut off on the
right. Although it does not affect the functionality as all the buttons are
still reachable, it is not visually apealing. \newline
\textbf{Login Page}: 0 inconsistancies.

Result: \textcolor{green}{Pass} \newline
Although there are still inconsistancy, they are
mainly due to the fact that mobile compatibilityis a nice to have. It is
expected that almost always users will be using the app on a standard sized
laptop or monitor. Hence, why there are only inconsistancies present in tablet
and mobile views and not the laptop or monitor views. \end{enumerate}

\subsection{Usability}
\begin{enumerate}
\item{\textbf{NFR-UH1}} \label{NFR:UH1}

Type: User Demo, Manual

Initial State: Fully functional application ready for user interaction, starting
at the login page

Input/Condition: User engagement with application, set of tasks to complete

Expected Output: 85\% of users ask for help no more than 3 times; 90\% of
responses to the Learning section of the usability survey are at least 'Neither
easy nor difficult'.

Actual Output: \newline
The answers for the Usability Survey have been provided in test output for test \ref{NFR:LF1} in section \ref{NFR-LF}.

The users did not have to ask the usability test conductor for help at all. With that said, however, it must be noted 
that the users had been walked through a virtual demo of the application a couple of weeks prior to the usability testing.

Result: \textcolor{green}{Pass}

\item{\textbf{NFR-UH2}} \label{NFR:UH2}

Type: Manual
  
Initial State: Fully developed application, starting at the login page
  
Input/Condition: Manual tester using application

Expected Output: 0 language discrepancies found in the application.
  
Actual Output: 0 language discrepancies found in the application.

Result: \textcolor{green}{Pass}
%if fail, explain why output doesn't meet expectations for pass.

\item{\textbf{NFR-UH3}} \label{NFR:UH3}

Type: Manual

Initial State: Application ready to take in data

Input/Condition: \texttt{SAMPLE\_.CSV\_FILE}

Expected Output: Error logs from unrecognized characters, or successful upload

Actual Output: The upload csv will still upload the unrecognized symbols without
any issues. The type will just be a string. As a result, there is nothing to log. 

Result: \textcolor{green}{Pass} 
%if fail, explain why output doesn't meet expectations for pass.

\item{\textbf{NFR-UH4}} \label{NFR:UH4}

Type: Manual

Initial State: Application ready for use

Input/Condition: Tester engagement using third party tools

Expected Output: List of accessibility issues in accordance to Web Content
Accessibility Guidelines (WCAG), and checklists for if page is screen-readable

Actual Output:\newline
\textbf{Dashbaord}: All buttons and form fields are do not have alt texts or
context. \newline
\textbf{Query Page}:All buttons and form fields are do not have alt texts or
context. Scrolling is also not possible from keyboard shortcuts. \newline
\textbf{Upload Page}: Upload button does not have alt text.\newline
\textbf{Graphs Page}: All buttons and form fields are do not have alt texts or
context.\newline
\textbf{Login Page}: Is accessible

Result: \textcolor{red}{Fail} \newline
Most pages do not comply with the WCAG guidelines. Many of the components and
buttons are missing alt texts which makes the page screen readable. 
 \end{enumerate}

\subsection{Performance}
\begin{enumerate}
  \item{\textbf{NFR-P1}} \label{NFR:P1}

Type: Manual

Initial State: Application navigated to upload page, ready to upload file

Input/Condition: \texttt{SAMPLE\_.CSV\_FILE}

Expected Output: Average upload duration $\leq$ \texttt{UPLOAD\_TIME}

\texttt{UPLOAD\_TIME} = 60 seconds

Actual Output: File uploaded in 48 seconds on average, satisfying the
performance requirements.

Result: \textcolor{green}{Pass}

  \item{\textbf{NFR-P2}} \label{NFR:P2}

Type: Manual

Initial State: Application navigated to querying page, ready to make query

Input/Condition: \texttt{SAMPLE\_QUERY}: \{\}

Expected Output: Average query response duration $\leq$ \texttt{QUERY\_TIME}
where \texttt{QUERY\_TIME} = 3 seconds

Actual Output: Query response time averaged less than 0.1, satisfying the
performance requirement.

Result: \textcolor{green}{Pass}

\item{\textbf{NFR-P3}} \label{NFR:P3}

Type: Manual

Initial State: Application ready to use

Input/Condition: Tester engagement

Expected Output: Average response time of buttons on website $\leq$
\texttt{RESPONSE\_TIME}

\texttt{RESPONSE\_TIME} = 2 seconds

Actual Output: \newline
\textbf{Dashbaord}: Dropdowns have a response time of 0.1 seconds \newline
\textbf{Query Page}:All buttons and form fields are do not have alt texts or
context. Scrolling is also not possible from keyboard shortcuts. \newline
\textbf{Upload Page}: Upload button takes about 2 seconds if there is an error
and on average 1.75 seconds for a successful upload. It also takes 0.1 seconds
to being uploading. \newline
\textbf{Graphs Page}: Graph, parameter type drop downs have a response time of
0.1. Date, x and y parameter drop downs has an average response time of 1.25
seconds \newline
\textbf{Login Page}: Login and register button have a reponse time of 0.1 seconds

Result: \textcolor{green}{Pass} 

\item{\textbf{NFR-P4}} \label{NFR:P4}

Type: Manual

Initial State: Application navigated to graphs page, ready to make query to
generate graphs

Input/Condition: \texttt{SAMPLE\_QUERY}: \{\}
  
Expected Output: Average time taken to generate graphs/visualizations $\leq$
\texttt{GRAPH\_TIME}

\texttt{GRAPH\_TIME}: 10 seconds

Actual Output:
\begin{itemize}
  \item Test Execution 1: 8.2 seconds
  \item Test Execution 2: 9.1 seconds
  \item Test Execution 3: 7.9 seconds
  \item Test Execution 4: 8.7 seconds
  \item Test Execution 5: 9.5 seconds
\end{itemize}
Average Time Taken = 8.7 seconds

Result: \textcolor{green}{Pass}

\item{\textbf{NFR-P5}} \label{NFR:P5}

Type: Manual

Initial State: Application navigated to query page, ready to make query

Input/Condition: \texttt{SAMPLE\_QUERY : \{NaCl $\ge$ 1.0\}}

Expected Output: A check for precision of numbers in different components of
system. Values should be accurate to 4 decimal places.

Actual Output: Parameter values displayed with an accuracy matching that of the
original spreadsheet data. All parameters exceeded the accuracy of 4 decimal
places, satisfying the requirement.

Result: \textcolor{green}{Pass}

\item{\textbf{NFR-P6}} \label{NFR:P6}

Type: Manual

Initial State: Application running as normal

Input/Condition: Tester temporarily taking down back end

Expected Output: Error messages properly displayed

Actual Output: Displayed appropriate error messages for all components
indicating that data could not be retrieved from backend services.

Result: \textcolor{green}{Pass}

\item{\textbf{NFR-P7}} \label{NFR:P7}

Type: Manual

Initial State: Application running as normal

Input/Condition: Tester temporarily disconnects internet connection

Expected Output: Previously generated plots and previous queries still working

Actual Output:
\begin{itemize}
  \item Previously generated plots remain visible and interactive.
  \item Queries executed before disconnection are still accessible and functional.
  \item No errors or crashes observed during the disconnection period.
\end{itemize}

Result: \textcolor{green}{Pass}

\item{\textbf{NFR-P8}} \label{NFR:P8}

Type: Manual

Initial State: Three devices ready to run application

Input/Condition: \texttt{SAMPLE\_QUERY}: \{\}

Expected Output: System response time while under load $\leq$
\texttt{RESPONSE\_TIME\_THRESHOLD}

Actual Output: \newline
\texttt{RESPONSE\_TIME\_THRESHOLD} = 3 seconds
\begin{itemize}
  \item Device 1 Response Time: 1.8 seconds
  \item Device 2 Response Time: 1.9 seconds
  \item Device 1 Response Time: 1.7 seconds
\end{itemize}
Average Response Time Under Load = 1.8 seconds

Result: \textcolor{green}{Pass}

\item{\textbf{NFR-P9}} \label{NFR:P9}

Control: Data Generation, Automated

Initial State: A database with a known amount of experiment data

Input/Condition: \texttt{DUMMY\_DATA}

Expected Output: Observations on system health after large payload

Actual Output: Large payload are processed with slight performance degradation,
but still satisfying performance requirements.

Result: \textcolor{green}{Pass}

\end{enumerate}

\subsection{Operational and Environmental}
\begin{enumerate}
\item{\textbf{NFR-OE1}} \label{NFR:OE1}

Type: Manual

Initial State: Application running on a windows device as a web application on a
Chromium based browser.

Input/Condition: Tester engagement

Expected Output: A list of all discovered issues with the application that arise
due to environment compatibility

Actual Output: No issues discovered when using Chromium based browser. The experience is same as
the expected.

Result: \textcolor{green}{Pass} 
            
\item\textbf{{NFR-OE2}} \label{NFR:OE2}

Type: Manual, User Demo

Initial State: Application running and ready for use on home screen

Input/Condition: User engagement

Expected Output: Survey results depicting subjective complexity of onboarding
process

Actual Output: \newline
Based on the usability testing results provided in section \ref{NFR-LF} for test \ref{NFR:LF1}, as a 100\% 
of the users found the navigation process to be at least `somewhat easy', it can be concluded that the onboarding
process is intuitive and user-friendly.

Result: \textcolor{green}{Pass}
\end{enumerate}

\subsection{Maintainability and Support}
\begin{enumerate}
\item{\textbf{NFR-MS1}} \label{NFR:MS1}

Type: Manual, User Demo

Initial State: Application running, ready for use

Input/Condition: User engagement

Expected Output: Observations on user's ability to complete tasks without
support

Actual Output: \newline
Observations from Usability Testing 
\begin{enumerate}
  \item \textbf{User 1:}
  \begin{itemize}
    \item Successfully completed all assigned tasks such as navigating to the graphs 
    page, generating a graph etc. without any assistance.
    \item Did not ask for help or clarification during the session.
    \item Completed the tasks in 7.63 minutes.
  \end{itemize}
  \item \textbf{User 2:}
  \begin{itemize}
    \item Successfully completed all assigned tasks such as navigating to the graphs 
    page, generating a graph etc. without any assistance.
    \item Did not ask for help or clarification during the session.
    \item Completed the tasks in 8.5 minutes.
  \end{itemize}
\end{enumerate}
Percentage of users who completed tasks without support = 100\%. 
Average Time to complete tasks = 8.07 minutes.

Result: \textcolor{green}{Pass}
					
\item{\textbf{NFR-MS2}} \label{NFR:MS2}

Type: Manual

Initial State: Multiple Chromium-based web browsers open

Input/Condition: Tester Engagement

Expected Output: All abnormal behaviour of web pages observed on each different
web browser

Actual Output: The application ran on Firefox, Chrome and Microsoft Edge. There
was no abnormal behaviour across the browsers running the app. 

Result: \textcolor{green}{Pass}
\end{enumerate}

\subsection{Security}
\begin{enumerate}
\item{\textbf{NFR-SR1}} \label{NFR:SR1}

Type: Manual

Initial State: Application login page is displayed.

Input/Condition: \texttt{SAMPLE\_VALID\_CREDENTIALS,
SAMPLE\_INVALID\_CREDENTIALS}

Expected Output: Access is granted or denied based on the validity of the
credentials provided.

Actual Output: Access is granted for the \texttt{SAMPLE\_VALID\_CREDENTIALS},
which consisted of a registered email and correct password, and consistently
denied for the \texttt{SAMPLE\_INVALID\_CREDENTIALS}, which consisted of the
following test inputs:
  \begin{itemize}
    \item Registered email, incorrect password
    \item Unregistered email, correct password
    \item Unregistered email, incorrect password
  \end{itemize}

Result: \textcolor{green}{Pass}

\item{\textbf{NFR-SR2}} \label{NFR:SR2}

Type: Manual

Initial State: Application logged in with multiple user roles (e.g., admin,
researcher, research assistant).

Input/Condition: Tester interacts with the application using on different user
roles.
  
Expected Output: Access to query or modify data and perform sensitive operations
is restricted according to user roles.

Actual Output: Only admin and researchers were able to modify data and these
operations were not available to users with the role of research assistant.

Result: \textcolor{green}{Pass}

\item{\textbf{NFR-SR3}} \label{NFR:S3}

Type: Manual

Initial State: User is logged into the application.

Input/Condition: User remains inactive for a specified period.

Expected Output: User is automatically logged out after a predefined period of
inactivity.

Actual Output: The user was not logged out after 15 minutes of inactivity.

Result: \textcolor{red}{Fail}
\item{\textbf{NFR-SR4}} \label{NFR:SR4}

Type: Manual

Initial State: Application is open, ready for data entry and CSV upload.

Input/Condition: \texttt{VALID\_DATA, INVALID\_DATA, CORRUPTED\_DATA}

Expected Output: All invalid inputs and CSV uploads are rejected, and only valid
data entries are processed.

Actual Output: Valid data entries were processed correctly. All invalid inputs
were rejected with a visual cue, and/or error message.

Result: \textcolor{green}{Pass}

  \item{\textbf{NFR-SR5}} \label{NFR:SR5}

Type: Manual

Initial State: Application database contains a set of unique, validated records.
Application interface is open for data entry, processing, and transfer actions.

Input/Condition: \texttt{SAMPLE\_DUPLICATE\_RECORD}

Expected Output: Duplicate records are detected and prevented, and data accuracy
is maintained during all transfer operations.

Actual Output: Duplicate records are detected and filtered out through indexing, thus data accuracy maintained
throughout the transfer process.

Result: \textcolor{green}{Pass}

\item{\textbf{NFR-SR6}}; Out of Scope \label{NFR:SR6}

Type: Manual/Automated

Initial State: Database is operational, with storage capacity at or below normal
usage. Alert system is configured, and the administrator contact information is
set up to receive notifications. 

Input/Condition: Tester simulates increasing database storage usage to exceed
the \texttt{STORAGE\_THRESHOLD}.

Expected Output: System successfully detects when storage usage exceeds
\texttt{STORAGE\_THRESHOLD} and sends a timely alert to administrators.

Actual Output: Feature not implemented; notification system is considered a
stretch goal.

Result: \textcolor{red}{Fail}

\item{\textbf{NFR-SR7}}; Out of Scope \label{NFR:SR7}

Type: Manual/Automated

Initial State: The application is operational, with logging features configured
and permissions for accessing logs assigned to administrators only.

Input/Condition: Tester performs various access and modification actions within
the application, then attempts to access the audit logs with both authorized and
unauthorized user accounts.
  
Expected Output: All actions are logged with timestamps and user identities, and
audit logs are accessible only to authorized users, with proper encryption.

Actual Output: Feature not implemented; audit system is considered a stretch
goal.

Result: \textcolor{red}{Fail}

\item{\textbf{NFR-SR8}}; Out of Scope \label{NFR:SR8}

Type: Manual

Initial State: Application login screen is open. Administrator contact
information is configured to receive security alerts.

Input/Condition: Tester attempts multiple failed logins to trigger the
suspicious activity detection mechanism.

Expected Output:  Application detects and blocks access after three failed
attempts, sends an alert to administrators, and locks out the user temporarily.

Actual Output: Application detects failed login attempts but does not send
alerts. Notification system is considered a stretch goal.

Result: \textcolor{red}{Fail}

\item{\textbf{NFR-SR9}}; Out of Scope \label{NFR:SR9}

Type: Automated/Manual

Initial State: Application is running under typical workload conditions.
Monitoring tools and alerts for CPU and memory usage are enabled.

Input/Condition: Tester simulates a high workload to approach system resource
limits, observing the system's monitoring and optimization response.
  
Expected Output: The system actively manages CPU and memory usage, preventing
overload and maintaining performance stability.

Actual Output: Not conducted due to resource constraints and complexity in
simulating high workload conditions in the current testing environment.

Result: \textcolor{red}{Fail}

\end{enumerate}

\section{Unit Testing}

The following subsections outline the unit tests relevant to each module. Each
subsection includes a table or table(s) with a summary of the results for each test,
where each test includes a link to its corresponding test file in the project repository.

\subsection{Data Storage Module}
Table \ref{UT:DataStorage} provides a summary of the results for each test that is
relevant to the User Management module.
\begin{table}[H]
  \centering
  \begin{tabular}{|l|l|l|}
    \hline
    \textbf{Test ID} & \textbf{Test Name} & \textbf{Result}\\
    \hline
    6.3-UT1 & \href{https://github.com/SumanyaG/Alkalytics/blob/main/src/backend/test/migrationServiceTest.py}{Uploads experiments and data} & Pass \\
    \hline
    6.3-UT2 & \href{https://github.com/SumanyaG/Alkalytics/blob/main/src/backend/test/migrationServiceTest.py}{Links data to experiments} & Pass \\
    \hline
    6.3-UT3 & \href{https://github.com/SumanyaG/Alkalytics/blob/main/src/backend/test/migrationServiceTest.py}{Uploads data when experiment exists} & Pass \\
    \hline
    6.3-UT4 & \href{https://github.com/SumanyaG/Alkalytics/blob/main/src/backend/test/migrationServiceTest.py}{Skips data upload if no experiment exists} & Pass \\
    \hline
    6.1-UT5 & \href{https://github.com/SumanyaG/Alkalytics/blob/main/src/backend/test/migrationServiceTest.py}{Cleans data by removing empty values} & Pass \\
    \hline
    6.1-UT6 & \href{https://github.com/SumanyaG/Alkalytics/blob/main/src/backend/test/migrationServiceTest.py}{Checks for duplicate experiments} & Pass \\
    \hline
    6.1-UT7 & \href{https://github.com/SumanyaG/Alkalytics/blob/main/src/backend/test/migrationServiceTest.py}{Handles multiple matching experiments} & Pass \\
    \hline
    6.1-UT8 & \href{https://github.com/SumanyaG/Alkalytics/blob/main/src/backend/test/migrationServiceTest.py}{Handles no matching experiments} & Pass \\
    \hline
    6.3-UT9 & \href{https://github.com/SumanyaG/Alkalytics/blob/main/src/backend/test/migrationServiceTest.py}{Runs full migration process} & Pass \\
    \hline
    6.3-UT10 & \href{https://github.com/SumanyaG/Alkalytics/blob/main/src/backend/test/migrationServiceTest.py}{Handles experiment upload errors} & Pass \\
    \hline
    6.3-UT11 & \href{https://github.com/SumanyaG/Alkalytics/blob/main/src/backend/test/migrationServiceTest.py}{Handles data upload errors} & Pass \\
    \hline
    6.3-UT12 & \href{https://github.com/SumanyaG/Alkalytics/blob/main/src/frontend/test/hooks/UploadArea.test.tsxy}{Shows upload status indicators} & Pass \\
    \hline
    6.3-UT13 & \href{https://github.com/SumanyaG/Alkalytics/blob/main/src/frontend/test/hooks/UploadArea.test.tsx}{Handles drag over and drop event with valid file} & Pass \\
    \hline
    6.3-UT14 & \href{https://github.com/SumanyaG/Alkalytics/blob/main/src/frontend/test/hooks/UploadArea.test.tsx}{Handles file input change event with valid file} & Pass \\
    \hline
    6.2-UT15 & \href{https://github.com/SumanyaG/Alkalytics/blob/main/src/frontend/test/hooks/TableBody.test.tsx}{Handles cell editing} & Pass \\
    \hline
  \end{tabular}
  \caption{Unit tests summary for Data Storage Module}
  \label{UT:DataStorage}
\end{table}


\newpage
\subsection{Data Retrieval Module}
Tables \ref{UT:DataRetrieval1} and \ref{UT:DataRetrieval2} provides a summary of the results for each test that is
relevant to the Data Retrieval module.
\begin{table}[H]
  \centering
  \begin{tabular}{|l|l|l|}
    \hline
    \textbf{Test ID} & \textbf{Test Name} & \textbf{Result}\\
    \hline
    6.2.1-UT1 & \href{https://github.com/SumanyaG/Alkalytics/blob/main/src/frontend/test/hooks/useTable.test.tsx}{Initialize with default values} & Pass \\
    \hline
    6.2.1-UT2 & \href{https://github.com/SumanyaG/Alkalytics/blob/main/src/frontend/test/hooks/useTable.test.tsx}{Fetch experiment IDs} & Pass \\
    \hline
    6.2.1-UT3 & \href{https://github.com/SumanyaG/Alkalytics/blob/main/src/frontend/test/hooks/useTable.test.tsx}{Fetch experiments} & Pass \\
    \hline
    6.2.1-UT4 & \href{https://github.com/SumanyaG/Alkalytics/blob/main/src/frontend/test/hooks/useTable.test.tsx}{Fetch data} & Pass \\
    \hline
    6.2.1-UT5 & \href{https://github.com/SumanyaG/Alkalytics/blob/main/src/frontend/test/hooks/useTable.test.tsx}{Handle Apollo error states} & Pass \\
    \hline
    6.2.1-UT6 & \href{https://github.com/SumanyaG/Alkalytics/blob/main/src/frontend/test/hooks/useTable.test.tsx}{Refetch experiments and data} & Pass \\
    \hline
    6.2.1-UT7 & \href{https://github.com/SumanyaG/Alkalytics/blob/main/src/frontend/test/hooks/useTable.test.tsx}{Call RefetchExperiments function} & Pass \\
    \hline
    6.2.1-UT8 & \href{https://github.com/SumanyaG/Alkalytics/blob/main/src/frontend/test/hooks/useTable.test.tsx}{Call RefetchData function} & Pass \\
    \hline
    6.2.1-UT9 & \href{https://github.com/SumanyaG/Alkalytics/blob/main/src/frontend/test/hooks/TableHeader.test.tsx}{Filter data by search keyword} & Pass \\
    \hline
  \end{tabular}
  \caption{Unit tests summary for Data Retrieval Module}
  \label{UT:DataRetrieval1}
\end{table}

\begin{table}[H]
  \centering
  \begin{tabular}{|l|l|l|}
    \hline
    \textbf{Test ID} & \textbf{Test Name} & \textbf{Result}\\
    \hline
    6.2.2-UT1 & \href{https://github.com/SumanyaG/Alkalytics/blob/documentation/VnV/src/frontend/test/components/modals/DataFormModel.test.tsx}{Initialize with default step} & Pass \\
    \hline
    6.2.2-UT2 & \href{https://github.com/SumanyaG/Alkalytics/blob/documentation/VnV/src/frontend/test/components/modals/DataFormModel.test.tsx}{Navigate through all steps correctly} & Pass \\
    \hline
    6.2.2-UT3 & \href{https://github.com/SumanyaG/Alkalytics/blob/documentation/VnV/src/frontend/test/components/modals/DataFormModel.test.tsx}{Validates required fields at each step} & Pass \\
    \hline
    6.2.2-UT4 & \href{https://github.com/SumanyaG/Alkalytics/blob/documentation/VnV/src/frontend/test/components/modals/DataFormModel.test.tsx}{Shows error for invalid parameter selection} & Pass \\
    \hline
    6.2.2-UT5 & \href{https://github.com/SumanyaG/Alkalytics/blob/documentation/VnV/src/frontend/test/components/modals/DataFormModel.test.tsx}{Validates axis ranges and shows errors} & Pass \\
    \hline
    6.2.2-UT6 & \href{https://github.com/SumanyaG/Alkalytics/blob/documentation/VnV/src/frontend/test/components/modals/DataFormModel.test.tsx}{Handles form submission successfully} & Pass \\
    \hline
  \end{tabular}
  \caption{Unit tests summary for Data Retrieval Form Module}
  \label{UT:DataRetrieval2}
\end{table}


\begin{table}[H]
  \centering
  \begin{tabular}{|l|l|l|}
    \hline
    \textbf{Test ID} & \textbf{Test Name} & \textbf{Result}\\
    \hline
    6.2.3-UT1 & \href{https://github.com/SumanyaG/Alkalytics/blob/main/src/frontend/test/hooks/useGraph.test.tsx}{should fetch all graphs when latest=0} & Pass \\
    \hline
    6.2.3-UT2 & \href{https://github.com/SumanyaG/Alkalytics/blob/main/src/frontend/test/hooks/useGraph.test.tsx}{should fetch latest graph when latest=1} & Pass \\
    \hline
    6.2.3-UT3 & \href{https://github.com/SumanyaG/Alkalytics/blob/main/src/frontend/test/hooks/useGraph.test.tsx}{should handle network errors} & Pass \\
    \hline
  \end{tabular}
  \caption{Unit tests summary for Data Retrieval useGraph Hook}
  \label{UT:DataRetrieval3}
\end{table}
\subsection{Visualization Module}

\subsection{User Management Module}
Table \ref{UT:UM} provides a summary of the results for each test that is
relevant to the User Management module.
\begin{table}[H]
  \centering
  \begin{tabular}{|l|l|l|l|}
    \hline
    \textbf{Test ID} & \textbf{Test Description} & \textbf{Result}\\
    \hline
    6.4-UT1 & \href{https://github.com/SumanyaG/Alkalytics/blob/main/src/backend/test/userServiceTest.py}{Creates a new user} & Pass\\
    \hline
    6.4-UT2 & \href{https://github.com/SumanyaG/Alkalytics/blob/main/src/backend/test/userServiceTest.py}{Skips user creation if already registered} & Pass\\
    \hline
    6.4-UT3 & \href{https://github.com/SumanyaG/Alkalytics/blob/main/src/backend/test/userServiceTest.py}{Validates login attempt with correct password} & Pass \\
    \hline
    6.4-UT4 & \href{https://github.com/SumanyaG/Alkalytics/blob/main/src/backend/test/userServiceTest.py}{Invalidates login attempt with incorrect password} & Pass\\
    \hline
    6.4-UT5 & \href{https://github.com/SumanyaG/Alkalytics/blob/main/src/backend/test/userServiceTest.py}{Retrieves current user's email and role} & Pass \\
    \hline
    6.4-UT6 & \href{https://github.com/SumanyaG/Alkalytics/blob/main/src/backend/test/userServiceTest.py}{Creates session information for current user} & Pass\\
    \hline
    6.4-UT7 & \href{https://github.com/SumanyaG/Alkalytics/blob/main/src/backend/test/userServiceTest.py}{Removes session information for current user} & Pass\\
    \hline
    6.4-UT8 & \href{https://github.com/SumanyaG/Alkalytics/blob/main/src/backend/test/userServiceTest.py}{Updates current user's password} & Pass\\
    \hline
    6.4-UT9 & \href{https://github.com/SumanyaG/Alkalytics/blob/main/src/backend/test/userServiceTest.py}{Updates current user's role} & Pass\\
    \hline
    6.4-UT10 & \href{https://github.com/SumanyaG/Alkalytics/blob/main/src/backend/test/userServiceTest.py}{Removes existing user} & Pass\\
    \hline
    6.4-UT11 & \href{https://github.com/SumanyaG/Alkalytics/blob/main/src/backend/test/userServiceTest.py}{Handles removal of nonexisting user} & Pass\\
    \hline
    6.4-UT13 & \href{https://github.com/SumanyaG/Alkalytics/blob/main/src/frontend/test/components/auth/AccountMenu.test.tsx}{Handles logout and user redirected to login page} & Pass\\
    \hline
    6.4-UT14 & \href{https://github.com/SumanyaG/Alkalytics/blob/main/src/frontend/test/components/auth/ProtectedRoute.test.tsx}{Handles page access when not logged in} & Pass\\
    \hline
    6.4-UT15 & \href{https://github.com/SumanyaG/Alkalytics/blob/main/src/frontend/test/context/authContext.test.tsx}{Checks for session creation when logged in} & Pass\\
    \hline
    6.4-UT16 & \href{https://github.com/SumanyaG/Alkalytics/blob/main/src/frontend/test/context/authContext.test.tsx}{Checks for session removal when logged out} & Pass\\
    \hline
  \end{tabular}
  \caption{Unit tests summary for User Management Module}
  \label{UT:UM}
\end{table}

\section{Changes Due to Testing}
\subsection{Changes Due to Supervisor Feedback}

Our supervisor suggested the current data visualization form intake could be
improved by allowing the user to filter the experiment data by additional
parameters, rather than only by the date it was performed. This feature will be
added to the final product.

They have also requested to add a new feature that was not originally part of
the requirements. This feature allows the user to compute a variety of
efficiency factors related to the data, measuring the efficiencies of the
membrane used in the experiment. This feature will be added to the final
product, but thorough validation testing will be limited due to time
constraints.


%\wss{This section should highlight how feedback from the users and from the
%supervisor (when one exists) shaped the final product.  In particular the
%feedback from the Rev 0 demo to the supervisor (or to potential users) should be
%highlighted.}

\subsection{Changes Due to Failed Tests}
The following sections outline changes that will be made to the implementation
following a failed test or tests. This does not include tests for stretch goal
requirements that failed.

\subsubsection{\hyperref[NFR:S3]{NFR-S3}} The system did not automatically log
out the user after a period of inactivity. This was due to the lack of a session
timeout manager, which will be implemented in the final product.

\subsubsection{\hyperref[NFR:UH4]{NFR-UH4}} The system was not complient to with
the WCAG standards. This is due to the fact that the current research team using
the project does not need any accessibility considerations when using the app.
However this may not be true in the future, as such, updates to the current
pages will be implemented using the feedback from SiteImprove in the final
product.

\section{Automated Testing}
Pytest was used as the automated testing framework for Python. Unit test suites
were executed using Pytest, which supports fixtures, parameterization, and code
coverage metrics, making automated testing more efficient and flexible. Jest was
used as the automated testing framework for React components.
		
\section{Trace to Requirements}

\begin{table}[H]
  \centering
  \centerline{
  \begin{tabular}{|c|c|c|c|c|c|c|c|}
  \hline
  Test ID & FR-ST1 & FR-ST1.1 & FR-ST2 & FR-ST3 & FR-ST4 & FR-ST5 & FR-ST6\\
  \hline
  \hline
  FR-1 & X & X & & & & &\\
  \hline
  FR-2 & X & X & & & & &\\
  \hline
  FR-3 & X & X & & & & &\\
  \hline
  FR-4 & X & X & & & & &\\
  \hline
  FR-5 & & & X & & & &\\
  \hline
  FR-6 & & & X & & & &\\
  \hline
  FR-7 & & & X & & & &\\
  \hline
  FR-8 & & & X & & & &\\
  \hline
  FR-9 & & & X & & & &\\
  \hline
  FR-10 & & & & X & & &\\
  \hline
  FR-11 & & & & X & & &\\
  \hline
  FR-12 & & & & & X & &\\
  \hline
  FR-13 & & & & & X & &\\
  \hline
  FR-14 & & & & & & X &\\
  \hline
  FR-15 & & & & & & & X\\
  \hline
  \end{tabular}
  }
  \caption{Traceability Matrix for Test Cases and Functional Requirements}
  \label{table:traceFR}
\end{table}

\begin{table}[H]
  \centering
  \centerline{
  \begin{tabular}{|c|c|c|c|c|c|c|c|c|c|c|}
  \hline
  Test ID \{NFR-\} & LF1 & LF2 & UH1 & UH2 & UH3 & UH4 & OE1 & OE2 & MSR1 &
  MSR2\\
  \hline
  \hline
  LFR-1 & X & & & & & & & & &\\
  \hline
  LFR-2 & & X & & & & & & & &\\
  \hline
  LFR-3 & X & & & & & & & & &\\
  \hline
  LFR-4 & X & & & & & & & & &\\
  \hline
  LFR-5 & X & & & & & & & & &\\
  \hline
  LFR-6 & X & & & & & & & & &\\
  \hline
  LFR-7 & X & & & & & & & & &\\
  \hline
  UHR-1 & & & X & & & & & & &\\
  \hline
  UHR-2 & & & & X & & & & & &\\
  \hline
  UHR-3 & & & & & X & & & & &\\
  \hline
  UHR-4 & & & X & & & & & & &\\
  \hline
  UHR-5 & & & X & & & & & & &\\
  \hline
  UHR-6 & & & & & & X & & & &\\
  \hline
  OER-2 & & & & & & & X & & &\\
  \hline
  OER-3 & & & & & & & X & & &\\
  \hline
  OER-4 & & & & & & & X & & &\\
  \hline
  OER-5 & & & & & & & & X & &\\
  \hline
  MSR-4 & & & & & & & & & X &\\
  \hline
  MSR-6 & & & & & & & & & & X\\
  \hline
  \end{tabular}
  }
  \caption{Traceability Matrix for Test Cases and Nonfunctional Requirements (Part 1)}
  \label{table:traceNFR1}
\end{table}

\begin{table}[H]
  \centering
  \centerline{
  \begin{tabular}{|c|c|c|c|c|c|c|c|c|c|c|}
  \hline
  Test ID \{NFR-\} & P1 & P2 & P3 & P4 & P5 & P6 & P7 & P8 & P9\\
  \hline
  \hline
  PR-1 & X & & & & & & & &\\
  \hline
  PR-2 & & X & & & & & & &\\
  \hline
  PR-3 & & & X & & & & & &\\
  \hline
  PR-4 & & X & & & & & & &\\
  \hline
  PR-5 & & & & X & & & & &\\
  \hline
  PR-6 & & & & & X & & & &\\
  \hline
  PR-7 & & & & & X & & & &\\
  \hline
  PR-8 & & & & & X & & & &\\
  \hline
  PR-9 & & & & & & X & & &\\
  \hline
  PR-10 & & & & & & & X & &\\
  \hline
  PR-11 & & & & & & & & X &\\
  \hline
  PR-12 & & & & & & & & & X\\
  \hline
  \end{tabular}
  }
  \caption{Traceability Matrix for Test Cases and Nonfunctional Requirements (Part 2)}
  \label{table:traceNFR2}
\end{table}

\begin{table}[H]
  \centering
  \centerline{
  \begin{tabular}{|c|c|c|c|c|c|c|c|c|c|c|}
  \hline
  Test ID \{NFR-\} & SR1 & SR2 & SR3 & SR4 & SR5 & SR6 & SR7 & SR8 & SR9\\
  \hline
  \hline
  SR-1 & X & & & & & & & &\\
  \hline
  SR-2 & & X & & & & & & &\\
  \hline
  SR-3 & & X & & & & & & &\\
  \hline
  SR-4 & & & X & & & & & &\\
  \hline
  SR-5 & & & & X & & & & &\\
  \hline
  SR-6 & & & & & X & & & &\\
  \hline
  SR-7 & & & & & X & & & &\\
  \hline
  SR-8 & & & & & X & & & &\\
  \hline
  SR-9 & & & & X & & & & &\\
  \hline
  SR-12 & & & & & & X & & &\\
  \hline
  SR-13 & & & & & & & X & &\\
  \hline
  SR-14 & & & & & & & X & &\\
  \hline
  SR-15 & & & & & & & X & &\\
  \hline
  SR-16 & & & & & & & & X &\\
  \hline
  SR-17 & & & & & & & & & X\\
  \hline
  \end{tabular}
  }
  \caption{Traceability Matrix for Test Cases and Nonfunctional Requirements (Part 3)}
  \label{table:traceNFR3}
\end{table}
		
\section{Trace to Modules}		
\begin{table}[H]
  \centering
  \centerline{
  \begin{tabular}{|c|c|c|c|c|c|c|c|}
  \hline
  Test ID & FR-ST1 & FR-ST1.1 & FR-ST2 & FR-ST3 & FR-ST4 & FR-ST5 & FR-ST6\\
  \hline
  \hline
  M6 & X & X & & & X & &\\
  \hline
  M7 & & & X & X & & &\\
  \hline
  M11 & & & X & & & &\\
  \hline
  M12 & & & X & & & &\\
  \hline
  M13 & & & & & & X &\\
  \hline
  M15 & X & & & & & &\\
  \hline
  M16 & & & & & X & &\\
  \hline
  M17 & & & & & & & X\\
  \hline
  \end{tabular}
  }
  \caption{Traceability Matrix for System Test Cases and Modules}
  \label{table:modTrace1}
\end{table}

\section{Code Coverage Metrics}
The team has determined code coverage metrics were not indicative of valuable
information in relation to the validation and verification of the
implementation, thus this section is omitted.

%\bibliographystyle{plainnat} \bibliography{../../refs/References}

\newpage{}
\section*{Appendix --- Reflection}

\begin{enumerate}
  \item \textbf{What went well while writing this deliverable?}

  Writing the general structure of the report was straightforward as we could
  refer back to the tests we had written in the VnV Plan. Having detailed system tests 
  planned out made testing the system a lot easier given that we had a well established
  jumping ground.

  \item \textbf{What pain points did you experience during this deliverable, and
  how did you resolve them?}

  Constructing and executing all of our unit tests was the biggest challenge primarily due to
  time constraints. To combat this, the team lead held regular check-ins to ensure that everyone's
  individual timelines were aligned.

  We also ran into the issue of interpreting ambiguous user feedback during the usability testing. This 
  was resolved by consulting back to our course 4HC3 - Human Computer Interaction and changing the style of the 
  session so that instead of open-ended feedback, the user is asked to choose from pre-determined options with 
  the choice to provide additional feedback later on.

  \item \textbf{Which parts of this document stemmed from speaking to your
  client(s) or a proxy (e.g. your peers)? Which ones were not, and why?}

  Most of the non-functional requirements stemmed from speaking to our clients who acted as the users for 
  the usability test. They were consulted when conducting system tests and ensuring that the outlined tests were 
  in alignment with the application requirements. 

  The parts of the document related to automated testing and unit testing such as testing the hooks and APIs and more 
  did not stem from speaking to the clients or peers. This is because most of the tests were to ensure that the system 
  compiles and functions correctly from a technical perspective rather than focusing on user experience or usability. These 
  tests were derived from the development team's understanding of the system architecture, coding standards, and best practices 
  for ensuring reliability, performance and maintainability.

  \item \textbf{In what ways was the Verification and Validation (VnV) Plan
  different from the activities that were actually conducted for VnV?  If there
  were differences, what changes required the modification in the plan?  Why did
  these changes occur?  Would you be able to anticipate these changes in future
  projects?  If there weren't any differences, how was your team able to clearly
  predict a feasible amount of effort and the right tasks needed to build the
  evidence that demonstrates the required quality?  (It is expected that most
  teams will have had to deviate from their original VnV Plan.)}

  The primary way in which the VnV Plan is different from the activities that were actually conducted for 
  the VnV Report is implementation of features. After Revision 0, the scope of the project was changed after consulting the professor 
  ans given the change in the priority of requirements from the client. This resulted in a number of features now being labelled as 
  stretch goals which were, at the time of testing, not implemented. Thus, the tests for these features had to be labelled as `Out of Scope'.

  Given more experience in the iterative nature of a software project of this scale, we would like to believe that we can manage the project 
  expectations better and define a more realistic implementation timeline at the get go, in the future.

  Additionally, for a few features, the scope of implementation had to be reduced. One example is the user management system and authorization 
  protocols. These were not implemented in as robust a fashion as earlier expected due to the limited resources at the team's disposal. For instance, it was
  mutually decided that it would be better to allocate the time and effort required to make the protocol more robust to instead implement additional
  analysis features.
\end{enumerate}

\end{document}
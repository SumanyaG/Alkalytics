\documentclass{article}

\usepackage{booktabs}
\usepackage{tabularx}

%% Common Parts

\newcommand{\progname}{Software Engineering} % PUT YOUR PROGRAM NAME HERE
\newcommand{\authname}{Team 21, Alkalytics
\\ Sumanya Gulati - gulats10
\\ Kate Min - mink9
\\ Jennifer Ye - yej52
\\ Jason Tran - tranj78} % AUTHOR NAMES                  

\usepackage{hyperref}
    \hypersetup{colorlinks=true, linkcolor=blue, citecolor=blue, filecolor=blue,
                urlcolor=blue, unicode=false}
    \urlstyle{same}
                                


\title{Usability Testing Survey\\\progname}

\author{\authname}

\date{}

\begin{document}

\maketitle

\begin{table}[hp]
\caption{Revision History} \label{TblRevisionHistory}
\begin{tabularx}{\textwidth}{llX}
\toprule
\textbf{Date} & \textbf{Developer(s)} & \textbf{Change}\\
\midrule
14 March 2025 & Sumanya Gulati & Initial Draft\\
17 March 2025 & Sumanya Gulati & Add list of navigation tasks\\
20 March 2025 & Sumanya Gulati & Update test details and add observations\\
\bottomrule
\end{tabularx}
\end{table}

\newpage

\section{User and Session Details}
Table \ref{UserDetails} contains details of each subject who agreed to participate in a 
Usability Testing Session.
\begin{table}[hp]
\begin{tabularx}{\textwidth}{l | X}
\toprule
\textbf{Subject Name} & Dr. Charles de Lannoy \\
\midrule
\textbf{Subject Role} & Project Supervisor \\
\midrule
\textbf{Test Conductor(s)} & Sumanya Gulati \\
\midrule
\textbf{Test Date} & 20 March 2025 \\
\midrule
\textbf{Test Mode} & In-person \\
\midrule
\textbf{Start Time} & 12:00 PM \\
\midrule
\textbf{End Time} & 12:50 PM \\
\midrule
\textbf{Additional Notes} & - \\
\bottomrule
\end{tabularx}
\caption{User and Session Details} \label{UserDetails}
\end{table}

\section{Navigation Tasks to be Performed}
For the purpose of this usability session, the subject has been asked to perform the following tasks, in order:
\begin{enumerate}
  \item Upload a data sheet. \label{T1}
  \item Upload another data sheet with the same experiment date. \label{T2}
  \item Locate this data sheet in the table view page. \label{T3}
  \item Navigate to the experiment table and perform the following operations - \label{T4}
  \begin{enumerate}
    \item Add a new row.
    \item Delete an existing row.
    \item Add a new column.
    \item Delete an existing column.
  \end{enumerate}
  \item Calculate the average of any couple of values in the experiment table. \label{T5}
  \item \emph{(Optional):} Perform additional excel-based functions on the values in the experiment table.
  \item Search for a datapoint in the table using the column and value. \label{T6}
  \item Generate a graph using all available customization options for the experiment data from \textbf{September 26}. \label{T7}
\end{enumerate}

\section{User Answers}

Users will be asked the following questions to gauge the usability of the
application during a user demo assessment. These questions relate to
the assessment of non-functional tests NFR-LF1, NFR-UH1, NFR-OE2, and NFR-MS1 outlined in the 
\href{https://github.com/SumanyaG/Alkalytics/blob/main/docs/SRS/SRS.pdf}{Software Requirements Specification (SRS)}.

\subsection*{\textbf{Navigation and Ease of Use}}
  \begin{itemize}
    \item[(a)] How easy or difficult was it to navigate through the
    application? \textit{(Very easy - Somewhat easy - Neither easy nor difficult
    - Somewhat difficult - Very difficult)}
    \item \emph{Answer:}
    \item[(b)] Rate the ease of finding a specific button/feature related to the
    task you were trying to perform. \textit{(Very easy - Somewhat easy -
    Neither easy nor difficult - Somewhat difficult - Very difficult)}
    \item \emph{Answer:}
  \end{itemize}

\subsection*{\textbf{Visual Appearance}}
  \begin{itemize}
    \item[(a)] On a scale of 1 to 5, how would you rate the visual appearance of
    the application? \textit{(1 - Too cluttered and unappealing, 2 - Somewhat
    unappealing 3 - Neutral, 4 - Somewhat appealing, 5 - Very clean and
    appealing)}
    \item \emph{Answer:}
    \item[(b)] Is the displayed content (text, tables, graphs) clear, legible
    and easy to understand?
    \item \emph{Answer:}
    \item[(c)] Did the design and layout appear consistent to you across
    the entire application? If no, please point out any inconsistencies you noticed.
    \item \emph{Answer:}
    \item[(d)] Did you encounter any unidentifiable symbols or icons? If yes,
    please describe them.
    \item \emph{Answer:}
  \end{itemize}

\subsection*{\textbf{Learning}}
  \begin{itemize}
    \item[(a)] How easy or difficult was it to learn how to use the different
    features of the application? \textit{(Very easy - Somewhat easy - Neither
    easy nor difficult - Somewhat difficult - Very difficult)}
    \item \emph{Answer:}
    \item[(b)] Were there any features or functions you found challenging to
    understand or figure out? If yes, please describe. 
    \item \emph{Answer:}
  \end{itemize}

\subsection*{\textbf{Responsiveness}}
  \begin{itemize}
    \item[(a)] Did you experience any noticeable delays or interruptions while
    using the application? If yes, please describe them.
    \item \emph{Answer:}
  \end{itemize}

\subsection*{\textbf{Overall Experience}}
  \begin{itemize}
    \item[(a)] On a scale of 1 to 5, rate your overall experience using the
    application. \textit{(1 - Very poor, 2 - Poor, 3 - Neutral, 4 - Good, 5 -
    Excellent)}
    \item \emph{Answer:}
    \item[(b)] Do you have any suggestions for improvements to enhance your
    experience with the application?
    \item \emph{Answer:}
  \end{itemize}

\section{Test Conductor Observations}
\subsection{Queries}
\begin{enumerate}
  \item For tasks \ref{T1} and \ref{T2}, do the sheets have to be uploaded as data sheets or experiment sheets? How 
  are they different? \label{Q1} 
  \item Very hard to validate task \ref{T3} by having to do multiple look-ups. \label{Q2}
  \item For task \ref{T3}, the subject had to spend roughly 2 minutes trying different format options to look up the file name in the 
  experiment table. \label{Q3}
  \item For task \ref{T4}, the subject was unable to find out which page to navigate to in order to perform the given actions. \label{Q4}
\end{enumerate}

\subsection{Elements Behaving Unexpectedly}
\begin{enumerate}
  \item The calculations bar did not work despite multiple tries. The help/info button next to the bar did not help much. Thus, subject was 
  unable to perform task \ref{T5}.
  \item For task \ref{T6}, the search bar does not return all relevant values. Only a few values are displayed.
\end{enumerate}

\subsection{Additional Subject Feedback}
\begin{enumerate}
  \item Change home icon to something more intuitive and more descriptive of `summary data'.
  \item The queries page looks very cluttered and similar buttons and modals should be positioned together to help with user flow.
  \item For edit functionality, there are no scenarios where the user should be allowed to overwrite imported data in the experiments table. 
  User action should be restricted to add/edit values only to columns that were not imported from a data file.
\end{enumerate}

\section{Proposed Changes}
\begin{itemize}
  \item For query \ref{Q1}, rename `data sheet' to `raw data' and `experiment sheet' to `processed data' or `summarized data/results'.
  \item For query \ref{Q2}, add date and timestamp to the dashboard for when a data sheet was uploaded.
  \item For query \ref{Q3}, add a pop-up or a help button mentioning that file ID must be entered in the format \emph{yyyy-mm-dd} to avoid 
  confusion.
  \item For query \ref{Q4}, rename queries page and replace icon to something more relevant.
  \item Separate the edit and calculations bar in the queries page. Place the edit button along the set column types button and the search bar,
  separate from calculations.
  \item For task \ref{T6}, allow for more comprehensive search functionality. For example, seraching \emph{1.5*} should display all values that start
  with 1.5 such as 1.58, 1.534 etc.
  \item Implement an edit button for graph generation so user can select different parameters for a graph without going through the entire form every time.
\end{itemize}

\end{document}
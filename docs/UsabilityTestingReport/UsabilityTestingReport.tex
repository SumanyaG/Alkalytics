\documentclass{article}

\usepackage{booktabs}
\usepackage{tabularx}
\usepackage{amssymb}

\input{../Comments}
%% Common Parts

\newcommand{\progname}{Software Engineering} % PUT YOUR PROGRAM NAME HERE
\newcommand{\authname}{Team 21, Alkalytics
\\ Sumanya Gulati - gulats10
\\ Kate Min - mink9
\\ Jennifer Ye - yej52
\\ Jason Tran - tranj78} % AUTHOR NAMES                  

\usepackage{hyperref}
    \hypersetup{colorlinks=true, linkcolor=blue, citecolor=blue, filecolor=blue,
                urlcolor=blue, unicode=false}
    \urlstyle{same}
                                


\title{Usability Testing Report\\\progname}

\author{\authname}

\date{}

\begin{document}

\maketitle

\newpage

\begin{table}[hp]
\caption{Revision History} \label{TblRevisionHistory}
\begin{tabularx}{\textwidth}{llX}
\toprule
\textbf{Date} & \textbf{Developer(s)} & \textbf{Change}\\
\midrule
2 April 2025 & Kate Min & Finalized document\\
\bottomrule
\end{tabularx}
\end{table}

\newpage

\tableofcontents

\newpage

\section{Executive Summary}
This report summarizes the findings from the usability testing performed for the
Alkalytics project. The purpose of this testing was to evaluate the usability of
the software and determine areas that required further refinement.
This report includes objectives of the testing, the methodologies used, results,
conclusions, and a list of proposed changes or modifications to the User
Interface (UI) and core features. Implemented changes are indicated where
applicable.

\section{Objectives}
The primary goals of the usability testing for the Alkalytics project are
outlined below:

\begin{itemize} 
    \item Evaluate the learnability of the application.
    \item Evaluate the navigation and ease of use of the application.
    \item Evaluate the design of the application.
    \item Evaluate the overall user satisfaction and whether it meets the user's
    expectations.
\end{itemize}

The corresponding minimum success criteria are included where relevant, taken
from the
\href{https://github.com/SumanyaG/Alkalytics/blob/main/docs/SRS/SRS.pdf}{Software
Requirements Specification (SRS)} and
\href{https://github.com/SumanyaG/Alkalytics/blob/main/docs/VnVPlan/VnVPlan.pdf}{Verification
and Validation (VnV) Plan} documentation:

\section{Test Methodology}
This section outlines the methodologies used to conduct the usability testing
sessions. 

\subsection{Test Participants}
The usability testing was conducted through two separate sessions with the
main project supervisor and client, Dr. Charles de Lannoy, hereafter referred
to as \textbf{P1}, and one of the lab's undergraduate research assistants, Meghna Saha,
hereafter referred to as \textbf{P2}.

Test participants were selected based on the following 3 aspects:
\begin{itemize}
    \item[a.] Familiarity with the application's intended use and the data involved,
    \item[b.] Availability to participate in a testing session \emph{in-person}, and
    \item[c.] Ability to provide valuable feedback on the usability of the
    application.
\end{itemize}

The team determined that asking users outside of the affiliated research team to
participate in the usability testing would not be a good reflection of the
representative users and may skew the results in a direction that would not be
beneficial to the actual end users. Though it is standard practice to have at
least 5-10 participants, time constraints, participant availability, and the
aforementioned reasoning led to the testing being conducted with only 2
participants.

\subsection{Test Environment}
The usability testing sessions were conducted in a controlled environment, with
the software installed and running on the test conductor's laptop. The test
conductor was present to observe the participants, take notes on the elapsed
time for each task, and provide assistance only if the participant requested it
or if abnormal behaviours in the application occurred.

\subsection{Test Tasks}
Each participant was asked to perform a series of tasks using the application,
listed below.
\begin{enumerate}
    \item Upload a data sheet. \label{T1}
    \item Upload another data sheet with the same experiment date. \label{T2}
    \item Locate this data sheet in the table view page. \label{T3}
    \item Navigate to the experiment table and perform the following operations - \label{T4}
    \begin{enumerate}
      \item Add a new row.
      \item Delete an existing row.
      \item Add a new column.
      \item Delete an existing column.
    \end{enumerate}
    \item Calculate the average of any couple of values in the experiment table. \label{T5}
    \item \emph{(Optional):} Perform additional excel-based functions on the values in the experiment table.
    \item Search for a datapoint in the table using the column and value. \label{T6}
    \item Generate a graph using all available customization options for the experiment data from \textbf{September 26}. \label{T7}
\end{enumerate}

\subsection{Follow-up Questionnaire}
Following a usability testing session, the participant was asked to complete a
questionnaire. A sample survey with all of the questions can be found
\href{https://github.com/SumanyaG/Alkalytics/blob/main/docs/UsabilityTestingReport/Survey/SampleSurvey.pdf}{here}.
The questions were designed to gather feedback on the usability of the
software, focusing on the following five aspects:
\begin{itemize}
    \item[a.] Navigation and Ease of Use
    \item[b.] Visual Appearance
    \item[c.] Learning
    \item[d.] Responsiveness
    \item[e.] Overall Experience
\end{itemize}

\section{Summarized Results and Statistics}
This section summarizes the results of the usability testing, including feedback
from users and resultant metrics.

\subsection{Survey Responses}
Section 3 in the following documents contain the survey responses for each
participant:
\begin{enumerate}
    \item
    \href{https://github.com/SumanyaG/Alkalytics/blob/main/docs/UsabilityTestingReport/Survey/SurveyResults/CharlesDeLannoy/UsabilityTestingSurveyDeLannoy.pdf}{Survey Results for P1}
    \item
    \href{https://github.com/SumanyaG/Alkalytics/blob/main/docs/UsabilityTestingReport/Survey/SurveyResults/MeghnaSaha/UsabilityTestingSurveySaha.pdf}{Survey
    Results for P2}
\end{enumerate}

Responses that were of particular interest are noted below.

\subsubsection{Navigation and Ease of Use}
\begin{itemize}
    \item[(b)] Rate the ease of finding a specific button/feature related to the
    task you were trying to perform.
    \item \textbf{\emph{P1}}: Somewhat difficult
\end{itemize}

\subsubsection{Visual Appearance}
\begin{itemize}
    \item[(b)] Is the displayed content (text, tables, graphs) clear, legible
    and easy to understand?
    \item \textbf{\emph{P1}}: The text should be larger, especially on the graph axes. The
    tables are clear and legible and the graphs need to have axes updated
    with selected parameters.
    \item \textbf{\emph{P2}}: Yes, very clear except for the queries page that was too
    cluttered.
    \item[(d)] Did you encounter any unidentifiable symbols or icons? If yes,
    please describe them.
    \item \textbf{\emph{P1}}: Yes, the queries page.
\end{itemize}

\subsubsection{Learning}
\begin{itemize}
    \item[(a)] How easy or difficult was it to learn how to use the different
    features of the application?
    \item \textbf{\emph{P1}}: Somewhat difficult as the labels need to better
    reflect their purpose.
    \item[(b)] Were there any features or functions you found challenging to
    understand or figure out? If yes, please describe. 
    \item \textbf{\emph{P1}}: Performing calculations on the rows and columns
    was not intuitive. This needed  better layout design. Also, building the
    graph needs improvement to better select parameters, make edits after a
    selection was made and identify the proper data to use.
\end{itemize}

\subsubsection{Overall Experience}
\begin{itemize}
    \item[(b)] Do you have any suggestions for improvements to enhance your
    experience with the application?
    \item \textbf{\emph{P1}}: It’s on the right track, just a few improvements
    to organization, label clarifications, and graphical outputs needed.
    \item \textbf{\emph{P2}}: The dashboard should display a list of previous
    insights each graph gave along with previous analyses instead. Additionally,
    there could be a welcome thing added to the very top of the dashboard. Also,
    just to maintain contextual consistency, all column names should be replaced
    with their full forms and units should be added for all of them.
\end{itemize}

\subsection{Conclusions Drawn from Results}
Overall, the survey results indicated that the application is generally
user-friendly and visually appealing, with an average rating of 3/5 across both
participants for overall satisfaction.
However, some issues were identified, particularly with the Queries page. Both
participants reported confusion regarding the terminology used, specifically the
distinction between ``data sheet'' and ``experiment sheet''. Additionally, the
visual appearance of the Queries page was not intuitive, with both participants
noting difficulties in locating it within the navigation bar, as well as finding
and using key features on the page itself.

\section{Summary of Proposed Changes}
This section summarizes the proposed changes to the user interface and
functionality of the application based on the results of the usability testing.
These points are taken and paraphrased from Section 5 of P1's Survey Results and
Section 4 of P2's Survey Results. 

Changes marked with a check mark ($\checkmark$) are considered high priority and
should be implemented as soon as possible. Due to time constraints, not all of
these may be implemented. If it has been implemented in the final product, the
exact commit implementing the change is hyperlinked in parentheses. Changes
marked with an $\times$ have been determined as low priority and should be
implemented if time allows, else it would be deferred to a future developing
team.

\subsection{Changes to the User Interface}

\begin{itemize}
    \item[$\checkmark$] Rename `data sheet' to `raw data' and `experiment sheet'
    to `processed data' or `summarized data/results' in the Upload page and any other
    subsequent appearances of the two terms. (see
    \href{https://github.com/SumanyaG/Alkalytics/commit/53ba5521bb378b55e89e74bd13ca00e57eb8eb28}{commit})
    \item[$\times$] Add a pop-up or a help button to explain that a file ID must be
    entered in the format \texttt{yyyy-mm-dd}.
    \item[$\checkmark$] Rename the Queries page and replace its icon in the
    navigation bar to something more relevant and intuitive. (see
    \href{https://github.com/SumanyaG/Alkalytics/commit/d3c98b683118b141b05c5a09b3c770c3c03a897c}{commit})
    \item[$\checkmark$] Re-arrange the different button elements in the Queries
    page, keeping the calculation bar separate from unrelated buttons. (see
    \href{https://github.com/SumanyaG/Alkalytics/commit/8a745c4c77a444c449af95b4ebfeed01ffa6b4b6}{commit})
    \item[$\checkmark$] All column names should be displayed in their full form
    with units. (see
    \href{https://github.com/SumanyaG/Alkalytics/commit/db5000a1cb3ff4bc0b9faedc4bab3176df5a1147}{commit})
    \item[$\checkmark$] The font for the graph should be bigger and units and
    title should be added for each graph.
    \item[$\checkmark$] The dashboard should display a list of previous insights
    or analyses and it could have a welcome message at the top as well. (see
    \href{https://github.com/SumanyaG/Alkalytics/commit/0a362f9e9cbe2f7c2189a084310a3527440535b9}{commit})
\end{itemize}

\subsection{Changes to the Functionality}
\begin{itemize}
    \item[$\checkmark$] Add date and timestamp for when a data sheet was
    uploaded. (see
    \href{https://github.com/SumanyaG/Alkalytics/commit/db5000a1cb3ff4bc0b9faedc4bab3176df5a1147}{commit})
    \item[$\times$] Allow for more comprehensive search functionality in the
    Queries page. For example, searching \texttt{1.5*} should display all values
    that start with 1.5.
    \item[$\checkmark$] The Excel function bar should have a more detailed usage guide.
    \item[$\checkmark$] Implement an edit button for graph generation so users
    can modify parameters for a graph without having to go through the entire
    form every single time.
    \item[$\times$] The functionality to add a new row should mimic the one that
    adds a new column.
    \item[$\times$] The past 3 graphs in the dashboard could be turned into
    interactive widgets.
\end{itemize}

\section{Reassessment of Usability After Changes}
The usability of the application must be re-evaluated after the proposed
changes have been implemented. This will help to determine if the changes have
improved the issues raised during the initial usability testing sessions.
However, due to time constraints and deadlines, the documentation of this
reassessment is not included in this report. The team will aim to conduct this
reassessment for a proper evaluation of the final product, but a comprehensive
summary of the results will not be available.

\section{Conclusions}
The usability testing sessions conducted with the two test participants provided
valuable insights into the usability of the Alkalytics application, highlighting
several areas for improvement. Although the overall design and functionality
mostly met the user's expectations, specific aspects of these needed refinement,
particularly in the navigation and visual appearance of the Queries page. The
proposed changes to the UI and key features will be taken into account for
future development.

\end{document}
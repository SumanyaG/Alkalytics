\documentclass{article}

\usepackage{float}
\restylefloat{table}

\usepackage{booktabs}

\title{Team Contributions: POC\\\progname}

\author{\authname}

\date{}

%% Comments

\usepackage{color}

\newif\ifcomments\commentstrue %displays comments
%\newif\ifcomments\commentsfalse %so that comments do not display

\ifcomments
\newcommand{\authornote}[3]{\textcolor{#1}{[#3 ---#2]}}
\newcommand{\todo}[1]{\textcolor{red}{[TODO: #1]}}
\else
\newcommand{\authornote}[3]{}
\newcommand{\todo}[1]{}
\fi

\newcommand{\wss}[1]{\authornote{blue}{SS}{#1}} 
\newcommand{\plt}[1]{\authornote{magenta}{TPLT}{#1}} %For explanation of the template
\newcommand{\an}[1]{\authornote{cyan}{Author}{#1}}

%% Common Parts

\newcommand{\progname}{Software Engineering} % PUT YOUR PROGRAM NAME HERE
\newcommand{\authname}{Team 21, Alkalytics
\\ Sumanya Gulati - gulats10
\\ Kate Min - mink9
\\ Jennifer Ye - yej52
\\ Jason Tran - tranj78} % AUTHOR NAMES                  

\usepackage{hyperref}
    \hypersetup{colorlinks=true, linkcolor=blue, citecolor=blue, filecolor=blue,
                urlcolor=blue, unicode=false}
    \urlstyle{same}
                                


\begin{document}

\maketitle

This document summarizes the contributions of each team member up to the POC
Demo.  The time period of interest is the time between the beginning of the term
and the POC demo.

\section{Demo Plans}

The following objectives have been derived from Section 9 of the
\href{https://github.com/SumanyaG/Alkalytics/blob/main/docs/DevelopmentPlan/DevelopmentPlan.pdf}{Development
Plan} and describe what will be demonstrated during the team's Proof of
Concept Demonstration.
\begin{itemize}
  \item Developing and implementing a migration algorithm for transferring CSV
  data files to the database. 
  \item Ensuring that 100\% of the existing data has been migrated  without loss
  or error to the new database.
  \item Demonstrating that all existing inter-parameter comparisons in the Excel
  templates have been replicated in the database.
\end{itemize}

\section{Team Meeting Attendance}

\begin{table}[H]
\centering
\begin{tabular}{ll}
\toprule
\textbf{Student} & \textbf{Meetings}\\
\midrule
Total & 9\\
Sumanya Gulati & 9\\
Jennifer Ye & 9\\
Jason Tran & 9\\
Kate Min & 9\\
\bottomrule
\end{tabular}
\end{table}

All team members attended every team meeting during the time period of interest.

\section{Supervisor/Stakeholder Meeting Attendance}

\begin{table}[H]
\centering
\begin{tabular}{ll}
\toprule
\textbf{Student} & \textbf{Meetings}\\
\midrule
Total & 5\\
Sumanya Gulati & 5\\
Jennifer Ye & 4\\
Jason Tran & 4\\
Kate Min & 4\\
\bottomrule
\end{tabular}
\end{table}

One of the five supervisor meetings involved a tour of the lab and the
apparatus, in which Sumanya attended alone due to differences in availability.
The total count of five supervisor meetings is fewer than the expected meeting
frequency established in Section 4.2 of the
\href{https://github.com/SumanyaG/Alkalytics/blob/main/docs/DevelopmentPlan/DevelopmentPlan.pdf}{Development
Plan} for the following reasons:
\begin{enumerate}
	\item Meetings with Dr. de Lannoy's participation were not strictly
	necessary as he is able to receive progress updates from Bassel, the
	secondary supervisor, and has been reviewing the team's deliverables sent
	via email. 
	\item There were no significant updates/questions during the weeks of
	October 8, October 15, and October 22 that would have necessitated a
	meeting; instead, some discussions were conducted via email.
\end{enumerate}

\section{Lecture Attendance}

\begin{table}[H]
\centering
\begin{tabular}{ll}
\toprule
\textbf{Student} & \textbf{Lectures}\\
\midrule
Total & 12\\
Sumanya Gulati & 2\\
Jennifer Ye & 1\\
Jason Tran & 0\\
Kate Min & 1\\
\bottomrule
\end{tabular}
\end{table}

Due to the busy schedules of the team, one member may attend lectures on behalf
of the team. The number of attended lectures should be more, however the issues
to be created for later lectures were forgotten. The numbers in the table are soley
based on the number of issues created. Although inaccurate to the true
attendence, it is accurate though the issues created. The team will work to add
issues for lectures in the future. 

\section{TA Document Discussion Attendance}

\begin{table}[H]
\centering
\begin{tabular}{ll}
\toprule
\textbf{Student} & \textbf{TA Discussions}\\
\midrule
Total & 4\\
Sumanya Gulati & 4\\
Jennifer Ye & 4\\
Jason Tran & 4\\
Kate Min & 4\\
\bottomrule
\end{tabular}
\end{table}

All members have attended all meetings with the TAs,

\section{Commits}

\wss{For each team member how many commits to the main branch have been made
over the time period of interest.  The total is the total number of commits for
the entire team since the beginning of the term.  The percentage is the
percentage of the total commits made by each team member.}

\begin{table}[H]
\centering
\begin{tabular}{lll}
\toprule
\textbf{Student} & \textbf{Commits} & \textbf{Percent}\\
\midrule
Total & Num & 100\% \\
Name 1 & Num & \% \\
Name 2 & Num & \% \\
Name 3 & Num & \% \\
Name 4 & Num & \% \\
Name 5 & Num & \% \\
\bottomrule
\end{tabular}
\end{table}

\wss{If needed, an explanation for the counts can be provided here.  For
instance, if a team member has more commits to unmerged branches, these numbers
can be provided here.  If multiple people contribute to a commit, git allows for
multi-author commits.}

\section{Issue Tracker}

\wss{For each team member how many issues have they authored (including open and
closed issues (O+C)) and how many have they been assigned (only counting closed
issues (C only)) over the time period of interest.}

\begin{table}[H]
\centering
\begin{tabular}{lll}
\toprule
\textbf{Student} & \textbf{Authored (O+C)} & \textbf{Assigned (C only)}\\
\midrule
Name 1 & Num & Num \\
Name 2 & Num & Num \\
Name 3 & Num & Num \\
Name 4 & Num & Num \\
Name 5 & Num & Num \\
\bottomrule
\end{tabular}
\end{table}

\wss{If needed, an explanation for the counts can be provided here.}

\section{CICD}

\wss{Say how CICD will be used in your project}

\wss{If your team has additional metrics of productivity, please feel free to
add them to this report.}

\end{document}
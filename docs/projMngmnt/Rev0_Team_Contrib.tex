\documentclass{article}

\usepackage{float}
\restylefloat{table}

\usepackage{booktabs}

\title{Team Contributions: Rev 0\\\progname}

\author{\authname}

\date{}

%% Common Parts

\newcommand{\progname}{Software Engineering} % PUT YOUR PROGRAM NAME HERE
\newcommand{\authname}{Team 21, Alkalytics
\\ Sumanya Gulati - gulats10
\\ Kate Min - mink9
\\ Jennifer Ye - yej52
\\ Jason Tran - tranj78} % AUTHOR NAMES                  

\usepackage{hyperref}
    \hypersetup{colorlinks=true, linkcolor=blue, citecolor=blue, filecolor=blue,
                urlcolor=blue, unicode=false}
    \urlstyle{same}
                                


\begin{document}

\maketitle

This document summarizes the contributions of each team member for the Rev 0
Demo.  The time period of interest is the time between the POC demo and the Rev
0 demo.

\section{Demo Plan}
The following objectives have been derived from section 6 of the \href{https://github.com/SumanyaG/Alkalytics/blob/main/docs/Design/MG.pdf}{Module Guide} 
and describe what will be demonstrated during the team’s Revision 0 demonstration.

\begin{itemize}
    \item \textbf{User Authentication}: Ensure users can log in using a test account,
    where authentication is verified and permissions are assigned based on their role.
    \item \textbf{Dashboard Overview}: Display the home page with navigation bars and
    the last uploaded dataset for quick access along with widgets for sample graphs.
    \item \textbf{Data Upload and Transformation}: Allow users to upload a CSV file,
    which is then converted into JSON using a migration algorithm, and store data
    points in the database.
    \item \textbf{Graph Generation}: Enable users to create line, bar, and scatter
    plots with customizable parameters, supporting multiple experiments in a single graph.
    \item \textbf{Data Analysis}: Provide tools for analyzing graphs, assessing key characteristics 
    and patterns and relate them to broader models to measure alignment.
    \item \textbf{Experiment History}: Offer a navigation pane where users can view
    all past experiment data.
    \item \textbf{CRUD Operations on Data Points}: Provide a page where users can
    manually create, read, update and delete (CRUD) uploaded data.
\end{itemize}

\section{Team Meeting Attendance}

\begin{table}[H]
\centering
\begin{tabular}{ll}
\toprule
\textbf{Student} & \textbf{Meetings}\\
\midrule
Total & 5\\
Jason Tran & 5\\
Jennifer Ye & 5\\
Kate Min & 5\\
Sumanya Gulati & 3\\
\bottomrule
\end{tabular}
\end{table}

The period of interest is from December 6, 2024 till January 29, 2025.

\section{Supervisor/Stakeholder Meeting Attendance}

\begin{table}[H]
\centering
\begin{tabular}{ll}
\toprule
\textbf{Student} & \textbf{Meetings}\\
\midrule
Total & 3\\
Jason Tran & 2\\
Jennifer Ye & 3\\
Kate Min & 3\\
Sumanya Gulati & 3\\
\bottomrule
\end{tabular}
\end{table}

The period of interest is from December 18, 2024 till January 29, 2025.

\section{Lecture Attendance}

\begin{table}[H]
\centering
\begin{tabular}{ll}
\toprule
\textbf{Student} & \textbf{Lectures}\\
\midrule
Total & 2\\
Jason Tran & 0\\
Jennifer Ye & 1\\
Kate Min & 0\\
Sumanya Gulati & 1\\
\bottomrule
\end{tabular}
\end{table}

The period of interest is from December 6, 2024 till January 29, 2025.

\section{TA Document Discussion Attendance}

\begin{table}[H]
\centering
\begin{tabular}{ll}
\toprule
\textbf{Student} & \textbf{Meetings}\\
\midrule
Total & 1\\
Jason Tran & 1\\
Jennifer Ye & 1\\
Kate Min & 1\\
Sumanya Gulati & 1\\
\bottomrule
\end{tabular}
\end{table}

The period of interest is from December 18, 2024 till January 29, 2025.

\section{Commits}
Each team member has been working on an individual, dedicated branch for separate features or 
components of the project. Each subsection refers to the commit history of one such branch.
Only currently active branches have been taken into account. 

\subsection{Branch: main} \label{sec:main}

\begin{table}[H]
\centering
\begin{tabular}{lll}
\toprule
\textbf{Student} & \textbf{Commits} & \textbf{Percent}\\
\midrule
Total & 5 & 100\% \\
Jason Tran & 2 & 40\% \\
Jennifer Ye & 0 & 0\% \\
Kate Min & 2 & 40\% \\
Sumanya Gulati & 2 & 40\% \\
\bottomrule
\end{tabular}
\end{table}

It must be noted that as described in Section 7 - Workflow Plan of the 
\href{https://github.com/SumanyaG/Alkalytics/blob/main/docs/DevelopmentPlan/DevelopmentPlan.pdf}{Development Plan},
commits from milestone specific branches are squashed into a single commit when merged into the
main branch which explains the low count. For example, a commit co-authored by Sumanya and Kate 
to add the Module Guide document to the main branch consists of 13 squashed commits.\\

The period of interest for this subsection is from December 6, 2024 till January 28, 2025. 

\subsection{Branch: frontend-feat/add-dashboard-page} \label{sec:dashboard}

\begin{table}[H]
\centering
\begin{tabular}{lll}
\toprule
\textbf{Student} & \textbf{Commits} & \textbf{Percent}\\
\midrule
Total & 5 & 100\% \\
Jason Tran & 3 & 60\% \\
Jennifer Ye & 0 & 0\% \\
Kate Min & 2 & 40\% \\
Sumanya Gulati & 0 & 0\% \\
\bottomrule
\end{tabular}
\end{table}

The period of interest for this subsection is from December 24, 2024 till January 28, 2025. 

\subsection{Branch: frontend-feat/add-table-view-page} \label{sec:tableview}

\begin{table}[H]
\centering
\begin{tabular}{lll}
\toprule
\textbf{Student} & \textbf{Commits} & \textbf{Percent}\\
\midrule
Total & 2 & 100\% \\
Jason Tran & 2 & 100\% \\
Jennifer Ye & 0 & 0\% \\
Kate Min & 0 & 0\% \\
Sumanya Gulati & 0 & 0\% \\
\bottomrule
\end{tabular}
\end{table}

The period of interest for this subsection is from December 24, 2024 till January 28, 2025. The commits 
from January 17, 2025 have not been included in the count as they have already been 
accounted for in subsection \ref{sec:dashboard}.

\subsection{Branch: frontend-feat/add-user-login-auth} \label{sec:login}

\begin{table}[H]
\centering
\begin{tabular}{lll}
\toprule
\textbf{Student} & \textbf{Commits} & \textbf{Percent}\\
\midrule
Total & 2 & 100\% \\
Jason Tran & 0 & 0\% \\
Jennifer Ye & 0 & 0\% \\
Kate Min & 2 & 100\% \\
Sumanya Gulati & 0 & 0\% \\
\bottomrule
\end{tabular}
\end{table}

The period of interest for this subsection is from December 24, 2024 till January 28, 2025. The commit 
from January 17, 2025 has not been included in the count as it has already been 
accounted for in subsection \ref{sec:tableview}. 

\section{Issue Tracker}

\begin{table}[H]
\centering
\begin{tabular}{lll}
\toprule
\textbf{Student} & \textbf{Authored (O+C)} & \textbf{Assigned (C only)}\\
\midrule
Jason Tran & 0 & 4 \\
Jennifer Ye & 7 & 4 \\
Kate Min & 0 & 4 \\
Sumanya Gulati & 4 & 4 \\
\bottomrule
\end{tabular}
\end{table}

The period of interest for this subsection is from December 6, 2024 till January 29, 2025.
Given the distribution of work and the fact that all assigned components were worked on 
individually by team members, issues were not opened by the team members which explains the 
low count.

\section{Pull Requests}

\begin{table}[H]
\centering
\begin{tabular}{lll}
\toprule
\textbf{Student} & \textbf{Pull Requests (O+C)} & \textbf{Percent}\\
\midrule
Total & 6 & 100\% \\
Jason Tran & 3 & 50\% \\
Jennifer Ye & 0 & 0\% \\
Kate Min & 1 & 16.67\% \\
Sumanya Gulati & 2 & 33.33\% \\
\bottomrule
\end{tabular}
\end{table}

The period of interest for this subsection is from December 23, 2024 till January 28, 2025.

\section{CICD}

The initial phase of Continuous Integration (CI) was implemented to the project in the form 
of a \LaTeX build checker that flags any compiling issues caused by incorrect syntax, missing 
elements or more. Additionally, Flake8 - a linter for Python and ESLint - a linter for JavaScript have been
implemented using GitHub Actions.

\end{document}
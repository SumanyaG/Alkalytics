\documentclass{article}

\usepackage{float}
\restylefloat{table}

\usepackage{booktabs}

\title{Team Contributions: Final\\\progname}

\author{\authname}

\date{}

%% Comments

\usepackage{color}

\newif\ifcomments\commentstrue %displays comments
%\newif\ifcomments\commentsfalse %so that comments do not display

\ifcomments
\newcommand{\authornote}[3]{\textcolor{#1}{[#3 ---#2]}}
\newcommand{\todo}[1]{\textcolor{red}{[TODO: #1]}}
\else
\newcommand{\authornote}[3]{}
\newcommand{\todo}[1]{}
\fi

\newcommand{\wss}[1]{\authornote{blue}{SS}{#1}} 
\newcommand{\plt}[1]{\authornote{magenta}{TPLT}{#1}} %For explanation of the template
\newcommand{\an}[1]{\authornote{cyan}{Author}{#1}}

%% Common Parts

\newcommand{\progname}{Software Engineering} % PUT YOUR PROGRAM NAME HERE
\newcommand{\authname}{Team 21, Alkalytics
\\ Sumanya Gulati - gulats10
\\ Kate Min - mink9
\\ Jennifer Ye - yej52
\\ Jason Tran - tranj78} % AUTHOR NAMES                  

\usepackage{hyperref}
    \hypersetup{colorlinks=true, linkcolor=blue, citecolor=blue, filecolor=blue,
                urlcolor=blue, unicode=false}
    \urlstyle{same}
                                


\begin{document}

\maketitle

This document summarizes the contributions of each team member for the final
demonstration and documentation.  The time period of interest is the time
between Rev 0 and the Final documentation.

\section{Team Meeting Attendance}

\wss{For each team member how many team meetings have they attended over the
time period of interest.  This number should be determined from the meeting
issues in the team's repo.  The first entry in the table should be the total
number of team meetings held by the team.}

\begin{table}[H]
\centering
\begin{tabular}{ll}
\toprule
\textbf{Student} & \textbf{Meetings}\\
\midrule
Total & Num\\
Name 1 & Num\\
Name 2 & Num\\
Name 3 & Num\\
Name 4 & Num\\
Name 5 & Num\\
\bottomrule
\end{tabular}
\end{table}

\wss{If needed, an explanation for the counts can be provided here.}

\section{Supervisor/Stakeholder Meeting Attendance}

\wss{For each team member how many supervisor/stakeholder team meetings have
they attended over the time period of interest.  This number should be determined
from the supervisor meeting issues in the team's repo.  The first entry in the
table should be the total number of supervisor and team meetings held by the
team.  If there is no supervisor, there will usually be meetings with
stakeholders (potential users) that can serve a similar purpose.}

\begin{table}[H]
\centering
\begin{tabular}{ll}
\toprule
\textbf{Student} & \textbf{Meetings}\\
\midrule
Total & Num\\
Name 1 & Num\\
Name 2 & Num\\
Name 3 & Num\\
Name 4 & Num\\
Name 5 & Num\\
\bottomrule
\end{tabular}
\end{table}

\wss{If needed, an explanation for the counts can be provided here.}

\section{Lecture Attendance}

\wss{For each team member how many lectures have they attended over the time
period of interest.  This number should be determined from the lecture issues in
the team's repo.  The first entry in the table should be the total number of
lectures since the beginning of the term.}

\begin{table}[H]
\centering
\begin{tabular}{ll}
\toprule
\textbf{Student} & \textbf{Lectures}\\
\midrule
Total & Num\\
Name 1 & Num\\
Name 2 & Num\\
Name 3 & Num\\
Name 4 & Num\\
Name 5 & Num\\
\bottomrule
\end{tabular}
\end{table}

\wss{If needed, an explanation for the lecture attendance can be provided here.}

\section{TA Document Discussion Attendance}

\wss{For each team member how many of the informal document discussion meetings
with the TA were attended over the time period of interest.}

\begin{table}[H]
\centering
\begin{tabular}{ll}
\toprule
\textbf{Student} & \textbf{Lectures}\\
\midrule
Total & Num\\
Name 1 & Num\\
Name 2 & Num\\
Name 3 & Num\\
Name 4 & Num\\
Name 5 & Num\\
\bottomrule
\end{tabular}
\end{table}

\wss{If needed, an explanation for the attendance can be provided here.}

\section{Commits}

\begin{table}[H]
\centering
\begin{tabular}{lll}
\toprule
\textbf{Student} & \textbf{Commits} & \textbf{Percent}\\
\midrule
Total & 40 & 100\% \\
Jason Tran & 5 & 12.5\% \\
Jennifer Ye & 6 & 15\% \\
Kate Min & 8 & 20\% \\
Sumanya Gulati & 20 & 50\% \\
\bottomrule
\end{tabular}
\end{table}

As described in Section 7 - Workflow Plan of the 
\href{https://github.com/SumanyaG/Alkalytics/blob/main/docs/DevelopmentPlan/DevelopmentPlan.pdf}{Development Plan},
in addition to creating separate branches for each milestone which which merged into the main
branch, each team member created a separate branch off of the milestone branch for their 
respective sections. Their commits from their section-specific milestone branches were squashed
into a single commit when merged to the milestone branch. Each of these squashed commits has been counted
singularly in the above table.\\
\newline
Additionally, the number of commits by Sumanya Gulati is over-represented in the count as 12/20 of the
commits are fixup! commits which were used to edit READMEs, upload relevant resources or fix minor 
formatting issues and such. These 12 commits do not reflect her contribution to the actual content of 
each milestone submission.

\section{Issue Tracker}

\begin{table}[H]
\centering
\begin{tabular}{lll}
\toprule
\textbf{Student} & \textbf{Authored (O+C)} & \textbf{Assigned (C only)}\\
\midrule
Total & 103 & 87 \\
Jason Tran & 21 & 18 \\
Jennifer Ye & 4 & 13 \\
Kate Min & 3 & 20 \\
Sumanya Gulati & 62 & 36 \\
\bottomrule
\end{tabular}
\end{table}

\wss{If needed, an explanation for the counts can be provided here.}

\wss{If your team has additional metrics of productivity, please feel free to
add them to this report.}

\section{Pull Requests}

\end{document}
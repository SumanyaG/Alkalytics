\documentclass[12pt]{article}

\usepackage{booktabs}
\usepackage{tabularx}
\usepackage{longtable}
\usepackage{hyperref}
\usepackage{graphicx}
\usepackage[center]{caption}
\usepackage{float}
\captionsetup{justification=centering}

\hypersetup{ bookmarks=true,         % show bookmarks bar?
    colorlinks=true,      % false: boxed links; true: colored links
    linkcolor=red,          % color of internal links (change box color with
    citecolor=green,        % color of links to bibliography
    filecolor=magenta,      % color of file links
    urlcolor=cyan           % color of external links
}

\newcommand{\lips}{\textit{Insert your content here.}}
\input{../Comments}
%% Common Parts

\newcommand{\progname}{Software Engineering} % PUT YOUR PROGRAM NAME HERE
\newcommand{\authname}{Team 21, Alkalytics
\\ Sumanya Gulati - gulats10
\\ Kate Min - mink9
\\ Jennifer Ye - yej52
\\ Jason Tran - tranj78} % AUTHOR NAMES                  

\usepackage{hyperref}
    \hypersetup{colorlinks=true, linkcolor=blue, citecolor=blue, filecolor=blue,
                urlcolor=blue, unicode=false}
    \urlstyle{same}
                                


\title{User Guide for \progname: Alkalytics} 
\author{\authname}
\date{\today}

\begin{document}

\maketitle
\newpage
\tableofcontents
\newpage

\section{Introduction}
\subsection*{Section Overview}
This section provides an orientation to the Alkalytics application, its purpose,
and the scope of this documentation.

Welcome to the Alkalytics User Guide. This comprehensive document provides
complete instructions for installing, configuring, and using all features of the
Alkalytics web application.

\subsection{Purpose}
Alkalytics is designed to:
\begin{itemize}
\item Manage experimental data efficiently
\item Provide powerful analysis tools
\item Generate customizable visualizations
\item Support collaborative research workflows
\end{itemize}

\section{System Overview}
\subsection*{Section Overview}
This section describes the technical architecture, components, and requirements
for running Alkalytics.

\subsection{Architecture}
Alkalytics is a React and TypeScript-based web application that runs locally for
security purposes. The system architecture consists of:

\begin{itemize}
\item \textbf{Frontend}: Built with React and TypeScript
\item \textbf{Backend}: Python processing with Univcorn server
\item \textbf{Database}: MongoDB for data storage
\end{itemize}

\subsection{Detailed Requirements}
\subsubsection*{Description}
Lists all hardware and software requirements for running the application.

\subsubsection{Hardware Requirements}
\begin{table}[H]
    \centering
    \begin{tabularx}{\textwidth}{llX}
        \toprule
        \textbf{Component} & \textbf{Minimum} & \textbf{Recommended} \\
        \midrule
        RAM & 8GB & 16GB \\
        Storage & 10GB & 50GB SSD \\
        Processor & 2 cores & 4 cores \\
        \bottomrule
    \end{tabularx}
\end{table}

\subsubsection{Software Dependencies}
\begin{table}[H]
    \centering
    \begin{tabularx}{\textwidth}{lXl}
        \toprule
        \textbf{Software} & \textbf{Purpose} & \textbf{Version} \\
        \midrule
        Node.js & Frontend runtime & 16.x LTS \\
        MongoDB & Database & 5.0+ \\
        Python & Backend processing & 3.8+ \\
        Univcorn & ASGI server & 0.15+ \\
        \bottomrule
    \end{tabularx}
\end{table}

\section{Installation Guide}
\subsection*{Section Overview}
This section provides complete step-by-step instructions for setting up the
Alkalytics environment.

\subsection{Step-by-Step Installation}
\begin{enumerate}
    \item \textbf{Prerequisite Installation}
    \begin{enumerate}
        \item Install Node.js from \url{https://nodejs.org}
        \item Install MongoDB from \url{https://www.mongodb.com}
        \item Install Python 3.x from \url{https://www.python.org}
    \end{enumerate}
    
    \item \textbf{Repository Setup}
    \begin{verbatim}
    git clone https://github.com/your-repo/alkalytics.git
    cd alkalytics
    \end{verbatim}
    
    \item \textbf{Dependency Installation}
    \begin{verbatim}
    yarn install
    pip install -r requirements.txt
    \end{verbatim}
    
    \item \textbf{Database Configuration}
    \begin{enumerate}
        \item Create data directory:
        \begin{verbatim}
        mkdir /data/db
        \end{verbatim}
        \item Start MongoDB service:
        \begin{verbatim}
        mongod --dbpath /data/db --port 27017
        \end{verbatim}
    \end{enumerate}
    
    \item \textbf{Application Launch}
    \begin{enumerate}
        \item Start backend:
        \begin{verbatim}
        yarn ts-node src/utils/server.ts
        \end{verbatim}
        \item Start frontend:
        \begin{verbatim}
        yarn start
        \end{verbatim}
        \item Launch ASGI server:
        \begin{verbatim}
        univcorn --port 8000 main:app
        \end{verbatim}
    \end{enumerate}
\end{enumerate}

\subsection{Verification}
After installation, verify all components are running:
\begin{enumerate}
    \item Frontend: \texttt{http://localhost:3000}
    \item Backend: \texttt{http://localhost:8000/healthcheck}
    \item Database: Check MongoDB connection on port 27017
\end{enumerate}

\section{User Management}
\subsection*{Section Overview}
This section details the different user roles (Admin and Researcher) and their
respective capabilities within the application.

\subsection{Admin Features}
\subsubsection*{Description}
Administrators have full control over all system functionality including data
management, user configuration, and system settings.


\subsection{Researcher Features}
\subsubsection*{Description}
Researchers can view data, run analyses, and generate reports but have limited
system configuration capabilities.

\section{Web Application Pages}
\subsection*{Section Overview}
This section covers all data handling operations including uploads, processing,
and table management.

\subsection{Upload Process}
\subsubsection*{Description}
The upload functionality allows users to import experimental data in various
formats for analysis.

\subsubsection{Step-by-Step Upload}
\begin{enumerate}
    \item Navigate to Upload Page
    \item Select file type (Experiment Log or Raw Data)
    \item Choose upload method:
    \begin{itemize}
        \item Drag and drop files
        \item Browse file system
    \end{itemize}
    \item Verify file preview
    \item Click \textbf{Upload} button
    \item Monitor progress in notification panel
\end{enumerate}

\subsubsection{File Requirements}
\begin{table}[H]
    \centering
    \begin{tabularx}{\textwidth}{lX}
        \toprule
        \textbf{Requirement} & \textbf{Specification} \\
        \midrule
        File Size & Maximum 10MB per file \\
        Columns & Minimum 5 required fields \\
        Date Format & YYYY-MM-DD \\
        Special Characters & Avoid in header names \\
        \bottomrule
    \end{tabularx}
\end{table}

\subsection{Table View}

\subsubsection{Table Editing}
\begin{enumerate}
    \item Navigate to Experiment Table
    \item Click \textbf{Edit} button to enable editing mode
    \item Modify cell values directly
    \item Use \textbf{Save} button to commit changes
\end{enumerate}

\subsubsection{Column Management}
\begin{itemize}
    \item \textbf{Adding Columns:}
    \begin{enumerate}
        \item Click \textbf{Add Column} button
        \item Enter column name in dialog
        \item Select data type from dropdown
        \item Click \textbf{Add Column} to confirm
    \end{enumerate}
    
    \item \textbf{Removing Columns:}
    \begin{enumerate}
        \item Click \textbf{Remove Column} button
        \item Select column from dropdown
        \item Confirm deletion
    \end{enumerate}
\end{itemize}

\subsubsection{Excel Function Bar}
\begin{enumerate}
    \item Select target rows using checkboxes
    \item Choose destination column
    \item Enter formula (e.g., \texttt{SUM(A,B)})
    \item Click \textbf{Apply} to execute
\end{enumerate}

\subsection{Graph Generation}
\subsection*{Section Overview}
This section provides detailed instructions for creating, customizing, and
exporting data visualizations.

\subsubsection{Detailed Workflow}
\subsubsection*{Description}
The five-step process for generating custom graphs from experimental data.

\begin{enumerate}
    \item \textbf{Select Graph Type}
    \begin{itemize}
        \item Line graph for trends
        \item Bar graph for comparisons
        \item Scatter plot for correlations
    \end{itemize}
    
    \item \textbf{Apply Filters}
    \begin{enumerate}
        \item Choose filter attribute (e.g., "\# of Stacks")
        \item Select filter value from dropdown
        \item Apply date range if needed
    \end{enumerate}
    
    \item \textbf{Set Parameters}
    \begin{itemize}
        \item X-axis: Typically time or independent variable
        \item Y-axis: Measurement or dependent variable
    \end{itemize}
    
    \item \textbf{Customize Display}
    \begin{itemize}
        \item Title: Descriptive graph name
        \item Axis Labels: Clear units of measurement
        \item Range: Manual or automatic scaling
    \end{itemize}
    
    \item \textbf{Generate \& Export}
    \begin{itemize}
        \item Click \textbf{Submit} to render
        \item Use \textbf{Export} button for PNG/PDF
    \end{itemize}
\end{enumerate}

\section{Troubleshooting}
\subsection*{Section Overview}
This section lists common issues, error messages, and their solutions, along
with advanced diagnostic procedures.

\subsection{Common Issues}
\begin{table}[H]
    \centering
    \begin{tabularx}{\textwidth}{lXl}
        \toprule
        \textbf{Issue} & \textbf{Solution} & \textbf{Error Code} \\
        \midrule
        Connection refused & Verify all services are running &
        ERR\_CONN\_REFUSED \\
        Upload timeout & Check file size and network & 408 \\
        Graph rendering failed & Validate data selection & VIS\_ERR\_001 \\
        Database not responding & Restart MongoDB service & DB\_ERR\_503 \\
        \bottomrule
    \end{tabularx}
\end{table}

\subsection{Advanced Diagnostics}
\subsubsection*{Description}
Technical procedures for diagnosing and resolving system issues.

\begin{enumerate}
    \item Check application logs:
    \begin{verbatim}
    cat /var/log/alkalytics/app.log
    \end{verbatim}
    
    \item Verify service status:
    \begin{verbatim}
    systemctl status mongod
    yarn run status
    \end{verbatim}
    
    \item Clear cache if needed:
    \begin{verbatim}
    yarn cache clean
    \end{verbatim}
\end{enumerate}

\section{Appendix}
\subsection*{Section Overview}
Additional reference materials including keyboard shortcuts, FAQs, and contact
information.

\subsection{Keyboard Shortcuts}
\begin{table}[H]
    \centering
    \begin{tabularx}{\textwidth}{lX}
        \toprule
        \textbf{Shortcut} & \textbf{Action} \\
        \midrule
        Ctrl+E & Toggle edit mode \\
        Ctrl+S & Save current table \\
        Ctrl+G & Open graph generator \\
        Ctrl+F & Focus search bar \\
        \bottomrule
    \end{tabularx}
\end{table}

\subsection{Frequently Asked Questions}
\begin{itemize}
    \item \textbf{Q: How to reset my password?}
    \begin{itemize}
        \item Contact admin for password reset
    \end{itemize}
    
    \item \textbf{Q: Can I import SQL databases?}
    \begin{itemize}
        \item Currently only CSV and Excel supported
    \end{itemize}
    
    \item \textbf{Q: Where are my uploaded files stored?}
    \begin{itemize}
        \item In MongoDB under collections: experiments and raw\_data
    \end{itemize}
\end{itemize}

\end{document}